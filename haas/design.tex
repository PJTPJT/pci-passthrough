\section{Virtualization support for HaaS}
% Virtualization support for HaaS include the following items
% CPU idleness detection and processing
% Direct NIC assignment
%   - VFIO
%   - CPU optimization for the baremetal network performance
% Direct interrupt delivery
%   - Posted-interrupt mechanism
%   - Direct timer-interrupt delivery
%     - periodic vs aperiodic timer interrupt
%     - one-shot vs periodic hardware timer interrupt
%     - guest-level access to the PIR page
%     - Eliminate the hardware lock, when the guest accesses
%       the PIR bit
%     - Spurious timer interrupts
% Seamless baremetal virtual machine migration
%   - NIC bonding with the hot plug/unplug operation migration
%   - DTID during the migration
The design goal of \name is to eliminate the hypervisor from
the guest I/O path while achieving bare-metal performance in
the guest. To achieve this \name considers direct device
assignment to the guest using Intel's VT-d support. First, the
guest meets the bare-metal network and disk I/O performance
with direct device assignment. We further apply optimizations
to reduce the CPU utilization by hypervisor. Second, the timer
interrupts are transformed into the posted interrupts, which
are directly delivered to the guest by the logical processor
without causing any VM exits. Finally, we present the
migration of virtual machine with directly assigned devices.

\subsection{CPU Idleness and Processing}
% CPU clock rate and implication to the power saving.


\subsection{Direct PCI Device Assignment}
% Describe the design choice of direct device assignment and
% its benefits.
The hardware-assisted direct device assignment helps the guest
achieve the baremetal I/O performance without the additional
virtualization overhead. Using the VT-x and VT-d support
fulfills our design goal. After proper VMCS and EPT
configuration, the guest gain the control of assigned device
by MMIO/PIO without the help of hypervisor. The VT-x APIC
virtualization permits that the guest writes to the
end-of-interrupt register without an VM exit. With the VT-d
support, the guest performs DMA with the enhanced security and
zero the VM-exit overhead due to the device interrupts by the
posted-interrupt mechanism. Under the normal circumstances,
the guest can not handle physical interrupts without the
hypervisor. When a physical interrupt is delivered to the
guest, it induces the following overheads. First, it triggers
the VM exit and transfer the control to the host. Saving and
loading the execution context between the root and non-root
mode waste the CPU cycles. Second, the host needs to examine
and handle the physical interrupt. If the physical interrupt
is meant for the guest, the host needs to deliver it as the
virtual interrupt upon the next VM entry. Otherwise, the host
handles it and schedules the next VM entry. Third, when the
guest's interrupt handler finishes, it writes to the EOI
register. Such write operation may induce the VM exit. Since
the guest is not aware of the distinction between the physical
and virtual interrupt, it signals the completion of interrupt
in the same way. After the guest handles the virtual
interrupt, its EOI update is normally emulated by the host.
Fourth, the host may need to use CPU cache and reduce the time
to handle the physical interrupt. This introduces the CPU
cache pollution.


\subsubsection{VFIO}
Virtual function I/O is a secure framework for the userspace
device drivers. It does not only support the direct PCI device
assignment to the userspace processes of VM, but also the
platform devices. Why do we need to allow the userspace
programs to gain the control of physical devices? For the
field of high performance computing, the I/O performances has
a great impact on the overall system performance. The
performance congestion comes, When the rate of data being read
is slower than the rate of data being consumed. Or it happens,
when the rate of data being written is slower than the rate of
data being computed and produced.

The VFIO needs to fulfill the three requirements of device
assignment to a userspace process. First, the userspace driver
can access to the device resources such as I/O ports. Second,
the userspace driver can perform the DMA securely. This is
provided by the IOMMU protection mechanism. Third, the device
interrupts is delivered to the device owner in the userspace.
The way how the VFIO fulfills the three requirements and
applies them with QEMU is briefly described below.

First, the VFIO exposes the configuration registers of
physical device as the memory regions. Typically, the device
driver communicates with the device registers by the PIO or
MMIO operations to the designated memory address space. VFIO
would like to expose the address space to the user space. The
VFIO retrieves the device information such as BARs from the
PCI configuration spaces and IRQ. It reconstructs them as
different memory regions in a file. The userspace driver uses
the device file descriptor and offset to access each region
and retrieve the device information. The VFIO decomposes the
physical device to a software interface. Such software
interface is turned into the assigned device by QEMU.
Essentially, the device read and write handler in the QEMU
memory API is forwarded to the VFIO read and write handler.
Nonetheless, accessing some parts of PCI configuration
requires the KVM/QEMU emulation. PCI configuration space is
not handled as memory regions in QEMU. Some of the accesses to
the PCI configuration space is passthroughed directly, while
others, such as MSI, BARs, and ROM, need to be emulated.

Second, the VFIO programs the IOMMU to transfer the data
between the userspace driver and device in a hardware
protected manner. The VT-d IOMMU provides the device isolation
using the per-device IOVA and paging structure. The virtual
virtual address requested by the device is translated to the
physical address through the set of paging structures by
IOMMU. In the case of VM, the address space of assigned device
is embedded within the guest address space. The IOMMU is
programmed to translate such IOVA to the host physical address
which is mapped to the guest address space. Such translation
is both realized and protected by the IOMMU.

Third, the VFIO has a mechanism to describe and register the
device interrupt to signal its userspace driver. When the VM
accesses the device configuration space, it is trapped through
QEMU. QEMU configures the IRQs by the VFIO interrupt ioctls
and sets up the event notifiers between the kernel, QEMU and
guest. When the kernel signals the IRQ to QEMU, QEMU injects
it into the VM. The interrupt signaling is further speeded up
by moving QEMU out of the way. KVM supports both ioeventfd and
irqfd. ioeventfd registers PIO and MMIO regions to trigger an
event notification, when written by the VM. irqfd allows to
inject a specific interrupt to the VM by KVM. Once ioeventfd
and irqfd are coupled together, the interrupt pathway remains
in the host kernel without exiting to the userspace QEMU.
Using the VT-d, the KVM and QEMU is completed removed from the
signaling path way. It enables the direct interrupt delivery
from the assigned device to its VM without a VM exit.

\subsubsection{CPU Optimization}
To ensure the guest have the bare-metal I/O performance and
reduced system CPU utilization, our design utilizes the
following optimization.

First, the direct assigned network card and disk drive under
the VFIO framework removes the host from the forward I/O path.
The guest has the control over the assigned devices and avoids
the virtualization overhead, when accessing the device control
registers and performing the
DMA~\cite{sdm:2018,vt-d:2017,williamson:2016}. After the
assigned device services the request, it delivers the
interrupt to the guest and triggers the VM exits due to the
external interrupts and EOI respectively. Using the VT-d
posted-interrupt support and APIC virtualization from VT-x,
the guest handles the device interrupt and update the EOI
without any VM exit. Thus, the host is completed removed from
the guest I/O path for the passthrough devices. Nonetheless,
utilizing the hardware posted-interrupt support is not enough
to reduce the system CPU utilization, especially when the
guest is idle.

Second, the VM exit due to the HLT instruction is disabled.
When the guest is idle, it issues the privileged HLT
instruction and yields the processor for other processes. It
induces the VM exit and transfers the control to KVM. KVM is
busy waiting for the incoming events before blocking the
virtual processor. This induces the high CPU overhead. There
are two solutions. We can either set the polling delay to 0 or
disable the VM exit due to the HLT instruction. After setting
the delay to 0, the host does not wait in the busy loop but
make a trip to the scheduler, which is costly. The latter
solution is chosen to further prevent the host from
intervening how the guest uses the logical processor. It
results in that the idle guest remains on its processor even
when it is idle. There is no VM exit due to the HLT
instruction. One of side effects of disabling the HLT exiting
is that the guest receives the increase number of interrupts
from the assigned device. The virtualization overhead of
direct interrupt delivery are handled by the posted-interrupt
hardware.

Third, the guest has a dedicated set of logical
processors~\cite{amit:2015}. This is done by pinning the
virtual processors to the isolated logical processors in the
one-to-one relationship. Doing so not only prevents the host
processes from competing with the guest's virtual processors,
but also reduces the degrees of interrupt routing from the
assigned devices.

\subsection{Direct Interrupt Delivery}
% Describe the direct interrupt delivery and its benefits.
Our design uses the hardware-assisted posted interrupt
mechanism and achieves the direct interrupt delivery of
assigned devices and local timers without the intervention of
hypervisor. It reduces the hypervisor CPU utilization and
dedicates the CPU time to the guest. Under the normal
circumstances, the guest can not handle physical interrupts
without the hypervisor. When a physical interrupt is delivered
to the guest, it induces the following overheads. First, it
triggers the VM exit and transfer the control to the host.
Saving and loading the execution context between the root and
non-root mode waste the CPU cycles. Second, the host needs to
examine and handle the physical interrupt. If the physical
interrupt is meant for the guest, the host needs to deliver it
as the virtual interrupt upon the next VM entry. Otherwise,
the host handles it and schedules the next VM entry. Third,
when the guest's interrupt handler finishes, it writes to the
EOI register. Such write operation may induce the VM exit.
Since the guest is not aware of the distinction between the
physical and virtual interrupt, it signals the completion of
interrupt in the same way. After the guest handles the virtual
interrupt, its EOI update is normally emulated by the host.
Fourth, the host may need to use CPU cache and reduce the time
to handle the physical interrupt. This introduces the CPU
cache pollution.


\subsubsection{Posted-Interrupt Mechanism}
VT-d supports the posted-interrupt capability and deliver the
external interrupts directly from the I/O devices and external
controllers without the cost of VM exits and the hypervisor
intervention. Before utilizing such feature, the system
software needs to define the posted interrupt notification
vector. The PIN signifies the incoming external interrupt from
the assigned device is subjected to the posted-interrupt
processing. The processing is achieved by updating the
posted-interrupt descriptor dynamically. When the VMCS is
actively used by the logical processor in the non-root mode,
it is prohibited to update its data structures. The PID is the
exception. Nonetheless, there is one requirement that the PID
modifications must be done using locked read-modify-write
instructions. Here is another benefit of posted-interrupt
support. When the virtual processor is scheduled on another
vCPU, the VMM can co-migrate its interrupts from the assigned
devices by setting the corresponding bits in posted-interrupt
register of PID.

The posted-interrupt support is accomplished in three general
steps. First, the VMM programs the interrupt-remapping
hardware with the mapping between the external interrupt and
virtual interrupt. Second, when the external interrupt is
delivered to the interrupt-remapping hardware, it sets the
outstanding bit and corresponding bit of virtual interrupt in
the posted-interrupt register of PID. It generates the PIN.
The IOAPIC delivers the PIN to the appropriate LAPIC. Third,
PIN notifies the logical processor that it is the
posted-interrupt event. The logical processor starts the
posted interrupt processing and delivers the virtual interrupt
without any VM exit.

The posted-interrupt processing is described in the following
steps. First, when the external interrupt is delivered to the
guest's processor, it is acknowledged by the LAPIC. LAPIC
provides the processor core the interrupt number. Second, if
the physical interrupt is equal to the PIN, the logical
processor starts the posted interrupt processing. Third, the
processor clears the outstanding notification bit from the
posted-interrupt descriptor. Fourth, the processor
acknowledges the EOI. Fifth, the processor updates the vIRR by
synchronizing it with the PIR. Sixth, the processor acquires
the next request virtual interrupt. It updates RVI by the
maximum of previous RVI and highest index of bits set in PIR,
before it clears PIR. Seventh, the processor evaluates the
pending virtual interrupt. Eighth, the processor delivers the
virtual interrupt.

\subsubsection{Shared-PID DTID} \label{subsubsec:shared_pid_dtid}
% Working on the shared-pid DTID.
% Direct timer-interrupt delivery
%   - periodic vs aperiodic timer interrupt
%   - one-shot vs periodic hardware timer interrupt
%   - guest-level access to the PIR page
%   - Eliminate the hardware lock, when the guest accesses
%     the PIR bit
%   - Spurious timer interrupts
Both the host and guest requires to receive timer interrupts
from the LAPIC or virtual APIC respectively. If the timer
interrupt is delivered to the processor core in the root mode,
the logical processor services the timer interrupt through the
host's timer interrupt handler. Then, the host sets up the
next timer event and execute the previously scheduled work
from the bottom halves. If the timer interrupt is delivered,
while the guest is running, it induces the VM exit. The host
handles the timer interrupt and injects the virtual timer
interrupt to the guest. When the guest receives the virtual
timer interrupt, it services the timer interrupt and set up
the next timer event by updating the LAPIC timer initial count
register through the x2APIC interface. This triggers the
MSR-write VM exit and the control is transfer to the host. The
host helps the guest to set up its next timer by the $hrtimer$
subsystem. If the interrupt is meant for the guest, the guest
should not pay the additional price due to the interrupt
processing by the host.

The task is to transform the timer interrupts into the virtual
interrupt and deliver through the posted-interrupt mechanism.
Since the virtual timer interrupt is injected through the
posted-interrupt mechanism, the corresponding bit in the
posted-interrupt register and outstanding notification bit
need to be set beforehand. When the logical processor receives
the posted interrupt notification, it activates the
posted-interrupt processing and delivers the timer interrupt
as the virtual timer interrupt. The transformation is done by
the two procedures. First, the ON bit and timer-interrupt bit
in the PIR are set by the host. Second, the LAPIC timer on the
guest's dedicated core is configured to deliver the PIN. When
the LAPIC timer is up, it delivers the PIN and activates the
posted-interrupt processing and virtual-interrupt delivery.


\subsection{Seamless VM Migration}
The direct device assignment makes it difficult to migrate the
guest to its destination~\cite{zhai:2008}. After the VM
migration, it is possible that the previously-assigned devices
may not be available at the destination. Even if the assigned
device is available, the internal state of device may not be
readable or still on its way to the destination. The host at
the destination has a hard time to passthrough the device
without the device-specific knowledge. Moreover, some devices
have the unique hardware information that cannot be
transferred, such as the MAC address of network interface
card.


\subsubsection{NIC Bonding Between the Assigned and Virtual NIC}
In this section we describe the mechanism of migrating a guest
with direct NIC assignment. The guest is configured with
virtio network device backed by the vhost driver and a
passthrough network interface card~\cite{zhai:2008}. For the
purpose of simplicity, we assume that the guest has one
assigned network device. The prototype overcomes the challenge
of migrating NIC assigned guest by the following strategy. It
uses the Ethernet bonding driver to direct the network traffic
between the assigned and virtual NIC. The migration procedure
is divided into two parts. During regular operation of guest,
the assigned NIC is used for higher network bandwidth. Before
the migration, the host uses the bonding driver and shifts the
network traffic from the assigned NIC to the virtual NIC. It
hot unplugs the assigned NIC and starts the migration. After
the guest resumes at the destination, the destination host hot
plugs the assigned NIC and switches the network traffic from
virtual NIC to assigned NIC.
%It announces the MAC address of assigned NIC by sending
%the gracious ARP packets to its local network before
%shifting the network traffic back to the
%assigned NIC.

% Insert the Guest's network interface configuration.
%
% Seamless VFIO hotplug at the destination. This is
% about the QEMU modification.
% Explain why hotplugging of NIC at the destination cause the
% network downtime?
% Three general steps:
%   - construct a QEMU software object that represent the
%     passthrough NIC. This is done during migration.
%   - realize the QEMU software object. This is done during
%     the migration.
%   - configure the QEMU software object, when the VM wakes
%     up. This is done when the VM wakes up.


% COMMENT
% - What are the components in play here? We have QEMU, VFIO
%   and guest. We can structure the design and flow of nic
%   bonding and migration with these comopnents in mind. We
%   also have a figure for the NIC bonding and migraiton. It
%   would help too.
% - We may need to explain the reason why we choose to use the
%   active-backup mode among other different modes such as
%   round robin, transmission-load balancing or adaptive load
%   balancing.
% - The active backup mode only handles the outgoing traffic
%   instead of the incoming traffic. For the outgoing traffic,
%   the bonding driver deposits the outgoing packet requests
%   to the target outgoing ring buffer with the respective to
%   the assigned or virtual NIC. For the incoming traffic, it
%   uses the gratuitous ARP to redirect the incoming traffic.
% - We may need to explain the reason why we would like to
%   shift the network traffic before hot unplugging. What if
%   we do not shift the network traffic and hot unplug the
%   assigned NIC. In the other words, how does shifting the
%   traffic helps with the objective of exitless guest?
% - We may want to describe what the "hot" plug/unplug mean
%   conceptually. It is a bit like plug-n-play device without
%   rebooting the machine. Currently, QEMU supports the hot
%   plug/unplug capabilities.
% - We may need to first highlight what was the problem we
%   have encountered when hot plugging the assigned device at
%   the destination, so the QEMU modification would make
%   sense. In the other words, the long time we spend inside
%   the main event loop on adding a I/O device, the longer it
%   takes to pause the guest. This is the QEMU implementation
%   restriction.Then, we would explain the QEMU modification
%   which helps to resolve the network service downtime during
%   the unmodified hotplugging event. Probably just need to
%   restructure the paragraphs a bit.
% - What would be the benefit to hide the first two steps of
%   hot plug during the migration? Or will it introduce the
%   addition migration downtime at all? If we do the first two
%   steps of configuration, the effects are taken at the
%   destination. It does not really seem to affect the QEMU
%   migrating the memory pages and CPU context from the source
%   to the destination.
% - We may want to mention when would be a good time to hot
%   unplug the assigned NIC, which is still performing the
%   network I/O. In the other words, what is the algorithm to
%   determine if it is a right time to hot unplug the NIC
%   device with the minimal network service downtime.
% - We may want to explain the concept of hot plug and unplug.
%   For the hot plug operation, the host transfers the NIC
%   ownership to the guest. For the hot unplug operation, the
%   host re-acquires the assigned NIC back. It signifies the
%   host still has the high order of device controls than the
%   guest. The guest is permitted to own the device and
%   achieves the baremetal network I/O performance.

Linux provides bonding driver to present multiple network
interfaces into a single logical interface. In \na the virtual
NIC and the assigned NIC are enslaved in active-backup mode
where only one of the interfaces is active at any time. When
any of the active interface fails, one of the slave interfaces
becomes active. Before the migration is initiated, the network
traffic is gradually shifted to the slave interface using ARP
packets before hot unplugging the assigned NIC. Once the
assigned NIC is hot unplugged, QEMU issues migrate command.
After the migration is completed, once the guest resumes on
the destination, the NIC device is hot plugged.

The hot plug mechanism of assigned NIC consists of the
following three steps. First, QEMU prepares a software object
that represents the passthrough NIC. It then realizes the QEMU
software object by getting a copy of configuarion space from
the NIC device. Finally, it resets the software NIC object and
setup the BAR and interrupt forwarding. The first and last
steps happen in QEMU main event loop during which the guest
remains paused. As a result, the guest experiences downtime
during hot plug operation. In \na, to mitigate the downtime
due to hot plug operation on the destination host, the first
two steps are executed during migration. QEMU allows to setup
and realize the software object during migration. However, the
BAR and interrupts can be setup only after resuming the guest.
Hence, we eliminate the downtime caused during the setup of
software NIC object phase.


\subsubsection{Migration with Shared-PID DTID}
% Working on DTID during the migration.
% Before the migration, we tear down the DTID.
% After the migration, we build the DTID.


