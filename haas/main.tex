%%%%%%%%%%%%%%%%%%%%%%%%%%%%%%%%%%%%%%%%%%%%%%%%%%%%%%%%%%%%%%%%%%%%%%%%%%%%%%%%
% Template for USENIX papers.
%
% History:
%
% - TEMPLATE for Usenix papers, specifically to meet requirements of
%   USENIX '05. originally a template for producing IEEE-format
%   articles using LaTeX. written by Matthew Ward, CS Department,
%   Worcester Polytechnic Institute. adapted by David Beazley for his
%   excellent SWIG paper in Proceedings, Tcl 96. turned into a
%   smartass generic template by De Clarke, with thanks to both the
%   above pioneers. Use at your own risk. Complaints to /dev/null.
%   Make it two column with no page numbering, default is 10 point.
%
% - Munged by Fred Douglis <douglis@research.att.com> 10/97 to
%   separate the .sty file from the LaTeX source template, so that
%   people can more easily include the .sty file into an existing
%   document. Also changed to more closely follow the style guidelines
%   as represented by the Word sample file.
%
% - Note that since 2010, USENIX does not require endnotes. If you
%   want foot of page notes, don't include the endnotes package in the
%   usepackage command, below.
% - This version uses the latex2e styles, not the very ancient 2.09
%   stuff.
%
% - Updated July 2018: Text block size changed from 6.5" to 7"
%
% - Updated Dec 2018 for ATC'19:
%
%   * Revised text to pass HotCRP's auto-formatting check, with
%     hotcrp.settings.submission_form.body_font_size=10pt, and
%     hotcrp.settings.submission_form.line_height=12pt
%
%   * Switched from \endnote-s to \footnote-s to match Usenix's policy.
%
%   * \section* => \begin{abstract} ... \end{abstract}
%
%   * Make template self-contained in terms of bibtex entires, to allow
%     this file to be compiled. (And changing refs style to 'plain'.)
%
%   * Make template self-contained in terms of figures, to
%     allow this file to be compiled.
%
%   * Added packages for hyperref, embedding fonts, and improving
%     appearance.
%
%   * Removed outdated text.
%
%%%%%%%%%%%%%%%%%%%%%%%%%%%%%%%%%%%%%%%%%%%%%%%%%%%%%%%%%%%%%%%%%%%%%%%%%%%%%%%%

\documentclass[letterpaper,twocolumn,10pt]{article}
\usepackage{main}
\usepackage{nameref}
\usepackage{graphicx}

% to be able to draw some self-contained figs
%\usepackage{tikz}
\usepackage{amsmath}

% Kartik: Latex shortcut definitions for figures and common commands

% Latex shortcut definitions for figures and common commands

% TODO: Placeholder names subject to change. 
% The ~ symbol adds a hard space after the name
%\newcommand{\fullname}{{Baremetal Virtual Machine~}} 
%\newcommand{\acro}{BVM} 
%\newcommand{\name}{{\acro~}}
\newcommand{\fullname}{{ITRI HaaS OS~}} 
\newcommand{\sna}{IHO} 
\newcommand{\na}{{\sna~}}

\newcommand{\mycomment}[1]{{}}

%\renewcommand{\baselinestretch}{1}
%\setcounter{secnumdepth}{3}

\newcommand{\figh}[3]
{
\begin{figure}
\centerline{\includegraphics[height=#2cm]{fig/#1.pdf}}
%\centerline{\includegraphics[height=#2]{fig/#1.eps}}
\caption{\label{fig:#1} #3}
\vspace{-0.1in}
\end{figure}
}

\newcommand{\figw}[3]
{
\begin{figure}[t]
\centerline{\includegraphics[width=#2cm]{fig/#1.pdf}}
%\centerline{\includegraphics[width=#2]{fig/#1.eps}}
\caption{\label{fig:#1} #3}
\vspace{-0.1in}
\end{figure}
}

\newcommand{\fighw}[3]
{
\begin{figure}
\centerline{\includegraphics[height=17cm,width=22cm]{fig/#1.pdf}}
%\centerline{\includegraphics[height=#2]{fig/#1.eps}}
\caption{\label{fig:#1} #3}
\vspace{-0.1in}
\end{figure}
}

\newcommand{\figwtwocol}[3]
{
\begin{figure*}[t]
\centerline{\includegraphics[width=#2cm]{fig/#1.pdf}}
%\centerline{\includegraphics[width=#2]{fig/#1.eps}}
\caption{\label{fig:#1} #3}
\end{figure*}
}

\newcommand{\minifigw}[2]
{
\centerline{\includegraphics[width=#2cm]{fig/#1.pdf}}
%\centerline{\includegraphics[width=#2]{fig/#1.eps}}
}



%-------------------------------------------------------------------------------
\begin{document}
%-------------------------------------------------------------------------------

%don't want date printed
\date{}

% make title bold and 14 pt font (Latex default is non-bold, 16 pt)
\title{\Large \bf Virtualization for Bare-Metal Cloud Computing}

%for single author (just remove % characters)
\author{
{\rm Your N.\ Here}\\
Your Institution
\and
{\rm Second Name}\\
Second Institution
% copy the following lines to add more authors
% \and
% {\rm Name}\\
%Name Institution
} % end author

\maketitle

% BUG: watch out the bbl generation

\begin{abstract}

--------Prof. Chiueh'ss abstract
Bare metal cloud service is an emerging form of cloud service in which users 
rent physical servers from a cloud operator because they want to make full 
use of the server’s underlying hardware without paying any virtualization overhead. 
Building a bare metal cloud service management system is technically challenging 
because of the constraint that no software agent could be installed on the 
physical servers to be rent out. Such a constraint is particularly limiting 
when it comes to the support for migration of physical machine state and 
performance monitoring for applications running on physical machines. 
The ITRI HaaS OS or IHO, is a bare metal cloud service management system 
that removes this constraint by installing on each physical server a 
single-VM virtualization hypervisor, which affords the user complete and 
direct access to all the devices on the server, and at the same time 
significantly enhances each server’s serviceability and manageability.

--------Our abstract -- TODO: merge above -----

%Motivation
Hardware-as-a-service (HaaS) enables customers to rent physical
machines on cloud platforms to execute their 
applications with near bare-metal performance.
% Problem
However, traditional hypervisor platforms used to host system virtual machines (VMs) 
are heavyweight and impose significant overheads 
in interrupt and I/O processing, making them unsuitable for use in HaaS platforms.
On the other hand, native execution of OS and applications limits 
a cloud provider's ability to migrate customer workloads for system maintenance
and failure recovery.
% Our contributions
In this paper, we propose the \fullname (\acro) to reduce
key virtualization overheads on HaaS platforms.
\name enables a traditional system VM to run atop a thin hypervisor 
and achieve near baremetal performance while retaining the 
administrator's flexibility to live migrate the VM.
A key feature of \name includes
direct delivery of timer and network interrupts to the VM without VM Exits, 
thus eliminating hypervisor-level emulation overheads in interrupt delivery.
% Implementation summary
We describe a prototype implementation on the KVM/QEMU hypervisor
that leverages Intel VT-d hardware support to achieve direct network and timer 
interrupt delivery with no hypervisor intervention, while
supporting live VM migration.
\end{abstract}


\section{Introduction}

% Introduction motivates the readers with the following aspects.
% - Bare-metal cloud or hardware as a service (HaaS)
% - Server support for HaaS
% - Network support for HaaS
% - Apply the virtualization to enhance the manageability of
%   servers in a HaaS
% - Single-VM virtualization requirements
%   - Direct device assignment for all PCIe devices
%   - Direct interrupt delivery
%   - Migration of bare-metal server
%   - VM introspection for the security and better visibility
\mycomment{
Infrastructure as a service (IaaS), which was popularized by AWS's EC2 service~\cite{ec2}, has evolved and morphed into multiple forms over the last decade.
The basic compute unit for IaaS began as a {\em virtual machine} (VM), which represents a slice of a physical machine carved out by a hardware-abstraction-layer, called the hypervisor.
A {\em container} is another basic compute unit, where a common operating system (OS) delimits the addressable system resources (or namespaces) for a group of processes and enforces usage limits.
A more recent IaaS compute unit is a {\em function}, which comes with a complete operating environment consisting of an OS and a middleware layer, and is created on demand.  
}

%Lately, to avoid multi-tenancy and security issues, a physical machine itself is 
%treated as a basic compute unit in {\em bare-metal cloud service}~\cite{bms-wiki}.
%or {\em hardware-as-a-service} (HaaS). 
%Infrastructure as a service (IaaS), which was popularized by AWS's EC2 service, has evolved and morphed into multiple forms over the last decade.
%In the beginning, the basic compute unit for IaaS was a {\em virtual machine}, which represents a slice of a physical machine carved out by a hardware-abstraction-layer hypervisor.
%Then the basic compute unit could also be a {\em container}, which is pre-configured with an operating system and corresponds to a piece of a physical machine delimited by that OS.
%%A more recent option for IaaS's basic compute unit is a {\em function}, which comes with a complete operating environment constsing of an OS and a middleware layer, and is created on demand.  
%Lately, even a physical machine could serve as the basic compute unit. This type of IaaS is known as {\em bare-metal cloud service} or {\em hardware as a service} (HaaS). 
%In the past three years, we have been developing a HaaS operating system called {\em ITRI HaaS OS} or \sna.  
%The focus of this paper is on \sna's virtualization support that enhances the manageability and serviceability required of a modern bare-metal cloud service. 


Conventional multi-tenant cloud services~\cite{ec2,azure,gcp} enable
users to rent virtual machines (VMs) or containers to scale 
up their IT infrastructure to the cloud. However, virtualization
introduces both performance overheads and security concerns
arising from co-located workloads of other users.
To address this concern, cloud operators 
such as  IBM SoftLayer~\cite{softlayer} and Oracle~\cite{oracle},
have begun to offer bare-metal cloud service, or Hardware-as-a-Service (HaaS),
%In the case of traditional multi-tenant IaaS, cloud operators own and manage 
%the physical machines, which are shared among multiple users.
%In contrast, bare-metal cloud operators, 
which allow users to rent dedicated  physical machines.
HaaS clouds enables users combine the benefits of 
scaling up their operations in the cloud with having dedicated 
hardware; users are assured stronger isolation than multi-tenant clouds and 
bare-metal performance for critical workloads 
such as high-performance computing, big data analytics, and AI.
%Other use cases of bare-metal cloud services include a preferred hypervisor 
%or OS that is not supported by cloud operators or special hardware for which virtualization 
%is not sufficiently mature, such as 
%GPUs, SoC-based micro-servers, and application-specific FPGA accelerators.

However, common management functions available on multi-tenant clouds,
such as live migration and introspection-based 
application performance management, are difficult to 
duplicate on HaaS servers, because HaaS providers typically 
do not install any software on these dedicated servers.

To address this manageability gap of existing HaaS platforms, 
we have been developing a HaaS management system, called 
%TODO: uncomment later
%\fullname (\sna) 
IHO,
with the goal of enhancing the manageability and serviceability 
of bare-metal cloud services.
The focus of this paper is on IHO's virtualization support
in the form of a specialized hypervisor, called the {\em Single VM
hypervisor (\sna)}.


\figwtwocol{newarch}{14}{
The \na runs as a thin shim layer
to provide the HaaS VM with dedicated CPU cores, memory, and I/O devices.
All hardware interrupts, including device, local timers, and IPIs
are directly deivered to HaaS VM without hypervisor intervention.
\na coordinates with a HaaS agent in the guest to provide
manageability services. 
}

Figure~\ref{fig:newarch} shows the high-level
architecture of \sna, which runs as a thin hypervisor on each
physical server. \na is optimized to run one VM, called the 
{\em HaaS VM}\footnote{The \na could conceptually be an extension of a physical server's trusted BIOS.}, 
on which  a user install a preferred OS and applications. 

Unlike traditional hypervisors, which are designed to limit and control a VM's 
access to physical resources, \na is designed to maximize the HaaS VM's 
access to physical hardware.
During normal execution, the \na allows 
the HaaS VM to directly interact with physical I/O devices and processor hardware
without the hypervisor's intervention, 
as if it runs directly on a physical server with near bare-metal performance.
\na primarily provides value-added manageability 
features of conventional clouds, such as live
migration, introspection, and performance monitoring.
A small HaaS agent in the guest (a self-contained kernel module)
transparently coordinates these services with the \sna.
Specifically, \na provides the following features for a HaaS VM. 

%Bare-metal cloud operators provide a user with a physical data center instance (PDCI), which is 
%composed of a set of physical machines connected in a way specified by the user. 
%In the past three years, we have been developing a HaaS operating system called 
%
%
%
%
%A HaaS user or tenant makes a HaaS service request to \na by specifying a PDCI, which consists of 
%The HaaS offerings from cloud operators such as IBM (SoftLayer) and Oracle provide a user a physical data center instance (PDCI), which is composed of a set of physical machines connected in a way specified by the user. HaaS users prefer physical machines to virtual machines primarily because they want to make the best of the underlying hardware resources for workloads that do not need the flexibility afforded by virtualization, such as HPC computation, big data analytics or AI training.
%Other HaaS use cases include that users have a preferred hypervisor or operating system which is not supported by cloud operators, and 
%that users need special hardware for which virtualization is not sufficiently mature, such as ARM SOC-based micro-server and GPU/FPGA cluster.
%
%In the case of IaaS, cloud operators own and manage the physical machines.
%In contrast, for HaaS, cloud operators own the physical machines but users manage them. 
%This way, HaaS users are still able to enjoy the multiplexing benefits of cloud computing that are due to sharing of hardware and facilities.
%A HaaS user or tenant makes a HaaS service request to \na by specifying a PDCI, which consists of 
%\begin{itemize} 
%\parskip 0mm
%\itemsep 0mm
%\item A set of physical servers, each with its CPU/memory/PCIe device specification, and configurations on the BIOS, and PCI devices,
%
%\item A set of storage volumes that exist in local or shared storage, and are attached to the servers,
%
%\item A set of IP subnets that describe how the servers are connected with one another and to the Internet, and  
%
%\item A set of public IP addresses to be bound to some of the servers facing the Internet, and their firewall policies. 
%
%\end{itemize}
%\na processes each PDCI request by first making corresponding allocations for server, network and storage resources, and 
%then setting up the PDCI's required network connectivity.  Because a HaaS operator cannot install any agent software on the physical servers rented out on a PDCI, the only way for \na to programmatically build a virtual network that meets a PDCI's IP subset specification is to leverage the VLAN and VXLAN capabilities in modern network switches and routers by properly configuring them according to the network connectivity specification.  
%Moreover, \na allows a HaaS tenant to {\em remotely} check, configure, and update the firmware on the physical servers, as well as install the desired operating systems and applications 
%on them, in a way that minimizes human errors and the adverse side effects that come with these errors. 
%Finally, \na enables a HaaS tenant to monitor the hardware status of the physical servers and the network traffic among them with full visibility, without revealing anything associated with other co-located tenants. 
%
%
%For the HaaS use case in which a tenant installs an operating system (Linux or Windows) rather than a 
%proprietary hypervisor on the physical servers of its PDCI, 
%

\begin{itemize} 
%\setlength\itemsep{-0.04in}
\parskip 0mm
\itemsep 0mm

\item {\bf Direct access to dedicated hardware:} 
\na assigns a HaaS VM with 
dedicated hardware resources, namely 
physical CPUs, memory, local APIC timers, and passthrough I/O devices,
which the VM can access directly 
during its normal execution without hypervisor intervention. 
The \sna, however, maintains its ability to
regain control over physical resources when needed, such 
before live migration of the HaaS VM.


\item {\bf Direct delivery of all interrupts:}
\na enables a HaaS VM to directly receive all hardware interrupts from 
its assigned PCIe devices and CPU cores (for timer interrupts and IPIs) 
without interception and emulation by the hypervisor.
Direct interrupt delivery to the HaaS VM greatly
reduces interrupt processing latency, which is imoportant for running 
latency-critical applications on bare-metal clouds.
In contrast, existing approaches~\cite{vfio,postedinterrupt,amit:2015,tu:2015}
focus only on direct delivery of PCIe device interrupts.


\item {\bf Seamless live migration:}
\na provides on-demand live migration of a HaaS VM having direct access to 
physical network devices and local APIC timers. 
%through coordinated switching to para-virtual I/O during migration.
\na seamlessly disables the HaaS VM's physical hardware access at the 
source machine before migration and 
re-establishes access at the destination after migration,
with minimal disruption to the VM's liveness and performance.
Unlike existing live migration~\cite{vfio-live-migration,blmvisor-journal,ondemand} 
approaches for VMs with pasthrough 
I/O access, \na does not require device-specific state capture and migration code 
nor does it require the hypervisor to trust the guest OS. 
%Yet \na matches the liveness and performance of traditional live migration.

\item {\bf Bare-metal performance:}  
To match bare-metal performance, \na implements mechanisms
for reducing the its CPU, I/O, and memory footprint
during runtime and minimizing interference with HaaS VM's execution.
\na has a small execution footprint, currently one 
CPU core and around 100MB of RAM, with room for further reductions.
Even with passthrough I/O support~\cite{intelvtd-paper,intelvtd-manual}, 
VM exits can still lower the I/O throughput and increase CPU usage of the host.
\na includes mechanisms to systematically eliminate
most VM exits for a HaaS VM to match bare-metal I/O performance and CPU utilization.
Other than during startup and live migration, the \na is not involved in the
HaaS VM's normal execution.

\end{itemize}

%With such a hypervisor installed on each physical server, \na can monitor each 
%server's internal user activities using VM introspection, and seamlessly migrate the user state 
%on the server from one physical machine to another. 

The rest of this paper is organized as follows.
We first describe the design and implementation of \sna,
specifically the detailed description of the above features.
Next, we evaluate the performance of our \na prototype
on the Linux and KVM/QEMU platform.
Finally, we discuss related work followed by conclusions.



\section{Related Work}
% Baremetal service
% - IBM SoftLayer
% - Oracle
% - Zenlayer
% - Vultr
In recent years, virtualization overheads and security concerns on 
multi-tenant IaaS platforms have led to the rise of a 
number of bare-metal or HaaS~\cite{softlayer,oracle,zenlayer,vultr,m2}
clouds which provision dedicated physical machines for customers
to provide native execution and I/O performance.
However, the absence of a virtualization layer 
limits the cloud provider's ability to manage, monitor, 
and live migrate~\cite{clark:2005,postcopy-osr}
customer workloads, leaving the customer responsible for 
these functions.
% Reducing virtualization overheads
On the other hand, virtualized IaaS cloud platforms~\cite{gcp,azure,ec2}
retain the cloud provider's ability to manage and live migrate customers'
VMs but also introduce overheads  due to
hypervisor-level emulation of hardware access by guests. To mitigate I/O-related 
virtualization costs, customers can provision VMs with 
direct device assignment~\cite{intelvtd}. 
However, doing so entails sacrificing the ability 
to live migrate VMs for load balancing and fault-tolerance
due to the difficulty of migrating the state of physical I/O devices.
\na aims to  support bare-metal server performance
while retaining the benefits of virtualization via a 
thin hypervisor.

%TODO container/process migration cannot migrate the OS state
% and limit the type of processes that can be migrated, such as 
% those without active network connections.
{\bf Live migration of bare-metal server:} 
On-demand virtualization~\cite{ondemand} presented an approach
to insert a hypervisor that deprivileges the OS at the 
source server just before migration and re-privileges it
at the destination after migration. However, this approach 
requires the hypervisor to fully trust the OS being migrated, 
the OS having full access to the hypervisor's memory and code.
Such lack of isolation rules out the use of many hypervior-level
cloud services that require strong isolation, such as VM introspection
and SLA enforcement. In contrast, with \na, the guest OS 
cannot access or modify hypervisor code or memory.
BLMVisor~\cite{blmvisor,blmvisor-journal} proposes live migration of VMs 
having direct access to physical devices by having the hypervisor
regulary intercept and capture low-level physical device states
during normal execution; the captured state is then reconstructed
at the destination after migration. This approach requires
implementing device-specific state capture and reconstruction code 
for each type of physical device, besides active 
interception of these operations by the hypervisor. 
In contrast, \na is device agnostic and does not require
the hypervisor to intercept guest I/O accesses.
Finally, unlike the above aproaches,
\na support direct delivery of timer interrupts 
to the guest OS without hypervisor intervention, besides supporting
live migration of guests having direct timer access.
\na seamlessly tears down the VM's
physical device and timer dependencies  at the source server before live migration and 
re-establishes these dependencies at the destination server after migration
with minimal disruption of the VM's execution.


{\em Reducing virtualization overheads:}
A number of techniques have been proposed to reduce virtualization
overheads in interrupt delivery by eliminating VM Exits to the hypervisor.
ELI~\cite{amit:2015} and DID~\cite{tu:2015} presented techniques
for direct delivery of I/O interrupts to the VM. 
Intel VT-d~\cite{intelvtd-paper,intelvtd-manual} supports a posted interrupt delivery mechanism~\cite{postedinterrupt}.
However, these techniques do not fully eliminate hypervisor's role
in the delivery of device interrupt.
Specifically, the hypervisor must intercept and deliver device interrupts
(using either virtual interrupts or inter-processor interrupts)
when the target VM's VCPU is not scheduled on the CPU
that receives the device interrupt. Further, idling guest VCPUs
waiting for I/O completion trap to the hypervisor, where a well-intended
halt-polling optimizations can inadvertently end up increasing CPU utilization.
In contrast, \na eliminates these virtualization
overheads in device interrupt delivery by not only leveraging
Intel VT-d posted interrupts, but also by isolating  hypervisor execution 
to a single physical core, and ensuring that physical 
cores assigned to the VM remain in guest (non-root) 
mode whenever the guest idles while waiting for I/O completion.

Furthermore, a key distinction of \na lies in LAPIC timer
access and timer interrupt processing by the VM.
None of the existing techniques eliminate hypervisor overheads 
when a VM accesses its LAPIC timer and nor do they 
support direct delivery of timer interrupts to the guest OS.
\na does so for the single-VM virtualization case by a novel use of Intel VT-d posted 
interrupt mechanism that allows the guest to directly set timer events
on LAPIC and receive timer interrupts without any VM Exits.

%TODO: Summarize Jailhouse

\mycomment{
\subsection{Jailhouse}
By far, Xen and KVM are the two Linux de facto hypervisors.
The are versatile and cooperate with QEMU to virtualize the
entire system. They utilize the modern hardware-assisted
virtualization and match the bare-metal CPU, memory and I/O
performance with the significantly reduced CPU overhead.
Nonetheless, there are some areas that these two
general-purpose solutions need further improvements. One such
area deals with the real time applications. A real-time
application needs to meet the minimal response time and/or the
worst latency. For example, the vehicular break system needs
to meet the minimal response time , when the driver presses
the break. Failing the requirement leads to a detrimental
consequence.

The jailhouse hypervisor is a partitioning hypervisor that
runs on the bare-metal and works closely with the
Linux\cite{sinitsyn:2015, ramsauer:2017}. It is not trying to
be the full-fledged KVM. Its responsibility is to partition
and assign the available hardware resource to the guests and
prevent a guest from interfering with the jailhouse or another
guest. Each guest has its own set of dedicated hardware
resource and do not share them. In the other words, there is
no overcommitment of resources. Moreover, the jailhouse is an
example of asymmetric multiprocessing design, which treats one
processor core differently from another. For example, one
processor can access the hard disk, while another accesses the
serial port. The jailhouse uses this design to create an
isolated environment, called a cell. When the jailhouse boots
up, it creates a Linux cell or root cell, containing all the
processor cores, memory and hardware resources. Before the
jailhouse boots up a guest, it creates a new cell and allocate
the requested hardware resources to the guest or inmate. This
set of hardware resources is dedicated to the guest. This
partitioning design indicates the jailhouse does not emulate
the devices or manage resources for the guests.

To passthrough the devices, the jailhouse requires the
hardware-assisted virtualization. On the x86, it is the VT-x
and VT-d. For the LAPIC registers, the jailhouse handles the
accesses differently depending on whether the host supports
the xAPIC or x2APIC mode. If the host only supports the xAPIC
mode, the jailhouse traps all the guest's accesses to the
LAPIC registers. Even if the guest would like to access the
LAPIC using the MSR interface in the x2APIC mode, the
jailhouse traps and emulates it on the top of xAPIC mode. If
the host supports the x2APIC, the jailhouse only traps the
access to the interrupt command register. ICR is used to send
IPIs to other process cores. The trap is required, so the
jailhouse can prevent the malicious guest from disturbing
other guests. In terms of the interrupt handling, the external
interrupts are delivered directly to the guest, which handles
them through the guest IDT. One exception is the non-maskable
interrupt. The jailhouse uses NMI to regain the controls of
guest CPUs.

Thus, the interrupts from the assigned devices and timer
interrupts are delivered directly into the guest without any
indirection, while the guest can have a fully control over its
assigned devices and LAPIC timer. These features help to meet
the real-time application requirement for the latency and
response time or the long-running computations. Furthermore,
the jailhouse confines the guest in its own cell and results
in security enhancement and no resource overcommitment.

% Previous work
\subsection{Exitless Interrupt}
Exitless Interrupt (ELI) is based on the following
conditions~\cite{amit:2015}. First, the guest has its own set
of dedicated cores. Second, the guest runs the I/O intensive
workload with the directly assigned SR-IOV devices. Third, the
number of interrupts that the guest receives from the assigned
devices is proportional to the guest execution time. Thus, the
ELI delivers the assigned interrupts to the guest directly and
non-assigned interrupts to the VMM. This is achieved by the
shadow interrupt descriptor table.

When the guest runs in the guest mode, it runs with this
shadow IDT prepared by the ELI instead of its own IDT. When
the logical processor in the non-root mode receives an
external interrupt, it screens the interrupt by the shadow
IDT. If the interrupt is from the assigned device, it
dereferences the corresponding table entry and invokes the
guest's ISR. Otherwise, it traps to the host, which handles
the non-assigned interrupts. To make such a distinction, the
ELI copies the guest's IDT, including the guest's exceptions
and assigned devices, to the shadow IDT. The ELI preserves
the device interrupt priorities and keeps the interrupt vector
numbers of each device the same between the corresponding
guest and host interrupt handlers. It marks guest entries as
present and the rest of entries as non-present. Moreover, the
ELI configures the logical processor to force exit on the
non-present exception. When the host handles the non-present
exception, it needs to inspect the exit reason. If it is due
to the non-assigned physical interrupt, it converts the
exception back to the original interrupt vector and invoke the
respective ISR. For the virtual interrupts from the emulated
device, the ELI marks them as the non-assigned interrupts.
After the trap, the host enters the special injection mode
that configures the logical processor to exit on any physical
interrupts and the guest to use its own IDT. The host injects
the virtual interrupt to the guest.

When the guest's ISR finishes handling its interrupt, it
updates the LAPIC EOI register and triggers the VM exit. The
VM exit can be disabled through the MSR bitmap, when
configuring the VMX module with the x2APIC programming
interface. Since the guest does not distinguish between the
injected virtual interrupt or the assigned interrupt, it
updates the EOI LAPIC register for all cases. This should not
be the case for the virtual interrupt from the host emulated
device. When the host operates in the special injection mode,
it traps the EOI write to the host. Once the guest finishes
all the pending EOI writes for the virtual interrupts, the
host leaves the special injection mode.

Thus, the ELI delivers the interrupts from the assigned
devices directly and virtual interrupts indirectly, while
preserving the interrupt priorities. It effectively reduces
the VM exits due to the assigned interrupts to 0 and handles
the EOI signals properly.

\subsection{Direct Interrupt Delivery}
Direct Interrupt Delivery (DID) solves the two
challenges~\cite{tu:2015}. First, the interrupts are directly
delivered to the guest without the hypervisor intervention on
the delivery path. Second, the guest completes the end of
interrupts without the VM exit. The two challenges are divided
into the following sub-tasks. First, if a guest is running,
the interrupts from the SR-IOV devices, timers and emulated
devices are delivered directly. Second, when the target guest
is not running, its interrupts are delivered through the
hypervisor. Third, the priority of physical and virtual
interrupts are preserved. Fourth, the number of interrupts
that the host needs to complete is zero. In addition, the DID
supports the unmodified guests.

The DID routes the interrupts to the guest or host
appropriately by configuring the interrupt routing and
remapping table from the IOAPIC and IOMMU respectively. For
the SR-IOV device interrupts, the DID disables the VM-exit due
to external interrupts. Consequently, the interrupts are
delivered normally through the guest interrupt descriptor
table. If the virtual processor is running, the device
interrupt is directly delivered. If the virtual processor is
rescheduled by the host, the interrupt is delivered to the
host through the non-maskable interrupt. The host injects the
corresponding virtual interrupt, when the virtual processor is
re-scheduled on the logical processor.

On the modern x86 architecture, each processor has its own
LAPIC. LAPIC generates timer interrupts, which are not routed
by the IOAPIC and IOMMU. Instead, the LAPIC delivers the timer
interrupts to its associated processor. After the guest
handles the timer interrupt, it sets up the next timer event
by configuring the LAPIC timer. This requires the host's help.
The DID ensures that the guest's timer interrupt only delivers
to guest instead of other user-level processes. The DID
installs the software timer on the host's dedicated core.
After the host receives the guest's timer interrupt, it
delivers the physical timer interrupt through the IPI. The
host delivers the timer IPI, only when the virtual processor
is active. If the virtual processor is preempted, the host
delivers the timer IPI, when the virtual processor is
scheduled on another logical processor.

The DID delivers the virtual interrupts as the IPIs, which are
treated as the external interrupts. Each QEMU virtual device
is represented by a thread and runs on its dedicate core.
Before delivering the virtual device interrupt, the device
thread need to check if the virtual processor is active. If
the virtual processor is running, the thread delivers the
virtual device interrupt through the IPI. Since the DID
disables the VM-exit due to external interrupts, the guest
receives the device interrupt directly. If the vCPU is not
running, the host receives the interrupt on the behalf of
guest. The host injects it to the appropriate guest as the
virtual interrupt, when the virtual-processor is scheduled on
the logical processor.

When the ISR completes, it instructs the logical processor to
update the LAPIC end-of-interrupt register. In the DID scheme,
the direct EOI write has the following considerations. First,
if the handler of virtual interrupts directly updates the EOI
register, the LAPIC may thinks there is no pending interrupt.
Second, it may also think the pending interrupt is completed,
which is still ongoing. Third, LAPIC may dispatch the lower
priority interrupt to preempt a higher-priority interrupt. The
root cause is that the virtual interrupts are not visible to
the hardware LAPIC, when they are injected via IRR and ISR.
The DID solves this problem by converting the virtual
interrupts as the IPIs, while disabling the VM-exit due to the
EOI writes. If there are multiple virtual interrupts, the host
issues them as the IPIs one at a time.

Thus, the DID delivers the external interrupts from the
assigned devices, timer and emulated devices directly into the
guest, while preserving the interrupt priorities. The DID
zeroes the number of VM exits due to the interrupts. Moreover,
it is not required to modify the guest kernel. This is one of
the advantages over ELI and ELVIS.
}

\section{Virtualization Support for HaaS}
% Virtualization support for HaaS include the following items
% CPU idleness detection and processing
% Direct NIC assignment
%   - VFIO
%   - CPU optimization for the baremetal network performance
% Direct interrupt delivery
%   - Posted-interrupt mechanism
%   - Direct timer-interrupt delivery
%     - periodic vs aperiodic timer interrupt
%     - one-shot vs periodic hardware timer interrupt
%     - guest-level access to the PIR page
%     - Eliminate the hardware lock, when the guest accesses
%       the PIR bit
%     - Spurious timer interrupts
% Seamless baremetal virtual machine migration
%   - NIC bonding with the hot plug/unplug operation migration
%   - DTID during the migration

\figw{architecture}{9}{Architecture of \na for single-VM virtualization (SVMV). 
The SVMV hypervisor is restricted to a single CPU core and provides
manageability services such as live migration and monitoring.
The HaaS VM runs on all the remaining CPU cores with direct access to I/O devices 
and local timers, including direct delivery of interrupts.}

%The design goal of \na is to eliminate the hypervisor from
%the guest I/O path while achieving bare-metal performance in
%the guest. To achieve this \na considers direct device
%assignment to the guest using Intel's VT-d support. First, the
%guest meets the bare-metal network and disk I/O performance
%with direct device assignment. We further apply optimizations
%to reduce the CPU utilization by hypervisor. Second, the timer
%interrupts are transformed into the posted interrupts, which
%are directly delivered to the guest by the logical processor
%without causing any VM exits. Finally, we present the
%migration of virtual machine with directly assigned devices.

To enhance the manageability of physical servers offered in a bare-metal cloud service, \na intalls 
a SVMV  hypervisor on each physical server  to create a single VM (called the HaaS VM), on which 
a \na user then installs a preferred OS and applications. 
Figure~\ref{fig:architecture} shows the high-level architecture of \na.
This  hypervisor is primarily meant to provide management functionalities, such as 
physical machine migration and application performance monitoring, rather than virtualization 
of server resources, and therefore is largely not involved in the I/O operations in which a HaaS VM participates.
%This section describes in detail the main design issues of this SVMV hypervisor, 
%which is based on Linux/KVM/QEMU, and our corresponding solutions.

I/O operations and interrupt processing using traditional VMs incur higher overheads than bare-metal execution. 
I/O operations issued by a VM typically trap into the hypervisor via VM exits for emulation.
Likewise, external device interrupts and local timer interrupts 
to the CPU running a VM result in VM exits for emulating virtual interrupt delivery.
Each VM exit is expensive, since it requires saving the VM's execution context upon exit,
emulation of the exit reason in hypervisor mode, 
and finally restoration of the VM's context before re-entry into guest mode.

The main technical challenge of providing the guest OS with the illusion of running on 
a bare-metal server is that its interactions with I/O devices, such as network 
interface card (NIC) and disk controller, must be direct without going through any intermediary.
\na allows the sole VM on the physical machine to directly interact with PCIe I/O devices and 
timer hardware by leveraging Intel VT-d~\cite{intelvtd-paper} and 
Linux Virtual Function I/O (VFIO)~\cite{vfio} mechanisms. 
To match bare-metal I/O performance, \na implements mechanisms 
for reducing the hypervisor's CPU, I/O, and memory 
footprint during runtime to minimize interference with guest operations.
In addition \na also enables direct delivery of both device interrupts
and timer interrupts to a HaaS VM by leveraging
the posted interrupt mechanism in Intel VT-d.
Direct interrupt delivery greatly reduces the interrupt processing latency and is
particularly useful for real-time applications running on bare-metal servers. 
When the VM must be live migrated, \na temorarily switches the VM to use
para-virtual I/O and virtualized timer, completes the migration
and, at the destination, switches the VM back to direct I/O device and direct 
timer access, all with minimal disruption of VM's workload.

In this section, we first  provide background on direct device access using 
Intel VT-d and Linux KVM/QEMU~\cite{kvm}, followed by the description of design challenges 
and our solutions for achieving bare-metal performance while supporting live migration.

\subsection{Background: Intel VT-d and VFIO}

%\subsection{Direct Interactions with PCI Devices}
Intel VT-d~\cite{intelvtd-paper} provides processor-level support for direct 
and safe access to hardware I/O devices by VMs running in non-root mode.
Virtual function I/O (VFIO)~\cite{vfio} is a Linux software framework that enables user-space
device drivers to interact with PCIe devices directly without involving the Linux kernel. 
In general, there are four types of interactions between an OS and the PCIe 
I/O devices under its management:
\begin{enumerate} 
\parskip 0mm
\itemsep 0mm
%\setlength\itemsep{-0.04in}
\item At the system start-up time, the OS probes and enumerates the PCIe devices existing in the system, and assigns a device number, configuration space address range, and interrupt to each discovered device.

\item The OS reads and writes a device's configuration register space and the memory regions specified in its base address registers (BAR), including setting up DMA operations.

\item A DMA engine moves data blocks between main memory and PCIe devices according to the DMA commands set up by the OS.

\item A device interrupts the OS for its attention to certain hardware events, such as new packet arrival or transmision completion.

\end{enumerate}   
Except the probe/enumeration operations, 
\na enables a guest OS to interact directly (without VM exits) with NICs and disk controllers
for the other three types of device interactions. 

\subsubsection{Direct DMA Operations Using VT-d and VFIO} 
Typically, a PCIe device driver communicates with its device using programmed I/O (PIO) or
memory-mapped I/O (MMIO) operations against the memory areas associated with to the device. 
Essentially, VFIO makes the configuration register space and the memory regions of
each PCIe device accessible to user-space processes.
%VFIO retrieves a PCIe device's device-specifc information such as BARs from its
%configuration register space, and maps them into distinct regions in a special file. 
%By reading and writing specific regions of this special file, a user-space program 
%is able to directly interact a particular PCIe device. 
%VFIO provides user-space programs a PCIe device read/write API, and converts these API calls
%into file read/write operations against the special PCIe device file.
%
%The userspace driver uses
%the device file descriptor and offset to access each region
%and retrieve the device information. The VFIO decomposes the
%physical device to a software interface. Such software
%interface is turned into the assigned device by QEMU.
%Essentially, the device read and write handler in the QEMU
%memory API is forwarded to the VFIO read and write handler.
%Not all accesses to a PCIe device's associated memory regions by user-space processes are direct, 
%because some parts of a PCIe configuration register space are privileged, such as 
%message signalled interrupts (MSI), BARs, and ROM,  and accessing them 
%requires KVM's or QEMU's emulation. 
%
%Nonetheless, accessing some parts of PCI configuration
%PCI configuration space is
%not handled as memory regions in QEMU. Some of the accesses to
%the PCI configuration space is passthroughed directly, while
%others, such as MSI, BARs, and ROM, need to be emulated.
%
Thus a user-space processe can use DMA operations to move data directly between a PCIe device and 
a region of its virtual address space.
The address of the source or destination buffer of a DMA operation resides in 
the {\em device address space} of the PCIe device involved in the DMA operation.
Intel's VT-d architecture provides an IOMMU~\cite{ben:2006}, that maps a 
PCIe operation's {\em device address} into a {\em physical address}, 
which is used to access main memory. 
\mycomment{
When a user-space program sets up a DMA operation, it only knows the virtual address of the associated buffer and thus places
this address in the DMA command, essentially treating this virtual address as a device address. 
Given such a DMA command, VFIO consults with the page table associated with this user-space program to extract the physical address corresponding
to the DMA buffer's virtual address, and creates an IOMMU entry that links the buffer's virtual (device) address with its associated physical address. 
When this DMA command is executed at run time, its buffer's device address is correctly translated to its corresponding physical address.
 }
  
The key field of each IOMMU entry includes a PCIe device number and a device address.
Therefore, a machine's IOMMU may map the same device address into different physical addresses when the device address is associated with different PCIe devices.
By controlling which user-space programs can access which PCIe devices, VFIO is able to leverage IOMMU to effectively prevent a user-space program 
from using DMA operations to corrupt the physical memory areas owned by other user-space programs.

To KVM, a VM runs as part of a user-space process  called QEMU~\cite{qemu},
specifically, as guest-mode threads within QEMU's virtual address space..
QEMU uses VFIO to configure a VM to directly access the device address space 
its assigned PCIe devices without emulation via KVM or QEMU.
In contrast, in para-virtual Vhost~\cite{vhost-net} I/O architecture, 
each incoming or outgoing I/O operation (network or block I/O) must go through 
a special hypervisor-level thread (called the vhost worker thread), 
which emulates a virtio~\cite{russell:2008} device in the kernel and 
therefore keeps QEMU out of the data plane.
However, QEMU still is responsible for such control plane processing as setting up, 
configuring, and negotiating features for an in-kernel virtio device.  



\mycomment{
Second, the VFIO programs the IOMMU to transfer the data
between the userspace driver and device in a hardware
protected manner. The VT-d IOMMU provides the device isolation
using the per-device IOVA and paging structure. The virtual
virtual address requested by the device is translated to the
physical address through the set of paging structures by
IOMMU. In the case of VM, the address space of assigned device
is embedded within the guest address space. The IOMMU is
programmed to translate such IOVA to the host physical address
which is mapped to the guest address space. Such translation
is both realized and protected by the IOMMU.

Third, the VFIO has a mechanism to describe and register the
device interrupt to signal its userspace driver. When the VM
accesses the device configuration space, it is trapped through
QEMU. QEMU configures the IRQs by the VFIO interrupt ioctls
and sets up the event notifiers between the kernel, QEMU and
guest. When the kernel signals the IRQ to QEMU, QEMU injects
it into the VM. The interrupt signaling is further speeded up
by moving QEMU out of the way. KVM supports both ioeventfd and
irqfd. ioeventfd registers PIO and MMIO regions to trigger an
event notification, when written by the VM. irqfd allows to
inject a specific interrupt to the VM by KVM. Once ioeventfd
and irqfd are coupled together, the interrupt pathway remains
in the host kernel without exiting to the userspace QEMU.
Using the VT-d, the KVM and QEMU is completed removed from the
signaling path way. It enables the direct interrupt delivery
from the assigned device to its VM without a VM exit.
}


\subsubsection{Posted Interrupts}

{\bf Conventional Interrupt Delivery:}
\mycomment{
Our design uses the hardware-assisted posted interrupt
mechanism and achieves the direct interrupt delivery of
assigned devices and local timers without the intervention of
hypervisor. It reduces the hypervisor CPU utilization and
dedicates the CPU time to the guest. Under the normal
circumstances, the guest can not handle physical interrupts
without the hypervisor. 
}
In the conventional architecture, when an interrupt is delivered to a CPU core,
it triggers a VM exit if the CPU core is not in the root mode and control is transferred to the hypervisor. 
%Saving and loading the execution context between the root and
%non-root mode waste the CPU cycles. Second, the host needs to
Then the hypervisor examines the cause of the interrupt. 
If the interrupt is meant for a VM scheduled on the CPU core,
the hypervisor delivers this interrupt as a virtual interrupt to the target VM
next time when it is scheduled on the CPU core; otherwise the hypervisor handles
the interrupt on its own.
After a VM completes servicing an interrupt, it writes to the EOI (End of Interrupt)  register 
to signal to the hardware that the interrupt in question has been handled. Because the EOI register is 
a privileged resource, every EOI register write causes a VM exit.
Therefore, processing of each physical interrupt costs at least two VM exits.
This per-interrupt overhead is too expensive for network-intensive applications that 
are designed to process multi-million packets per second. 

\mycomment{
and handle the physical interrupt. If the physical
interrupt is meant for the guest, the host needs to deliver it
as the virtual interrupt upon the next VM entry. Otherwise,
the host handles it and schedules the next VM entry. Third,
when the guest's interrupt handler finishes, it writes to the
EOI register. Such write operation may induce the VM exit.
Since the guest is not aware of the distinction between the
physical and virtual interrupt, it signals the completion of
interrupt in the same way. After the guest handles the virtual
interrupt, its EOI update is normally emulated by the host.
Fourth, the host may need to use CPU cache and reduce the time
to handle the physical interrupt. This introduces the CPU
cache pollution.
}

%In the X86 architecture, 
A VM may experience two types of interrupts: {\em external} and {\em local} interrupts.
External interrupts originate from external I/O devices, such as network card or disk controller.
When these I/O devices generate a hardware interrupt, this signal first goes to an IOAPIC, which, through an {\em Interrupt Redirection Table}, 
converts the hardware interrupt into an interrupt message that contains a vector number and is sent to a particular CPU core.
There is a local APIC associated with each CPU core to field interrupts sent to the CPU core.
In the Intel VT-d architecture, all interrupts sent to any local APIC are intercepted by  an {\em Interrupt Remapping Table} in the IOMMU unit, 
which provides a similar functionality to that of an Interrupt Redirection Table
to those external interrupts that do not come from an IOAPIC, e.g., message signaled interrupts from PCI devices.

{\bf Posted Interrupt Delivery:}
The Posted Interrupt mechanism~\cite{intelvtd-paper,intelvtd-manual} allows a CPU core that is running in the non-root mode to receive and handle interrupts with a specific vector
(Post Interrupt Notification or PIN vector) directly without involving the hypervisor.
When a CPU core handles a PIN interrupt, it examines a bitmap data structure associated with the CPU core called {\em Virtual Interrupt Request Register} (vIRR),
which is copied from another privileged data structure also associated with the CPU core called {\em Posted Interrupt Requests} (PIR) as a side effect of an PIN interrupt delivery,  
to determine the vectors of the interrupts behind the PIN interrupt, and processes these interrupts one by one according to their vectors and priorities.
The PIR bitmap is part of a per-VCPU data structure called {\em Posted Interrupt Descriptor} (PID), which in addition contains an Outstanding Notification (ON) flag, which, when set, indicates that there is a PIN interrupt pending.  
The address of the PID and the PIN vector associated with a CPU core are both contained in the virtual machine control structure (VMCS) associated with the VCPU running on the CPU core.

%\subsubsection{Direct External Interrupt Delivery}
To deliver an external interrupt directly to a VM, KVM sets up an Interrupt Remapping Table (IRT) entry for that external interrupt as follows.
First , it sets the IRT entry's IM bit to 1, which means that any interrupt matching this entry is to be delivered via the posted interrupt mechanism.  
Then, it ses the IRT entry's PID address field to the PID address associated with this entry's target CPU core. 
When an external interrupt arrives at the IOMMU and matches an IRT entry, the 
hardware first locates the PIR bitmap in the target CPU core's PID, then sets the bit in the PIR bitmap corresponding to the external interrupt's associated vector,
converts this interrupt into an interrupt labelled with the PIN vector, and finally delivers this PIN interrupt to the target CPU core's local APIC.
Because the external interrupts ``pretend'' to be a PIN interrupt, the target CPU core processes them directly without causing any VM exit.

When a VM completes the service of an interrupt, it needs to clear the EOI register to indicate to the 
associated local APIC that it is done with the interrupt. Because the EOI register is a privileged resource, accessing the EOI register
would normally requires a VM exit, which would add to the interrupt processing overhead. 
Hence, by default, when configuring a VM, KVM disables all EOI-triggered VM exits by setting the corresponding fields in the VM's VMCS.
%To allow a VM to avoid a VM exit due to clearing the EOI register after handling an interrupt of vector X,  KVM clears X's corresponding bit in 
%the {\em EOI exit bitmap} in the VM's VMCS. As a result, at run time after this VM completes servicing an interrupt of vector X and clears the EOI register via a memory mapped 
%intreface (called APIC-access page), no VM exit occurs.   
%Helped with the posted interrupt and EOI virtualization mechanism, a HaaS VM is now able to process any external interrupt directly without triggering any VM exit.

 
\mycomment{
VT-d supports the posted-interrupt capability and deliver the
external interrupts directly from the I/O devices and external
controllers without the cost of VM exits and the hypervisor
intervention. Before utilizing such feature, the system
software needs to define the posted interrupt notification
vector. The PIN signifies the incoming external interrupt from
the assigned device is subjected to the posted-interrupt
processing. The processing is achieved by updating the
posted-interrupt descriptor dynamically. When the VMCS is
actively used by the logical processor in the non-root mode,
it is prohibited to update its data structures. The PID is the
exception. Nonetheless, there is one requirement that the PID
modifications must be done using locked read-modify-write
instructions. Here is another benefit of posted-interrupt
support. When the virtual processor is scheduled on another
VCPU, the VMM can co-migrate its interrupts from the assigned
devices by setting the corresponding bits in posted-interrupt
register of PID.

The posted-interrupt support is accomplished in three general
steps. First, the VMM programs the interrupt-remapping
hardware with the mapping between the external interrupt and
virtual interrupt. Second, when the external interrupt is
delivered to the interrupt-remapping hardware, it sets the
outstanding bit and corresponding bit of virtual interrupt in
the posted-interrupt register of PID. It generates the PIN.
The IOAPIC delivers the PIN to the appropriate LAPIC. Third,
PIN notifies the logical processor that it is the
posted-interrupt event. The logical processor starts the
posted interrupt processing and delivers the virtual interrupt
without any VM exit.

The posted-interrupt processing is described in the following
steps. First, when the external interrupt is delivered to the
guest's processor, it is acknowledged by the LAPIC. LAPIC
provides the processor core the interrupt number. Second, if
the physical interrupt is equal to the PIN, the logical
processor starts the posted interrupt processing. Third, the
processor clears the outstanding notification bit from the
posted-interrupt descriptor. Fourth, the processor
acknowledges the EOI. Fifth, the processor updates the vIRR by
synchronizing it with the PIR. Sixth, the processor acquires
the next request virtual interrupt. It updates RVI by the
maximum of previous RVI and highest index of bits set in PIR,
before it clears PIR. Seventh, the processor evaluates the
pending virtual interrupt. Eighth, the processor delivers the
virtual interrupt.
}


\subsection{Achieving Bare-metal I/O Performance}
\figw{virtualization_overhead}{9}{Virtualization overheads in I/O operations due to VM exits in (a) para-virtualized I/O, (b) pass-through I/O, and (c) optimized pass-through I/O in \sna.}
%TODO: Describe I/O challenges here
%	Why CPU utilization increases
% 	VM Exit overhead
While using the VT-d and posted-interrupt mechanisms removes 
the hypervisor of from I/O data plane path of the VM, these alone are
not sufficient to reduce hypervisor overheads. 
To ensure that a VM using VT-d and VFIO can achieve bare-metal 
I/O performance and with least overheads, \na implements 
the following overhead reduction mechanisms

{\bf Dedicated CPUs:}
First, the hypervisor is assigned one dedicated physical CPU core
(CPU 0 in our implementation)
where it executes all of VM management operations. 
The HaaS VM is assigned all the remaining physical 
CPU cores by pinning the guest VCPUs to 
dedicated physical cores one-to-one~\cite{amit:2015}. 
This prevents the hypervisor's threads 
from competing with the VM's VCPU threads and 
also reduces the need to re-route interrupts
from the VM's assigned devices.
To avoid unnecessary VM exits on the HaaS VM's CPUs,
all external interrupts which are not directly handled
by the Haas VM, are delivered to the hypervisor's physical 
CPU by the IOAPIC.

{\bf Disabling  HLT-triggered VM exits:}
The second optimization relates to reducing hypervisor's CPU utilization
under the following scenario.
When handling frequent I/O operations, 
a guest VCPU  may become idle for brief intervals, and 
run a special idle thread.
On X86, this idle thread consists of a loop of {\tt HLT} instructions, 
each of which is {\em intended} to place the CPU in the C1 energy-saving mode 
until an external interrupt occurs.
However, since {\tt HLT} is a privileged instruction, it triggers a 
VM exit when executed by the guest.
To emulate the {\tt HLT} instruction, currently, the KVM hypervisor
polls (busy waits) for external events for a short time 
(adaptively between 0-200ns). If no events occur during this 
polling period, the guest VCPU is blocked and the physical CPU 
is placed in the low-power state.

The problem we observe is that, when the VM sends or receives network packets
at a high rate, the hypervisor-level CPU utilization increases
dramatically. This is because frequent {\tt HLT}-triggered VM exits
result in frequent invocation of the event polling loop in the hypervisor. 
Often, the physical CPU re-enters the guest mode directly 
from the polling loop as soon as the next network packet 
or interrupt arrives for the VM, without ever entering the low-power mode.

Although peak network throughput is not reduced when 
using VFIO (see Figure~\ref{fig:network_bandwidth}), 
both VM exits and polling loop cause significant 
CPU utilization by the hypervisor 
(see Figure~\ref{tab:cpu_utilization_40gbps}).
{\em Ironically, this overhead occurs even when using pass-through I/O and 
posted interrupts which were meant to reduce hypervisor overheads.}
Eliminating this event polling alone is insufficient because {\tt HLT}-triggered VM exits 
are also expensive by themselves; the guest VCPU thread is blocked and
must make a trip through the hypervisor's CPU scheduler before 
re-entry into guest mode.

Since \na is designed for a single VM setting, {\em we choose to
disable {\tt HLT}-triggred exits altogether}
by modifying the corresponding execution control fields in the VM's VMCS.
As a result, whenever a CPU becomes idle, no {\tt HLT}-triggered
VM exits occur, the CPU lowers its clock frequency while in guest mode, 
the total energy consumption is reduced, and importantly, 
hypervisor code is not invoked in I/O data path. 

Another interesting side effect of disabling {\tt HLT}-triggered VM exits
is that the HaaS VM is observed to receive more interrupts
from its direct-assigned device, possibly due to 
faster EOI response and hence less interrupt coalescing by 
the hardware, similar to the behavior on a bare-metal OS.

%
%First, the direct assigned network card and disk drive under
%the VFIO framework removes the host from the forward I/O path.
%%The guest has the control over the assigned devices and avoids
%the virtualization overhead, when accessing the device control
%registers and performing the
%DMA~\cite{sdm:2018,intelvtd-manual,williamson:2016}. After the
%assigned device services the request, it delivers the
%interrupt to the guest and triggers the VM exits due to the
%external interrupts and EOI respectively. Using the VT-d
%posted-interrupt and VT-x APIC virtualization~\cite{postedinterrupt},
%the guest handles the device interrupt and updates the EOI
%without any VM exit. Thus, the host is completed removed from
%the guest I/O path for the passthrough devices. Nonetheless,
%


\subsection{Direct Local Interrupt Delivery} 
\label{subsubsec:shared_pid_dtid}
\figw{dtid}{9}{Architecture for direct delivery of timer interrupts in \sna. 
%TODO: mention disabling WRMSR VM Exits.
The SVMV hypervisor configures each guest LAPIC to transform the local timer 
interrupt into a PIN interrupt and allows the guest
to directly update the PID page.}

Unlike an external interrupt, a local interrupt is delivered to a CPU core's local APIC without going through the IOMMU's Interrupt Remapping Table.
Therefore, such local interrupts cannot be delivered to a VM via the posted interrupt mechanism.
In this section, we describe how to achieve direct delivery of two types of local interrupts, {\em timer interrupt} and {\em inter-processor interrupt} (IPI).

{\bf Timer Interrupts:}To enable a timer interrupt to be delivered directly to a HaaS VM running on a CPU core,
\na first leverages the {\em local vector table} in the local APIC to turn each timer interrupt into a PIN interrupt, and 
then programmatically sets a particular bit in the CPU core's PIR bitmap to indicate to the target VM that underlying the PIN interrupt is a timer interrupt.

If the PIR bit associated with a timer interrupt is set way ahead of the interrupt's next expiration time, 
the interrupt's target VM may receive spurious timer interrupts. 
For example, suppose the next expiration time of a timer interrupt is $T_1$, and a NIC interrupt is delivered via the posted interrupt mechanism to the target VM at $T_2$, where ${T_2} < {T_1}$.
When the target  VM processes the PIN interrupt triggered by the NIC interrupt, it finds in the vIRR bitmap that two bits are turned on, one for the timer interrupt and the other for the NIC
interrupt, {\em even though the timer interrupt is spurious} because the real local timer has not yet expired.

To solve this spurious timer interrupt problem, a HaaS VM's guest OS is modified to keep track of the next expiration time of every timer interrupt, ignore an ostensible timer interrupt when  
the current time is substantially smaller than the next expiration time, 
and set the timer interrupt bit in the PIR because the PIR bitmap is cleared 
after a PIN interrupt is delivered.
The last step is required to ensure that a timer interrupt that is considered spurious and thus ignored because of an external interrupt,  still has a chance to be delivered directly when the timer truly expires.  

To avoid VM exit when a HaaS VM modifies the PIR of the CPU core it runs on, 
\na allocates a separate page to house the PIR and makes the page accessible to the VM.
With this set-up, a HaaS VM can write to the PIR of the CPU core it runs on without triggering any VM exit.

After a timer interrupt handler services a timer interrupt, it may need to configure the next timer expiry by updating the initial counter (TMICT) register of the local APIC timer, 
which would normally cause a VM exit. To avoid this VM exit, \na leverages VT-d's hardware-assisted APIC virtualization by properly configuring the MSR bitmap in the associated VMCS. 
As a result, the KVM's  intercept of any TMICT MSR update is disabled. When a VM writes to the TMICT, the change is written to the associated register directly without triggering any VM exit.     

Finally, from a security viewpoint, control over the timer interrupt is considered important for a hypervisor or OS to maintain control over all physical CPUs.
In \na, although the HaaS VM can directly control the timer hardware on its assigned CPUs, the SVMV hypervisor can still regain 
control over the CPUs when needed, such as before live migration. 
The hypervisor always controls the CPU0 and its local timer. 
To regain control over other physical CPUs assigned to the VM, the hypervisor 
simply disables direct timer access for the VM by reconfiguring the IOMMU and VMCS to 
disable PIN and trigger VM exits for timer interrupts; the hypervisor then delivers
emulated virtual interrupts to the guest.

Combining all the above techniques, \na is able to deliver local timer interrupts 
directly to a HaaS VM and have them properly serviced, 
all without causing any VM exits and while maintaining control over all physical CPUs.


{\bf Direct IPI Delivery:} 
Just as with direct delivery of timer interrupts, the posted interrupt mechanism can be used for direct delivery of inter-core IPIs.
Here we describe the design of direct IPI delivery mechanism which is currently under development in our prototype.
When a source CPU core sends an IPI to a destination CPU core, the source CPU core configures the Interrupt Command Register (ICR) in its local APIC, 
and, after the low double-word of the ICR is written to, triggers an interrupt message to be sent to the destination CPU core's local APIC through an inter-APIC system bus.
To enable an IPI to be delivered directly to the HaaS VM, \na configures the source CPU core's ICR so that the resulting interrupt message carries a PIN vector,
and then sets a certain bit in the PIR of the destination CPU core to indicate that the interrupt behind the PIN interrupt is an IPI.

 



\mycomment{

The longer the guest stays on its CPU, the more local timer
interrupts it receives. The goal is to let the guest have its
dedicated CPUs. Our design does not only directly deliver the
timer interrupts to the guest, but also take one step further
by allowing the guest update its next timer event directly.

The local timer interrupt is delivered to the guest and
results in two scenarios. First, the timer interrupt is meant
for the guest. It induces the VM exit and the control is
transferred back to the host. The host handles the timer
interrupt and injects the virtual timer interrupt to the
guest. When the guest receives the virtual timer interrupt, it
services the timer interrupt and set up the next timer event
by updating the LAPIC timer initial count register through the
x2APIC interface. This triggers the MSR-write VM exit and the
control is transfer to the host. The host helps the guest to
set up its next timer by registering the \texttt{hrtimer}
object of guest next timer event. Second, the timer interrupt
is not meant for the guest. It induces the VM exit and
transfer the control back to the host. The host processes the
timer interrupt but does not inject the virtual timer
interrupt. Nonetheless, if the timer interrupt is not meant
for the guest, the guest should not pay the price.

The first task is to transform the local timer interrupt into
the posted interrupt, which is directly delivered to the guest
by the VT-d hardware. It requires two actions. The
timer-interrupt bit of posted-interrupt request needs to be
set, before the posted-interrupt notification is delivered to
the guest core. Since the guest is responsible for its own
timer interrupt, the guest should set the bit in the PIR.
However, such a PIR structure is embedded in the
posted-interrupt descriptor and protected by the host. The
host needs to share the PIR with the guest by isolating the
entire PID to a shared page. If the guest messes up setting
the proper bits in the PID through the EPT, it does not affect
the host normal operations. In our design, the shared PID page
is accessible by three entities: host, guest and
virtualization hardware. The second task is to allow the guest
control the timer initial count register of LAPIC timer. With
the hardware-assisted APIC virtualization, this is achieved by
updating the MSR bitmap of VM control structure. The KVM
intercept of TMICT MSR update is disabled. When the guest
configures the TMICT, the change is written to the register
directly without triggering the VM exit. The third task is to
configure the LAPIC timer chip to deliver the posted-interrupt
notification instead of the actual timer interrupt. In
summary, we reach our goal of guest having dedicated CPUs by
disabling the HLT- and timer-related VM exits.

Using the shared PID has the draw back. It induces the
spurious timer interrupts causing additional interrupt
processing in the guest. Since the guest sets the PIR
timer-interrupt bit before its next timer event, the it
induces the spurious timer interrupts. Such a fake timer
interrupt is induced in two cases. First, the guest
experiences the spurious timer interrupt when performing
I/O-bound activities with the assigned device. Let's take the
assigned network card for an example. Both the bits of timer
interrupt and network-device interrupt are set in the PIR.
Based on the Intel architecture, the timer interrupt has a
higher priority than the network interrupt does. The timer
interrupt is delivered before the network interrupt. Although
the guest should have only processed the network interrupt, it
first processes the timer and then network interrupt. Second,
upon the VM entry, the PIR is synced to the virtual
interrupt-request register because of the KVM implementation.
One of time points to evaluate the virtual interrupt delivery
is at the VM entry time. If the PIR timer-interrupt bit is
present during the copy, the fake virtual timer interrupt is
delivered into the guest after the VM entry. If the arrival of
virtual timer interrupt is earlier than the expected
expiration, the guest ignores it and processes the next
interrupt. Thus, the CPU overhead is reduced in comparison
with the full timer interrupt processing. 

}
% Currently, 3.4 is under the review.
\subsection{Seamless Device State Migration}

%The VM on an \na server is special because it is the only VM on the server and it 
%interacts directly with PCIe devices and local timers.
%We call such a VM a HaaS VM.
Because the states associated with directly accessible devices complicate the migration process~\cite{zhai:2008}, 
\na augments KVM's VM migration capability with additional mechanisms to support seamless migration of HaaS VMs. 
More specifically, \na adopts the following unified strategy to hide the states of directly accessed device states from KVM's VM migration logic:
In the normal mode, a HaaS VM directly accesses PCIe devices and timers; immediately before and during when the HaaS VM is migrated,  the VM 
accesses PCIe devices and timers indirectly; after the migration, the migrated HaaS VM accesses PCIe devices and timers directly again.

Moreover, \na assumes that the source and destination physical servers involved in a migration have an identical
set of hardware devices with which a HaaS VM interacts directly.
This way, the driver code directly interacting with these devices has a chance to properly work on 
both the source and destination server.
In this section, we focus on two types of directly accessed devices: NIC and timer.

\mycomment{
The direct device assignment makes it difficult to migrate the
guest to its destination~\cite{zhai:2008}. After the VM
migration, it is possible that the previously-assigned devices
may not be available at the destination. Even if the assigned
device is available, the internal state of device may not be
readable or still on its way to the destination. The host at
the destination has a hard time to passthrough the device
without the device-specific knowledge. Moreover, some devices
have the unique hardware information that cannot be
transferred, such as the MAC address of network interface
card. In the case of guest-controlled timer, it depends on the
VT-x availability at the destination. Our design takes the
approach of alternating the usage of passthrough and
respective virtual device with the acceptable service downtime
or number of missed time interrupts. We assume that the
devices and hardware supports are available. For the network
activity, the network traffic is switch from the assigned to
virtual network device, before the migration starts. The
network traffic is switched back after the guest starts up at
the destination. For the local timer interrupt, the direct
timer interrupt delivery is switched back to the indirect
delivery with the help of \texttt{hrtimer} object and TMICT
WRMSR VM exit is enabled, before the migration. The changes
are reverted after the guest starts up at the destination.
}

\subsubsection{NIC State Migration}
\figh{nic_bonding}{5}{NIC bonding}

We assume each HaaS physical server is equipped with an SR-IOV Ethernet NIC~\cite{dong:2008},
which provides one physical function and multiple virtual functions, each with its own MAC address.
\na sets up a directly accessed or pass-through NIC using one of the virtual functions
and an indirectly accessed or virtual NIC using another virtual function,
and then teams them up using the active-backup mode of the Linux bonding driver~\cite{bond-dri}, as shown in Figure~\ref{fig:nic_bonding}.
The HaaS VM running on every \na physical server sends and receives network packets through such a bonded interface. 


During the normal run time, the pass-through NIC is the Active slave and the virtual NIC is the Backup slave,
so that a HaaS VM could make full use of the underlying physical NIC's capability.
Before a HaaS VM is to be migrated, \na hot-unplugs the pass-through NIC so as to fail the current Active slave.
Upon detecting the Active slave's failure, the bonding driver immediately switches the Backup slave to be the Active slave,
and from this point on, all network traffic goes through the virtual NIC and KVM has now captured the NIC-related state. Then \na kicks off a VM migration transaction for the HaaS VM to transfer its NIC-related state.
%During the migration, \na . 
After the migration transaction is completed, the HaaS VM is resumed on the destination server and still continues to use the virtual NIC.
In the mean time, \na hot-plugs the pass-through NIC on the destination  server to make it the Backup slave, and then hot-unplugs the virtual NIC so that the current Backup slave or the pass-through  NIC becomes the new Active slave.


\mycomment{
In this section we describe the mechanism of migrating a guest
with direct NIC assignment. As shown in Figure
\ref{fig:nic_bonding} the guest is configured with virtio
network device backed by the vhost driver and a passthrough
network interface card~\cite{zhai:2008}. Using SR-IOV
~\cite{dong:2008} the physical NIC is presented as virtual NIC
through virtual functions. For the purpose of simplicity, we
assume that the guest has one assigned network device. The
prototype overcomes the challenge of migrating NIC assigned
guest by the following strategy. It uses the Ethernet bonding
driver to direct the network traffic between the assigned and
virtual NIC. The migration procedure is divided into two
parts. During regular operation of guest, the assigned NIC is
used for higher network bandwidth. Before the migration, the
host uses the bonding driver and shifts the network traffic
from the assigned NIC to the virtual NIC. The source host
takes the control of the assigned NIC through hot unplug event
of the assigned NIC and starts the migration. After the guest
resumes at the destination, the destination host transfers the
control back to the guest using hot plug event of the assigned
NIC and switches the network traffic from virtual NIC to
assigned NIC.

Linux provides bonding driver to present multiple network
interfaces into a single logical interface. The modes of
bonding driver define the behavior of the bonded interfaces.
To maintain higher network performance during regular
operation of guest, we configure the bonding driver in
active-backup mode where the assigned NIC is chosen as the
active interface and the virtual NIC as a backup-slave
interface. In active-backup mode only one of the interfaces is
active at any time. When the active interface fails, one of
the slave interfaces becomes active. The bonding driver always
takes the MAC of the active interface. On failure of the
active interface, the bonding driver takes the MAC address of
the next to be slave interface. The change in MAC address is
notified by broadcasting ARP packets to avoid the network
packets loss in guest. Before the migration is initiated, on
hot unplug command, the outgoing network traffic is redirected
to the virtual NIC interface by the bonding driver and the
incoming traffic is shifted to the virtual NIC by broadcasting
ARP packets. Once the assigned NIC is hot unplugged, QEMU
issues migrate command. After the migration is completed, once
the guest resumes on the destination, the NIC device is hot
plugged.
}

The most important performance metric for VM migration is the service disruption time.
The additional service disruption time that \na introduces is attributed to the transition from
the pass-through NIC to the virtual NIC on the source server, and the transition from
the virtual NIC to the pass-through NIC on the destination server.
Measurements on a earilier \na prototype suggest that the transition from the virtual to the pass-through NIC introduces non-trivial (about 0.3 second) service disruption, which results from hot-plugging the pass-through NIC on the destination server. 

A deeper analysis shows that the hot-plug operation consists of three steps:
(1) QEMU prepares a software object to represent the pass-through
NIC, (2) then It populates this software object by with parameter values from extracted 
fiom the PCIe configuarion space of the NIC, (3) and finaly it
resets the software NIC object to set up the BAR and interrupt forwarding
information. The first and third step must take place in QEMU's main event
loop when the HaaS VM requsting the hot-plug operation must be paused.

To minimize this service disruption time due to hot-plug, 
\na performs the first and second step of the operation of hot-plugging the pass-through NIC
on the destination server while the HaaS VM is being migrated, and performs the third step after the HaaS VM is migrated and resumed. This brings the service disruption time due to hot-plug to ? second.


\mycomment{
However, by forcing the migrated HaaS VM resumed on the destination server to continue using the virtual NIC,
the time required to hot-plug the directly accessed NIC on the destination server could be fully masked. 
After \na successfully brings up the directly accessed NIC and makes it the Backup slave,  it fails the virtual NIC so as 
to direct all network traffic to the directly accessed NIC.  


We observed that on hot plug event of NIC device on the
destination host, the network service in guest drops until for
0.3seconds. Further, we investigate the reason for the network
packet loss in the guest. The hot plug mechanism of assigned
NIC consists of the following three steps. First, QEMU
prepares a software object that represents the passthrough
NIC. It then realizes the QEMU software object by getting a
copy of configuarion space from the NIC device. Finally, it
resets the software NIC object and setup the BAR and interrupt
forwarding. The first and last step happen in QEMU main event
loop during which the guest remains paused. As a result, the
guest experiences downtime during hot plug operation. In \na,
to mitigate the downtime due to hot plug operation on the
destination host, the first two steps are executed during
migration. During the first phase of pre-copy live
migration~\cite{clark:2005,postcopy-osr}, all the memory pages are
transferred to the destination over the network. The dirty
memory pages are then transferred in multiple iterative
rounds. The VCPU and I/O state of guest are transferred in the
final phase to resume the guest on destination. Step one and
two are executed on the destination host during migration.
Consequently, the network service in guest does not get
affected as it runs on the source host. QEMU allows to setup
and realize the software object during migration. However, the
BAR and interrupts can be setup only after resuming the guest.
Hence, we eliminate the downtime caused during the setup of
software NIC object phase.
}



\subsubsection{Timer State Migration}

To leverage the posted interrupt mechanism to directly deliver local APIC timer interrupts,
\na allows a HaaS VM running on a CPU core to directly access the following device state:
(1) setting the timer interrupt bit in the PIR associated with the CPU core's local APIC ,  (2) 
using a WRMSR to modify the initial counter register of the local APIC timer, and (3) 
computing the next timer expiration time from conversions between clock cycles and nano-seconds, the multiplication
and shift factor of the calibrated timer, etc., which are provided by the hypervisor.

During the normal run time, timer interrupts are delivered to a HaaS VM directly. 
Before a HaaS VM is to be migrated, \na notifies the HaaS VM to stop the 
durect timer interrupt delivery (DTID) mechanism, unmaps the PIR page, 
enables the TMICT WRMSR VM exit, configures the LVT in the local APIC to 
fire timer interrupts as they are rather than as posted-interrupt notification interrupts. 
Upon receving this notification, the HaaS VM uses a different set of 
multiplication and shift parameters to compute the next timer value, 
and convey the resulting value to the hypervisor via the \texttt{hrtimer} object
when control is transferred to the hypervisor upon a TMICT WRMSR VM exit. 


After a HaaS VM is successfully migrated and resumed, both the hypervisor and the VM restart 
the DTID mechanism, by running 
the aforementioned steps in reverse order.


\mycomment{   
The direct timer interrupt delivery depends the following
factors. First, the host shares the posted-interrupt
descriptor with the guest. When the guest udpates the
timer-interrupt bit of posted-interrupt request through EPT,
it does not trigger the EPT violations and the local timer
interrupt is delivered as the posted interrupt. Second, the
guest directly configures the timer initial count register
without a WRMSR VM exit. This is achieved by updating the MSR
bitmap of VM control structure. Third, the guest needs to
correctly compute its next timer events with the
host-calibrated LAPIC timer. To compute the next timer event
from the nano-seconds to the clock cycles, the multiplication
and shift factor of the calibrated timer are required. The
host needs to convey such information to the guest. The guest
configures the TMICT with the correct timer event in clock
cycles.

Before the migration starts, both host and guest need to tear
down the DTID. In the host, it
} 

\mycomment{
It does not only support the direct PCI device
assignment to the userspace processes of VM, but also the
platform devices. Why do we need to allow the userspace
programs to gain the control of physical devices? For the
field of high performance computing, the I/O performances has
a great impact on the overall system performance. The
performance congestion comes, When the rate of data being read
is slower than the rate of data being consumed. Or it happens,
when the rate of data being written is slower than the rate of
data being computed and produced.

The VFIO needs to fulfill the three requirements of device
assignment to a userspace process. First, the userspace driver
can access to the device resources such as I/O ports. Second,
the userspace driver can perform the DMA securely. This is
provided by the IOMMU protection mechanism. Third, the device
interrupts is delivered to the device owner in the userspace.
The way how the VFIO fulfills the three requirements and
applies them with QEMU is briefly described below.
}

\mycomment{
The hardware-assisted direct device assignment helps the guest
achieve the baremetal I/O performance without the additional
virtualization overhead using the VT-x and VT-d support.
After proper VMCS and EPT configuration, the guest gains
the control of assigned device
by MMIO/PIO without the help of hypervisor. The VT-x APIC
virtualization permits the guest to write to the
end-of-interrupt register without a VM exit. With the VT-d
support, the guest performs DMA with the enhanced security and
eliminates the VM exit overhead due to the device interrupts by using 
the posted-interrupt mechanism. In addition to the hardware
support, VFIO provides the software framework for the
userspace device drivers. It works with VT-d and QEMU and sets
up the direct PCI device assignment. Although it is expected
to move the hypervisor out of the guest I/O path, the
hypervisor still induces high CPU utilization due to the
HLT emulation. This greatly deviates our goal of guest having
its own dedicated cores. Nonetheless, our CPU optimization
strategies remedy such a problem.
}


\subsection{Implementation-specific Details}
% Low-level implementation details include the hypervisor
% modifications.
% - Disable the HLT existing: KVM
% - DTID: KVM and timer-interrupt handler
% - Seamless VM migration: QEMU

{\bf Code Changes:}
The SVMV hypervisor takes up 80-120 MB of memory of the physical machine it runs on when it is idle.
DTID requires modifications to the guest OS, and entails 387 lines of code changes: 43 lines are added to the timer subsystem, 110 lines adjusts the clock multiplication and shift factor, and 234 lines map/unmap/test the shared PID page and cleanup.

\mycomment{
%MOSTLY REDUNDANT
In addition to the existing supports from the VT-x and VT-d,
Linux and its modules, QEMU and KVM, we modify the kernel, KVM
and QEMU and reach our goal of baremetal virtual machine. The
following features are provided. First, the guest has its own
dedicated CPU resources and PCIe devices. The guest's cores
are isolated, so no other host processes or threads compete
the CPU resources with the guest. Each virtual processor is
pinned to the isolated core in a one-to-one fashion. It is
important to move the host out of the way, when the guest is
idle. The VM exit due to the HTL instruction is disabled.
Second, the interrupts from the assigned devices and LAPIC
timer are handled directly by the guest without the hypervisor
intervention. Our approach utilizes the posted-interrupt
mechanism to deliver the local timer interrupt. It has the
requirement that the guest needs to set the PIR
timer-interrupt bit before the arrival of posted-interrupt
notification. While the host protects the posted-interrupt
descriptor, it shares the only the PID from the associated
VMCS. When the DTID is enabled in the host and guest, it
introduces spurious timer interrupts. The guest ignores the
fake timer interrupt, when the timer interrupt arrives earlier
than the expected. Third, the guest updates the LAPIC TMICT
directly. This is achieved by updating the MSR bitmap of
running guest. As a result, when the guest writes its next
period to the LAPIC TMICT, it does not trigger a VM exit and
avoids the overhead of interrupt processing and complexity of
\texttt{hrtimer} subsystem in the host. Fourth, when switching
the traffic from the passthrough to the virtual network
device, the network service down time is reduced by the
Ethernet bonding driver. Furthermore, the network service down
time is eliminated when hot plugging the assigned network
device to the running guest.



\subsection{Disable HLT Exiting and Update the MSR Bitmap}
To disable HLT-related VM exit, the HLT-exiting bit of
processor-based VM-execution control of VMCS is cleared. 
Our
implementation uses the existing KVM function to set or unset
the HLT-exiting bit. The function is \texttt{vmcs\_write64}.
To update the MSR bitmap, we disable the VM exit due to TMICT
WRMSR. The function is
\texttt{vmx\_disable\_intercept\_msr\_x2apic}. We uses this
function as the prototype and enable the TMICT WRMSR VM exits,
while consulting the MSR bitmap from the Intel Software
Developer Manual.

\subsection{CPU Idleness Processing}

When a CPU on a Linux-based machine does not have any applications to run,
Linux schedules a special idle task to run on the CPU. On X86 CPU, this idle task consists of a loop of {\tt HLT} instructions, 
each of which places the CPU in the C1 energy-saving mode until an external interrupt occurs.
Because {\tt HLT} is a privileged instruction, its execution inside a guest VM triggers a VM exit and causes the control to be transferred to KVM. 
Currently, KVM emulates a {\tt HLT} instruction using a busy waiting loop, which polls the CPU until the associated external interrupt comes along. 
%One of our design goals is to let the guest stay on its CPU as
%long as it can. We encounter two different types of
%virtualization overheads. First, the idle guest issues the
%privileged HLT instruction. Such an instruction induces the VM
%exit and transfers the control to the KVM which starts to poll
%on the CPU until the event arrival. 

This polling-loop emulation of {\tt HLT} instruction incurs higher CPU utilization than that when {\tt HLT} instruction is executed natively. 
To address this issue, \na prevents VM exit when executing a {\tt HLT} instruction inside a HaaS VM by modifying the VM-execution control fields 
in the VM's virtual machine control structure (VMCS).
As a result, whenever a CPU becomes idle, no {\tt HLT}-related VM exits occur, the CPU lowers its clock frequency, and the total energy consumption is reduced. 
%It is eliminated
%by updating the primary processor-based VM execution control
%and disabling the VM exit due to the HLT instruction. It
%allows the idle guest to stay on its CPU without polling
%and results in the CPU clock frequency remains at minimal.
%Thus, disabling HLT-induced VM exit helps to reduce the CPU
%power consumption of idle guest. 
%Second, the local timer
%interrupt fires and causes the VM exit, when the guest's time
%quantum is expired. The longer the guest stays on its CPU, the
%higher number of physical timer interrupt it receives. 

%To support our objective, our work utilizes the posted-interrupt
%mechanism and directly deliver the interrupt into the guest
%without triggering any VM exit. This feature is discussed in
%Interrupts that wake a CPU up from the {\tt HLT}-induced energy-saving mode used to go through the hypervisor and thus may incur additional VM exits.
%As disucussed in the~\nameref{subsubsec:shared_pid_dtid}, we leverage X86's {\em posted interrupt} mechanism to deliver these interrupts directly to a VM in the energy-saving mode.
%Consequently, processing of CPU idleness incurs no VM exit, as is the case when the user OS runs directly on a bare metal server.
}

{\bf Gues OS access to the PIR bitmap:}
The Posted-Interrupt Descriptor (PID) contains the PIR bitmap 
which the guest OS must access for direct timer interrupt delivery in \sna.
A CPU core's PID is accessible to the IOMMU (Interrupt Remapping Table), 
the hypervisor, and (in \sna) to  the HaaS VM running on the CPU core. 
%Originally, the PID  is embedded within the HaaS VM's VMCS structure.
To make a PID accessible to the HaaS VM's guest OS,
when a VCPU is created, \na allocates a separate page for its PID, 
and places the PID's base address in the VCPU's VMCS, and 
in all the Interrupt Remapping Table entries that target at the vCPU.
%To enable the VCPU's associated VM to access this PID page, 
The guest OS provides the address of a page in the guest physical address 
space to the hypervisor via a hypercall, and the hypervisor maps this address 
to the host physical address of the PID page in QEMU's page table and 
the Extended Page Table (EPT)~\cite{ept-wiki}, and updates the 
PID page's reference count accordingly.
   
\mycomment{
and extended page table entry to the physical location of
shared page and the reference count of shared page. To have
the DTID reversible, the implementation saves the host
physical address of target GPA. When the DTID is torn down,
our implementation reverts the PTE and EPTE back to the saved
HPA and updates the reference counts of shared page. Second,
the TMICT WRMSR VM exit is disabled, so the guest is able to
update its TMICT without the additional cost. Third, the host
needs to inform the gust the multiplication and shift factor
of calibrated LAPIC timer. When the guest programs the LAPIC
TMICT, it needs the right factor to convert the duration in
time to the number of clock cycles. Fourth, we implement the
screening algorithm in the guest timer interrupt handlers,
\texttt{smp\_apic\_timer\_interrupt} and
\texttt{local\_apic\_timer\_interrupt}. If the timer interrupt
arrives than the expected expiration, it is the spurious
interrupt. Then, the guest ignores it by skipping the regular
processing of timer interrupt. Fifth, the guest updates the
PIR timer-interrupt bit, whenever it receives the timer
interrupt. 
}

%According to Intel IA-32 Intel Architecture Software Developer's Manual~\cite{sdm:2018} 
Intel IA-32 Intel Architecture~\cite{sdm:2018}  requires that all accesses to the PIR bitmap in a PID
must use the locked read-modify-write instruction to ensure mutual exclusion. 
In \sna, when network packets come in at a high rate, a HaaS VM may repeatedly turn 
on the timer interrupt bit in the PIR bitmap to ignore  spurious timer interrupts that 
accompany the NIC interrupts. However, these accesses to the PIR, being based on the
atomic test-and-set instruction, may prevent the Interrupt Remapping Table from 
accessing the PIR on behalf of incoming packets, and cause some of these 
packets to be dropped, eventually degrading the network performance.
 
\na solves this performance problem by requiring a HaaS VM to set the timer interrupt bit
in the PIR using a non-atomic instruction, which immediately cuts down the extent of lock contention
and boosts the network performance. This lockless design is safe because the PIR bits 
that the HaaS VM and the Interrupt Remapping Table access are guaranteed to be 
disjoint and thus do not need to be locked before being accessed.


{\bf NIC Bonding}
Linux's bonding driver provides a \texttt{fail\_over\_mac} option to change 
the MAC address of the bonded interface and broadcast ARP packets for the new MAC address 
when the Active slave fails. \na configures the bonding driver by setting   
\texttt{fail\_over\_mac} to 1, and cuts down the
network downtime due to hot unplug operations.

To transition a NIC between the VFIO and Vhost mode, \na hot-plugs and hot-unplugs the corresponding devcices at the right times.  It uses the QEMU command \texttt{device\_add} to implement a hot plug operation, and the QEMU command \texttt{device\_del} to implement a hot unplug operation. 



\mycomment{
\texttt{device\_add} transfers the control of NIC device from host to guest and.
on \texttt{device\_del} command the NIC device control is transferred back to host. 
In \na, the \texttt{device\_del} command itself is further broken down 
into three commands \texttt{setup\_nic}, which sets up a software NIC object, 
\texttt{realize\_vfio\_nic}, which configures a software NIC object, and \texttt{reset\_nic\_device}, which resets a software NIC object by setting the 
BARs and redirecting the associated interrrupts. 
The \texttt{setup\_nic} and \texttt{reset\_nic\_device} commands must be executed when the VM issuing them is paused. The \texttt{reset\_nic\_device} command is issued on VM resumption. 
}





\section{Performance Evaluation}
% Performance evaluation includes the following items.
% - Evaluation testbed and methodology
% - Network bandwidth and latency for the direct NIC assignment
%   - Our CPU optimization
%   - vhost
%   - vfio
%   - NIC bonding of vhost and VFIO
% - DID Efficiency
%   - Delivery effectiveness of NIC interrupts
%   - Delivery effectiveness of local timer interrupts
%   - Time the handling of spurious timer interrupt
% - Service disruption time for the baremetal guest migration
%   - Network downtime
%   - The required time to set up or tear down the VFIO and DTID

\subsection{Evaluation testbed and methodology}

\subsection{Direct assignment of 1Gbps and 10Gbps NIC}

\subsubsection{Network Bandwidth, Latency and CPU Utilization}

\subsection{NIC bonding}

\subsubsection{Hot-plug and Hot-unplug}

\subsubsection{Migration of guest}

%% Suggested subsections
%\subsection{Evaluation Testbed and Methodology}
%% Describe the tested including the following items.
% - Hardware configuration: CPU, memory and network cards.
% - Guest: CPU, memory, bonding driver, assigned and virtual devices.
% - QEMU
% - KVM
%\figw{cpu_state_diagram}{8}{CPU State Diagram}

The experiments are run on machines equipped with the 10-core
Intel Xeon CPU E4 v4 of 2.2GHz, 32GB memory, 40Gbps Mellanox
ConnectX-3 Pro network interface and Intel Corporation
Ethernet Connection I217-LM. The Linux kernel 4.10.1 and QEMU
2.9.0 are installed in the host. The guest is configured with
1 to 9 vCPUs, 10GB of RAM, 1 Virtio and 1 pass-through network
device. The Linux kernel of 4.10.1 and the Ethernet bonding
driver are installed in the guest. The bonding driver operates
in active-backup mode.

The tools to measure the CPU, memory and network I/O
performance are listed as follows. iPerf 2.0.5~\cite{iperf}
measures the network bandwidth. Ping~\cite{ping} measures the
round-trip delay. Atopsar 2.3.0~\cite{atopsar} measures the
CPU utilization. Free 3.3.10~\cite{free} measures the memory
consumption. Perf 4.10.1~\cite{perf} measures the number of VM
exits. Cyclictest 0.93~\cite{cyclictest} benchmarks the timer
interrupt latency.
%Kernbench 0.42~\cite{kernbench} benchmarks the CPU throughput.
The following configurations are evaluated:
%depending on the physical or virtual network device, CPU
%optimization and DTID.
\mycomment{
\begin{enumerate}[(a)]
 \item The guest uses the Virtio network device backed by the
  vHost driver (Guest + vHost).
  \item The guest uses the assigned network device (Guest +
  VFIO).
  \item The guest uses the assigned network device. We also
  apply the CPU optimization (OPTI Guest). There are no VM
  exits due to the network interrupt and HLT instruction.
  \item The guest uses the assigned network device. We apply
  both the CPU optimization and DTID (DTID Guest). guest.
  There are no VM exits due the network interrupts, HLT
  instruction, local timer interrupts, direct timer updates or
  EPT violations when accessing the shared PID page.
\end{enumerate}
}

\begin{itemize}
\parskip 0mm
\itemsep 0mm
\item {\bf Bare-metal}: A machine without virtualization.

\item {\bf VHOST}: A HaaS VM accessing I/O devices using the
                   vHost interface.

\item {\bf VFIO}: A HaaS VM accessing I/O devices using the
                  VFIO interface without incurring VM exits
                  due to network interrupts.

\item {\bf OPTI}: A HaaS VM accessing I/O devices using the
                  VFIO interface without incurring VM exits
                  due to network interrupts or HLT
                  instructions.

\item{\bf  DTID}: A HaaS VM accessing I/O devices using the
                  VFIO interface without incurring VM exits
                  due to network interrupts, HLT instructions
                  or local timer interrupts.

\item{\bf  DID}: A HaaS VM accessing I/O devices using the
                 VFIO interface without incurring VM exits due
                 to network interrupts, HLT instructions or
                 local timer interrupts or IPIs.
\end{itemize}


\mycomment{
In Figure~\ref{fig:cpu_state_diagram}, it shows the transition
among host and different guest configurations. The control is
transferred to the host upon a VM exit. After the host has
done it emulation, the control is return back to the guest. In
the case of live migration, OPTI or DTID guest are reverted
back to the unmodified guest before the migration starts.
After the migration ends, the unmodified guest is again
transformed to the OPTI or DTID guest.
}


In our experiment, it is necessary to use two CPU cores to
saturate a 40Gbps Infiniband link for all configurations. One
core is handling the interrupts and soft IRQs, while the other
is running the network performance testing workload. We use a third
core to monitor the CPU utilization, which does not
affect the network performance. In contrast, only one
core is needed to saturate a 1 Gbps Ethernet link.

%
%\subsection{CPU and Performance of Assigned Network Device}
%% Excel: cpu_network_io_performance.xlsx
%% network bandwidth performance: 1 gbps and 40 gbps.
%% CPU utilization when guest performs the network I/O.
%% number of VM exits during the network I/O.
%
%\subsection{Direct Interrupt Delivery Efficiency}
%% Excel: did_efficiency.xlsx
%% number of interrupts from the network card.
%% number of timer interrupts.
%% number of spurious interrupts.
%% time to handle the spurious interrupts.
%% cyclic test and cumulative probability distribution.
%
%\subsection{Seamless Live Migration}
%% Excel: migration_performancy.xlsx
%% migration downtime
%% network downtime
%% time to set up and tear down the VFIO NIC.
%% time to set up and tear down the DTID.

% Acknowledge our team members, fans and sponsors.

% USENIX program committees give extra points to submissions
% that are backed by artifacts that are publicly available. If
% you made your code or data available, it's worth mentioning
% this fact in a dedicated section.


%
\begin{abstract}

--------Prof. Chiueh'ss abstract
Bare metal cloud service is an emerging form of cloud service in which users 
rent physical servers from a cloud operator because they want to make full 
use of the server’s underlying hardware without paying any virtualization overhead. 
Building a bare metal cloud service management system is technically challenging 
because of the constraint that no software agent could be installed on the 
physical servers to be rent out. Such a constraint is particularly limiting 
when it comes to the support for migration of physical machine state and 
performance monitoring for applications running on physical machines. 
The ITRI HaaS OS or IHO, is a bare metal cloud service management system 
that removes this constraint by installing on each physical server a 
single-VM virtualization hypervisor, which affords the user complete and 
direct access to all the devices on the server, and at the same time 
significantly enhances each server’s serviceability and manageability.

--------Our abstract -- TODO: merge above -----

%Motivation
Hardware-as-a-service (HaaS) enables customers to rent physical
machines on cloud platforms to execute their 
applications with near bare-metal performance.
% Problem
However, traditional hypervisor platforms used to host system virtual machines (VMs) 
are heavyweight and impose significant overheads 
in interrupt and I/O processing, making them unsuitable for use in HaaS platforms.
On the other hand, native execution of OS and applications limits 
a cloud provider's ability to migrate customer workloads for system maintenance
and failure recovery.
% Our contributions
In this paper, we propose the \fullname (\acro) to reduce
key virtualization overheads on HaaS platforms.
\name enables a traditional system VM to run atop a thin hypervisor 
and achieve near baremetal performance while retaining the 
administrator's flexibility to live migrate the VM.
A key feature of \name includes
direct delivery of timer and network interrupts to the VM without VM Exits, 
thus eliminating hypervisor-level emulation overheads in interrupt delivery.
% Implementation summary
We describe a prototype implementation on the KVM/QEMU hypervisor
that leverages Intel VT-d hardware support to achieve direct network and timer 
interrupt delivery with no hypervisor intervention, while
supporting live VM migration.
\end{abstract}


%\section{Introduction}

% Introduction motivates the readers with the following aspects.
% - Bare-metal cloud or hardware as a service (HaaS)
% - Server support for HaaS
% - Network support for HaaS
% - Apply the virtualization to enhance the manageability of
%   servers in a HaaS
% - Single-VM virtualization requirements
%   - Direct device assignment for all PCIe devices
%   - Direct interrupt delivery
%   - Migration of bare-metal server
%   - VM introspection for the security and better visibility
\mycomment{
Infrastructure as a service (IaaS), which was popularized by AWS's EC2 service~\cite{ec2}, has evolved and morphed into multiple forms over the last decade.
The basic compute unit for IaaS began as a {\em virtual machine} (VM), which represents a slice of a physical machine carved out by a hardware-abstraction-layer, called the hypervisor.
A {\em container} is another basic compute unit, where a common operating system (OS) delimits the addressable system resources (or namespaces) for a group of processes and enforces usage limits.
A more recent IaaS compute unit is a {\em function}, which comes with a complete operating environment consisting of an OS and a middleware layer, and is created on demand.  
}

%Lately, to avoid multi-tenancy and security issues, a physical machine itself is 
%treated as a basic compute unit in {\em bare-metal cloud service}~\cite{bms-wiki}.
%or {\em hardware-as-a-service} (HaaS). 
%Infrastructure as a service (IaaS), which was popularized by AWS's EC2 service, has evolved and morphed into multiple forms over the last decade.
%In the beginning, the basic compute unit for IaaS was a {\em virtual machine}, which represents a slice of a physical machine carved out by a hardware-abstraction-layer hypervisor.
%Then the basic compute unit could also be a {\em container}, which is pre-configured with an operating system and corresponds to a piece of a physical machine delimited by that OS.
%%A more recent option for IaaS's basic compute unit is a {\em function}, which comes with a complete operating environment constsing of an OS and a middleware layer, and is created on demand.  
%Lately, even a physical machine could serve as the basic compute unit. This type of IaaS is known as {\em bare-metal cloud service} or {\em hardware as a service} (HaaS). 
%In the past three years, we have been developing a HaaS operating system called {\em ITRI HaaS OS} or \sna.  
%The focus of this paper is on \sna's virtualization support that enhances the manageability and serviceability required of a modern bare-metal cloud service. 


Conventional multi-tenant cloud services~\cite{ec2,azure,gcp} enable
users to rent virtual machines (VMs) or containers to scale 
up their IT infrastructure to the cloud. However, virtualization
introduces both performance overheads and security concerns
arising from co-located workloads of other users.
To address this concern, cloud operators 
such as  IBM SoftLayer~\cite{softlayer} and Oracle~\cite{oracle},
have begun to offer bare-metal cloud service, or Hardware-as-a-Service (HaaS),
%In the case of traditional multi-tenant IaaS, cloud operators own and manage 
%the physical machines, which are shared among multiple users.
%In contrast, bare-metal cloud operators, 
which allow users to rent dedicated  physical machines.
HaaS clouds enables users combine the benefits of 
scaling up their operations in the cloud with having dedicated 
hardware; users are assured stronger isolation than multi-tenant clouds and 
bare-metal performance for critical workloads 
such as high-performance computing, big data analytics, and AI.
%Other use cases of bare-metal cloud services include a preferred hypervisor 
%or OS that is not supported by cloud operators or special hardware for which virtualization 
%is not sufficiently mature, such as 
%GPUs, SoC-based micro-servers, and application-specific FPGA accelerators.

However, common management functions available on multi-tenant clouds,
such as live migration and introspection-based 
application performance management, are difficult to 
duplicate on HaaS servers, because HaaS providers typically 
do not install any software on these dedicated servers.

To address this manageability gap of existing HaaS platforms, 
we have been developing a HaaS management system, called 
%TODO: uncomment later
%\fullname (\sna) 
IHO,
with the goal of enhancing the manageability and serviceability 
of bare-metal cloud services.
The focus of this paper is on IHO's virtualization support
in the form of a specialized hypervisor, called the {\em Single VM
hypervisor (\sna)}.


\figwtwocol{newarch}{14}{
The \na runs as a thin shim layer
to provide the HaaS VM with dedicated CPU cores, memory, and I/O devices.
All hardware interrupts, including device, local timers, and IPIs
are directly deivered to HaaS VM without hypervisor intervention.
\na coordinates with a HaaS agent in the guest to provide
manageability services. 
}

Figure~\ref{fig:newarch} shows the high-level
architecture of \sna, which runs as a thin hypervisor on each
physical server. \na is optimized to run one VM, called the 
{\em HaaS VM}\footnote{The \na could conceptually be an extension of a physical server's trusted BIOS.}, 
on which  a user install a preferred OS and applications. 

Unlike traditional hypervisors, which are designed to limit and control a VM's 
access to physical resources, \na is designed to maximize the HaaS VM's 
access to physical hardware.
During normal execution, the \na allows 
the HaaS VM to directly interact with physical I/O devices and processor hardware
without the hypervisor's intervention, 
as if it runs directly on a physical server with near bare-metal performance.
\na primarily provides value-added manageability 
features of conventional clouds, such as live
migration, introspection, and performance monitoring.
A small HaaS agent in the guest (a self-contained kernel module)
transparently coordinates these services with the \sna.
Specifically, \na provides the following features for a HaaS VM. 

%Bare-metal cloud operators provide a user with a physical data center instance (PDCI), which is 
%composed of a set of physical machines connected in a way specified by the user. 
%In the past three years, we have been developing a HaaS operating system called 
%
%
%
%
%A HaaS user or tenant makes a HaaS service request to \na by specifying a PDCI, which consists of 
%The HaaS offerings from cloud operators such as IBM (SoftLayer) and Oracle provide a user a physical data center instance (PDCI), which is composed of a set of physical machines connected in a way specified by the user. HaaS users prefer physical machines to virtual machines primarily because they want to make the best of the underlying hardware resources for workloads that do not need the flexibility afforded by virtualization, such as HPC computation, big data analytics or AI training.
%Other HaaS use cases include that users have a preferred hypervisor or operating system which is not supported by cloud operators, and 
%that users need special hardware for which virtualization is not sufficiently mature, such as ARM SOC-based micro-server and GPU/FPGA cluster.
%
%In the case of IaaS, cloud operators own and manage the physical machines.
%In contrast, for HaaS, cloud operators own the physical machines but users manage them. 
%This way, HaaS users are still able to enjoy the multiplexing benefits of cloud computing that are due to sharing of hardware and facilities.
%A HaaS user or tenant makes a HaaS service request to \na by specifying a PDCI, which consists of 
%\begin{itemize} 
%\parskip 0mm
%\itemsep 0mm
%\item A set of physical servers, each with its CPU/memory/PCIe device specification, and configurations on the BIOS, and PCI devices,
%
%\item A set of storage volumes that exist in local or shared storage, and are attached to the servers,
%
%\item A set of IP subnets that describe how the servers are connected with one another and to the Internet, and  
%
%\item A set of public IP addresses to be bound to some of the servers facing the Internet, and their firewall policies. 
%
%\end{itemize}
%\na processes each PDCI request by first making corresponding allocations for server, network and storage resources, and 
%then setting up the PDCI's required network connectivity.  Because a HaaS operator cannot install any agent software on the physical servers rented out on a PDCI, the only way for \na to programmatically build a virtual network that meets a PDCI's IP subset specification is to leverage the VLAN and VXLAN capabilities in modern network switches and routers by properly configuring them according to the network connectivity specification.  
%Moreover, \na allows a HaaS tenant to {\em remotely} check, configure, and update the firmware on the physical servers, as well as install the desired operating systems and applications 
%on them, in a way that minimizes human errors and the adverse side effects that come with these errors. 
%Finally, \na enables a HaaS tenant to monitor the hardware status of the physical servers and the network traffic among them with full visibility, without revealing anything associated with other co-located tenants. 
%
%
%For the HaaS use case in which a tenant installs an operating system (Linux or Windows) rather than a 
%proprietary hypervisor on the physical servers of its PDCI, 
%

\begin{itemize} 
%\setlength\itemsep{-0.04in}
\parskip 0mm
\itemsep 0mm

\item {\bf Direct access to dedicated hardware:} 
\na assigns a HaaS VM with 
dedicated hardware resources, namely 
physical CPUs, memory, local APIC timers, and passthrough I/O devices,
which the VM can access directly 
during its normal execution without hypervisor intervention. 
The \sna, however, maintains its ability to
regain control over physical resources when needed, such 
before live migration of the HaaS VM.


\item {\bf Direct delivery of all interrupts:}
\na enables a HaaS VM to directly receive all hardware interrupts from 
its assigned PCIe devices and CPU cores (for timer interrupts and IPIs) 
without interception and emulation by the hypervisor.
Direct interrupt delivery to the HaaS VM greatly
reduces interrupt processing latency, which is imoportant for running 
latency-critical applications on bare-metal clouds.
In contrast, existing approaches~\cite{vfio,postedinterrupt,amit:2015,tu:2015}
focus only on direct delivery of PCIe device interrupts.


\item {\bf Seamless live migration:}
\na provides on-demand live migration of a HaaS VM having direct access to 
physical network devices and local APIC timers. 
%through coordinated switching to para-virtual I/O during migration.
\na seamlessly disables the HaaS VM's physical hardware access at the 
source machine before migration and 
re-establishes access at the destination after migration,
with minimal disruption to the VM's liveness and performance.
Unlike existing live migration~\cite{vfio-live-migration,blmvisor-journal,ondemand} 
approaches for VMs with pasthrough 
I/O access, \na does not require device-specific state capture and migration code 
nor does it require the hypervisor to trust the guest OS. 
%Yet \na matches the liveness and performance of traditional live migration.

\item {\bf Bare-metal performance:}  
To match bare-metal performance, \na implements mechanisms
for reducing the its CPU, I/O, and memory footprint
during runtime and minimizing interference with HaaS VM's execution.
\na has a small execution footprint, currently one 
CPU core and around 100MB of RAM, with room for further reductions.
Even with passthrough I/O support~\cite{intelvtd-paper,intelvtd-manual}, 
VM exits can still lower the I/O throughput and increase CPU usage of the host.
\na includes mechanisms to systematically eliminate
most VM exits for a HaaS VM to match bare-metal I/O performance and CPU utilization.
Other than during startup and live migration, the \na is not involved in the
HaaS VM's normal execution.

\end{itemize}

%With such a hypervisor installed on each physical server, \na can monitor each 
%server's internal user activities using VM introspection, and seamlessly migrate the user state 
%on the server from one physical machine to another. 

The rest of this paper is organized as follows.
We first describe the design and implementation of \sna,
specifically the detailed description of the above features.
Next, we evaluate the performance of our \na prototype
on the Linux and KVM/QEMU platform.
Finally, we discuss related work followed by conclusions.


%
\section{Related Work}
% Baremetal service
% - IBM SoftLayer
% - Oracle
% - Zenlayer
% - Vultr
In recent years, virtualization overheads and security concerns on 
multi-tenant IaaS platforms have led to the rise of a 
number of bare-metal or HaaS~\cite{softlayer,oracle,zenlayer,vultr,m2}
clouds which provision dedicated physical machines for customers
to provide native execution and I/O performance.
However, the absence of a virtualization layer 
limits the cloud provider's ability to manage, monitor, 
and live migrate~\cite{clark:2005,postcopy-osr}
customer workloads, leaving the customer responsible for 
these functions.
% Reducing virtualization overheads
On the other hand, virtualized IaaS cloud platforms~\cite{gcp,azure,ec2}
retain the cloud provider's ability to manage and live migrate customers'
VMs but also introduce overheads  due to
hypervisor-level emulation of hardware access by guests. To mitigate I/O-related 
virtualization costs, customers can provision VMs with 
direct device assignment~\cite{intelvtd}. 
However, doing so entails sacrificing the ability 
to live migrate VMs for load balancing and fault-tolerance
due to the difficulty of migrating the state of physical I/O devices.
\na aims to  support bare-metal server performance
while retaining the benefits of virtualization via a 
thin hypervisor.

%TODO container/process migration cannot migrate the OS state
% and limit the type of processes that can be migrated, such as 
% those without active network connections.
{\bf Live migration of bare-metal server:} 
On-demand virtualization~\cite{ondemand} presented an approach
to insert a hypervisor that deprivileges the OS at the 
source server just before migration and re-privileges it
at the destination after migration. However, this approach 
requires the hypervisor to fully trust the OS being migrated, 
the OS having full access to the hypervisor's memory and code.
Such lack of isolation rules out the use of many hypervior-level
cloud services that require strong isolation, such as VM introspection
and SLA enforcement. In contrast, with \na, the guest OS 
cannot access or modify hypervisor code or memory.
BLMVisor~\cite{blmvisor,blmvisor-journal} proposes live migration of VMs 
having direct access to physical devices by having the hypervisor
regulary intercept and capture low-level physical device states
during normal execution; the captured state is then reconstructed
at the destination after migration. This approach requires
implementing device-specific state capture and reconstruction code 
for each type of physical device, besides active 
interception of these operations by the hypervisor. 
In contrast, \na is device agnostic and does not require
the hypervisor to intercept guest I/O accesses.
Finally, unlike the above aproaches,
\na support direct delivery of timer interrupts 
to the guest OS without hypervisor intervention, besides supporting
live migration of guests having direct timer access.
\na seamlessly tears down the VM's
physical device and timer dependencies  at the source server before live migration and 
re-establishes these dependencies at the destination server after migration
with minimal disruption of the VM's execution.


{\em Reducing virtualization overheads:}
A number of techniques have been proposed to reduce virtualization
overheads in interrupt delivery by eliminating VM Exits to the hypervisor.
ELI~\cite{amit:2015} and DID~\cite{tu:2015} presented techniques
for direct delivery of I/O interrupts to the VM. 
Intel VT-d~\cite{intelvtd-paper,intelvtd-manual} supports a posted interrupt delivery mechanism~\cite{postedinterrupt}.
However, these techniques do not fully eliminate hypervisor's role
in the delivery of device interrupt.
Specifically, the hypervisor must intercept and deliver device interrupts
(using either virtual interrupts or inter-processor interrupts)
when the target VM's VCPU is not scheduled on the CPU
that receives the device interrupt. Further, idling guest VCPUs
waiting for I/O completion trap to the hypervisor, where a well-intended
halt-polling optimizations can inadvertently end up increasing CPU utilization.
In contrast, \na eliminates these virtualization
overheads in device interrupt delivery by not only leveraging
Intel VT-d posted interrupts, but also by isolating  hypervisor execution 
to a single physical core, and ensuring that physical 
cores assigned to the VM remain in guest (non-root) 
mode whenever the guest idles while waiting for I/O completion.

Furthermore, a key distinction of \na lies in LAPIC timer
access and timer interrupt processing by the VM.
None of the existing techniques eliminate hypervisor overheads 
when a VM accesses its LAPIC timer and nor do they 
support direct delivery of timer interrupts to the guest OS.
\na does so for the single-VM virtualization case by a novel use of Intel VT-d posted 
interrupt mechanism that allows the guest to directly set timer events
on LAPIC and receive timer interrupts without any VM Exits.

%TODO: Summarize Jailhouse

\mycomment{
\subsection{Jailhouse}
By far, Xen and KVM are the two Linux de facto hypervisors.
The are versatile and cooperate with QEMU to virtualize the
entire system. They utilize the modern hardware-assisted
virtualization and match the bare-metal CPU, memory and I/O
performance with the significantly reduced CPU overhead.
Nonetheless, there are some areas that these two
general-purpose solutions need further improvements. One such
area deals with the real time applications. A real-time
application needs to meet the minimal response time and/or the
worst latency. For example, the vehicular break system needs
to meet the minimal response time , when the driver presses
the break. Failing the requirement leads to a detrimental
consequence.

The jailhouse hypervisor is a partitioning hypervisor that
runs on the bare-metal and works closely with the
Linux\cite{sinitsyn:2015, ramsauer:2017}. It is not trying to
be the full-fledged KVM. Its responsibility is to partition
and assign the available hardware resource to the guests and
prevent a guest from interfering with the jailhouse or another
guest. Each guest has its own set of dedicated hardware
resource and do not share them. In the other words, there is
no overcommitment of resources. Moreover, the jailhouse is an
example of asymmetric multiprocessing design, which treats one
processor core differently from another. For example, one
processor can access the hard disk, while another accesses the
serial port. The jailhouse uses this design to create an
isolated environment, called a cell. When the jailhouse boots
up, it creates a Linux cell or root cell, containing all the
processor cores, memory and hardware resources. Before the
jailhouse boots up a guest, it creates a new cell and allocate
the requested hardware resources to the guest or inmate. This
set of hardware resources is dedicated to the guest. This
partitioning design indicates the jailhouse does not emulate
the devices or manage resources for the guests.

To passthrough the devices, the jailhouse requires the
hardware-assisted virtualization. On the x86, it is the VT-x
and VT-d. For the LAPIC registers, the jailhouse handles the
accesses differently depending on whether the host supports
the xAPIC or x2APIC mode. If the host only supports the xAPIC
mode, the jailhouse traps all the guest's accesses to the
LAPIC registers. Even if the guest would like to access the
LAPIC using the MSR interface in the x2APIC mode, the
jailhouse traps and emulates it on the top of xAPIC mode. If
the host supports the x2APIC, the jailhouse only traps the
access to the interrupt command register. ICR is used to send
IPIs to other process cores. The trap is required, so the
jailhouse can prevent the malicious guest from disturbing
other guests. In terms of the interrupt handling, the external
interrupts are delivered directly to the guest, which handles
them through the guest IDT. One exception is the non-maskable
interrupt. The jailhouse uses NMI to regain the controls of
guest CPUs.

Thus, the interrupts from the assigned devices and timer
interrupts are delivered directly into the guest without any
indirection, while the guest can have a fully control over its
assigned devices and LAPIC timer. These features help to meet
the real-time application requirement for the latency and
response time or the long-running computations. Furthermore,
the jailhouse confines the guest in its own cell and results
in security enhancement and no resource overcommitment.

% Previous work
\subsection{Exitless Interrupt}
Exitless Interrupt (ELI) is based on the following
conditions~\cite{amit:2015}. First, the guest has its own set
of dedicated cores. Second, the guest runs the I/O intensive
workload with the directly assigned SR-IOV devices. Third, the
number of interrupts that the guest receives from the assigned
devices is proportional to the guest execution time. Thus, the
ELI delivers the assigned interrupts to the guest directly and
non-assigned interrupts to the VMM. This is achieved by the
shadow interrupt descriptor table.

When the guest runs in the guest mode, it runs with this
shadow IDT prepared by the ELI instead of its own IDT. When
the logical processor in the non-root mode receives an
external interrupt, it screens the interrupt by the shadow
IDT. If the interrupt is from the assigned device, it
dereferences the corresponding table entry and invokes the
guest's ISR. Otherwise, it traps to the host, which handles
the non-assigned interrupts. To make such a distinction, the
ELI copies the guest's IDT, including the guest's exceptions
and assigned devices, to the shadow IDT. The ELI preserves
the device interrupt priorities and keeps the interrupt vector
numbers of each device the same between the corresponding
guest and host interrupt handlers. It marks guest entries as
present and the rest of entries as non-present. Moreover, the
ELI configures the logical processor to force exit on the
non-present exception. When the host handles the non-present
exception, it needs to inspect the exit reason. If it is due
to the non-assigned physical interrupt, it converts the
exception back to the original interrupt vector and invoke the
respective ISR. For the virtual interrupts from the emulated
device, the ELI marks them as the non-assigned interrupts.
After the trap, the host enters the special injection mode
that configures the logical processor to exit on any physical
interrupts and the guest to use its own IDT. The host injects
the virtual interrupt to the guest.

When the guest's ISR finishes handling its interrupt, it
updates the LAPIC EOI register and triggers the VM exit. The
VM exit can be disabled through the MSR bitmap, when
configuring the VMX module with the x2APIC programming
interface. Since the guest does not distinguish between the
injected virtual interrupt or the assigned interrupt, it
updates the EOI LAPIC register for all cases. This should not
be the case for the virtual interrupt from the host emulated
device. When the host operates in the special injection mode,
it traps the EOI write to the host. Once the guest finishes
all the pending EOI writes for the virtual interrupts, the
host leaves the special injection mode.

Thus, the ELI delivers the interrupts from the assigned
devices directly and virtual interrupts indirectly, while
preserving the interrupt priorities. It effectively reduces
the VM exits due to the assigned interrupts to 0 and handles
the EOI signals properly.

\subsection{Direct Interrupt Delivery}
Direct Interrupt Delivery (DID) solves the two
challenges~\cite{tu:2015}. First, the interrupts are directly
delivered to the guest without the hypervisor intervention on
the delivery path. Second, the guest completes the end of
interrupts without the VM exit. The two challenges are divided
into the following sub-tasks. First, if a guest is running,
the interrupts from the SR-IOV devices, timers and emulated
devices are delivered directly. Second, when the target guest
is not running, its interrupts are delivered through the
hypervisor. Third, the priority of physical and virtual
interrupts are preserved. Fourth, the number of interrupts
that the host needs to complete is zero. In addition, the DID
supports the unmodified guests.

The DID routes the interrupts to the guest or host
appropriately by configuring the interrupt routing and
remapping table from the IOAPIC and IOMMU respectively. For
the SR-IOV device interrupts, the DID disables the VM-exit due
to external interrupts. Consequently, the interrupts are
delivered normally through the guest interrupt descriptor
table. If the virtual processor is running, the device
interrupt is directly delivered. If the virtual processor is
rescheduled by the host, the interrupt is delivered to the
host through the non-maskable interrupt. The host injects the
corresponding virtual interrupt, when the virtual processor is
re-scheduled on the logical processor.

On the modern x86 architecture, each processor has its own
LAPIC. LAPIC generates timer interrupts, which are not routed
by the IOAPIC and IOMMU. Instead, the LAPIC delivers the timer
interrupts to its associated processor. After the guest
handles the timer interrupt, it sets up the next timer event
by configuring the LAPIC timer. This requires the host's help.
The DID ensures that the guest's timer interrupt only delivers
to guest instead of other user-level processes. The DID
installs the software timer on the host's dedicated core.
After the host receives the guest's timer interrupt, it
delivers the physical timer interrupt through the IPI. The
host delivers the timer IPI, only when the virtual processor
is active. If the virtual processor is preempted, the host
delivers the timer IPI, when the virtual processor is
scheduled on another logical processor.

The DID delivers the virtual interrupts as the IPIs, which are
treated as the external interrupts. Each QEMU virtual device
is represented by a thread and runs on its dedicate core.
Before delivering the virtual device interrupt, the device
thread need to check if the virtual processor is active. If
the virtual processor is running, the thread delivers the
virtual device interrupt through the IPI. Since the DID
disables the VM-exit due to external interrupts, the guest
receives the device interrupt directly. If the vCPU is not
running, the host receives the interrupt on the behalf of
guest. The host injects it to the appropriate guest as the
virtual interrupt, when the virtual-processor is scheduled on
the logical processor.

When the ISR completes, it instructs the logical processor to
update the LAPIC end-of-interrupt register. In the DID scheme,
the direct EOI write has the following considerations. First,
if the handler of virtual interrupts directly updates the EOI
register, the LAPIC may thinks there is no pending interrupt.
Second, it may also think the pending interrupt is completed,
which is still ongoing. Third, LAPIC may dispatch the lower
priority interrupt to preempt a higher-priority interrupt. The
root cause is that the virtual interrupts are not visible to
the hardware LAPIC, when they are injected via IRR and ISR.
The DID solves this problem by converting the virtual
interrupts as the IPIs, while disabling the VM-exit due to the
EOI writes. If there are multiple virtual interrupts, the host
issues them as the IPIs one at a time.

Thus, the DID delivers the external interrupts from the
assigned devices, timer and emulated devices directly into the
guest, while preserving the interrupt priorities. The DID
zeroes the number of VM exits due to the interrupts. Moreover,
it is not required to modify the guest kernel. This is one of
the advantages over ELI and ELVIS.
}

%\section{Virtualization Support for HaaS}
% Virtualization support for HaaS include the following items
% CPU idleness detection and processing
% Direct NIC assignment
%   - VFIO
%   - CPU optimization for the baremetal network performance
% Direct interrupt delivery
%   - Posted-interrupt mechanism
%   - Direct timer-interrupt delivery
%     - periodic vs aperiodic timer interrupt
%     - one-shot vs periodic hardware timer interrupt
%     - guest-level access to the PIR page
%     - Eliminate the hardware lock, when the guest accesses
%       the PIR bit
%     - Spurious timer interrupts
% Seamless baremetal virtual machine migration
%   - NIC bonding with the hot plug/unplug operation migration
%   - DTID during the migration

\figw{architecture}{9}{Architecture of \na for single-VM virtualization (SVMV). 
The SVMV hypervisor is restricted to a single CPU core and provides
manageability services such as live migration and monitoring.
The HaaS VM runs on all the remaining CPU cores with direct access to I/O devices 
and local timers, including direct delivery of interrupts.}

%The design goal of \na is to eliminate the hypervisor from
%the guest I/O path while achieving bare-metal performance in
%the guest. To achieve this \na considers direct device
%assignment to the guest using Intel's VT-d support. First, the
%guest meets the bare-metal network and disk I/O performance
%with direct device assignment. We further apply optimizations
%to reduce the CPU utilization by hypervisor. Second, the timer
%interrupts are transformed into the posted interrupts, which
%are directly delivered to the guest by the logical processor
%without causing any VM exits. Finally, we present the
%migration of virtual machine with directly assigned devices.

To enhance the manageability of physical servers offered in a bare-metal cloud service, \na intalls 
a SVMV  hypervisor on each physical server  to create a single VM (called the HaaS VM), on which 
a \na user then installs a preferred OS and applications. 
Figure~\ref{fig:architecture} shows the high-level architecture of \na.
This  hypervisor is primarily meant to provide management functionalities, such as 
physical machine migration and application performance monitoring, rather than virtualization 
of server resources, and therefore is largely not involved in the I/O operations in which a HaaS VM participates.
%This section describes in detail the main design issues of this SVMV hypervisor, 
%which is based on Linux/KVM/QEMU, and our corresponding solutions.

I/O operations and interrupt processing using traditional VMs incur higher overheads than bare-metal execution. 
I/O operations issued by a VM typically trap into the hypervisor via VM exits for emulation.
Likewise, external device interrupts and local timer interrupts 
to the CPU running a VM result in VM exits for emulating virtual interrupt delivery.
Each VM exit is expensive, since it requires saving the VM's execution context upon exit,
emulation of the exit reason in hypervisor mode, 
and finally restoration of the VM's context before re-entry into guest mode.

The main technical challenge of providing the guest OS with the illusion of running on 
a bare-metal server is that its interactions with I/O devices, such as network 
interface card (NIC) and disk controller, must be direct without going through any intermediary.
\na allows the sole VM on the physical machine to directly interact with PCIe I/O devices and 
timer hardware by leveraging Intel VT-d~\cite{intelvtd-paper} and 
Linux Virtual Function I/O (VFIO)~\cite{vfio} mechanisms. 
To match bare-metal I/O performance, \na implements mechanisms 
for reducing the hypervisor's CPU, I/O, and memory 
footprint during runtime to minimize interference with guest operations.
In addition \na also enables direct delivery of both device interrupts
and timer interrupts to a HaaS VM by leveraging
the posted interrupt mechanism in Intel VT-d.
Direct interrupt delivery greatly reduces the interrupt processing latency and is
particularly useful for real-time applications running on bare-metal servers. 
When the VM must be live migrated, \na temorarily switches the VM to use
para-virtual I/O and virtualized timer, completes the migration
and, at the destination, switches the VM back to direct I/O device and direct 
timer access, all with minimal disruption of VM's workload.

In this section, we first  provide background on direct device access using 
Intel VT-d and Linux KVM/QEMU~\cite{kvm}, followed by the description of design challenges 
and our solutions for achieving bare-metal performance while supporting live migration.

\subsection{Background: Intel VT-d and VFIO}

%\subsection{Direct Interactions with PCI Devices}
Intel VT-d~\cite{intelvtd-paper} provides processor-level support for direct 
and safe access to hardware I/O devices by VMs running in non-root mode.
Virtual function I/O (VFIO)~\cite{vfio} is a Linux software framework that enables user-space
device drivers to interact with PCIe devices directly without involving the Linux kernel. 
In general, there are four types of interactions between an OS and the PCIe 
I/O devices under its management:
\begin{enumerate} 
\parskip 0mm
\itemsep 0mm
%\setlength\itemsep{-0.04in}
\item At the system start-up time, the OS probes and enumerates the PCIe devices existing in the system, and assigns a device number, configuration space address range, and interrupt to each discovered device.

\item The OS reads and writes a device's configuration register space and the memory regions specified in its base address registers (BAR), including setting up DMA operations.

\item A DMA engine moves data blocks between main memory and PCIe devices according to the DMA commands set up by the OS.

\item A device interrupts the OS for its attention to certain hardware events, such as new packet arrival or transmision completion.

\end{enumerate}   
Except the probe/enumeration operations, 
\na enables a guest OS to interact directly (without VM exits) with NICs and disk controllers
for the other three types of device interactions. 

\subsubsection{Direct DMA Operations Using VT-d and VFIO} 
Typically, a PCIe device driver communicates with its device using programmed I/O (PIO) or
memory-mapped I/O (MMIO) operations against the memory areas associated with to the device. 
Essentially, VFIO makes the configuration register space and the memory regions of
each PCIe device accessible to user-space processes.
%VFIO retrieves a PCIe device's device-specifc information such as BARs from its
%configuration register space, and maps them into distinct regions in a special file. 
%By reading and writing specific regions of this special file, a user-space program 
%is able to directly interact a particular PCIe device. 
%VFIO provides user-space programs a PCIe device read/write API, and converts these API calls
%into file read/write operations against the special PCIe device file.
%
%The userspace driver uses
%the device file descriptor and offset to access each region
%and retrieve the device information. The VFIO decomposes the
%physical device to a software interface. Such software
%interface is turned into the assigned device by QEMU.
%Essentially, the device read and write handler in the QEMU
%memory API is forwarded to the VFIO read and write handler.
%Not all accesses to a PCIe device's associated memory regions by user-space processes are direct, 
%because some parts of a PCIe configuration register space are privileged, such as 
%message signalled interrupts (MSI), BARs, and ROM,  and accessing them 
%requires KVM's or QEMU's emulation. 
%
%Nonetheless, accessing some parts of PCI configuration
%PCI configuration space is
%not handled as memory regions in QEMU. Some of the accesses to
%the PCI configuration space is passthroughed directly, while
%others, such as MSI, BARs, and ROM, need to be emulated.
%
Thus a user-space processe can use DMA operations to move data directly between a PCIe device and 
a region of its virtual address space.
The address of the source or destination buffer of a DMA operation resides in 
the {\em device address space} of the PCIe device involved in the DMA operation.
Intel's VT-d architecture provides an IOMMU~\cite{ben:2006}, that maps a 
PCIe operation's {\em device address} into a {\em physical address}, 
which is used to access main memory. 
\mycomment{
When a user-space program sets up a DMA operation, it only knows the virtual address of the associated buffer and thus places
this address in the DMA command, essentially treating this virtual address as a device address. 
Given such a DMA command, VFIO consults with the page table associated with this user-space program to extract the physical address corresponding
to the DMA buffer's virtual address, and creates an IOMMU entry that links the buffer's virtual (device) address with its associated physical address. 
When this DMA command is executed at run time, its buffer's device address is correctly translated to its corresponding physical address.
 }
  
The key field of each IOMMU entry includes a PCIe device number and a device address.
Therefore, a machine's IOMMU may map the same device address into different physical addresses when the device address is associated with different PCIe devices.
By controlling which user-space programs can access which PCIe devices, VFIO is able to leverage IOMMU to effectively prevent a user-space program 
from using DMA operations to corrupt the physical memory areas owned by other user-space programs.

To KVM, a VM runs as part of a user-space process  called QEMU~\cite{qemu},
specifically, as guest-mode threads within QEMU's virtual address space..
QEMU uses VFIO to configure a VM to directly access the device address space 
its assigned PCIe devices without emulation via KVM or QEMU.
In contrast, in para-virtual Vhost~\cite{vhost-net} I/O architecture, 
each incoming or outgoing I/O operation (network or block I/O) must go through 
a special hypervisor-level thread (called the vhost worker thread), 
which emulates a virtio~\cite{russell:2008} device in the kernel and 
therefore keeps QEMU out of the data plane.
However, QEMU still is responsible for such control plane processing as setting up, 
configuring, and negotiating features for an in-kernel virtio device.  



\mycomment{
Second, the VFIO programs the IOMMU to transfer the data
between the userspace driver and device in a hardware
protected manner. The VT-d IOMMU provides the device isolation
using the per-device IOVA and paging structure. The virtual
virtual address requested by the device is translated to the
physical address through the set of paging structures by
IOMMU. In the case of VM, the address space of assigned device
is embedded within the guest address space. The IOMMU is
programmed to translate such IOVA to the host physical address
which is mapped to the guest address space. Such translation
is both realized and protected by the IOMMU.

Third, the VFIO has a mechanism to describe and register the
device interrupt to signal its userspace driver. When the VM
accesses the device configuration space, it is trapped through
QEMU. QEMU configures the IRQs by the VFIO interrupt ioctls
and sets up the event notifiers between the kernel, QEMU and
guest. When the kernel signals the IRQ to QEMU, QEMU injects
it into the VM. The interrupt signaling is further speeded up
by moving QEMU out of the way. KVM supports both ioeventfd and
irqfd. ioeventfd registers PIO and MMIO regions to trigger an
event notification, when written by the VM. irqfd allows to
inject a specific interrupt to the VM by KVM. Once ioeventfd
and irqfd are coupled together, the interrupt pathway remains
in the host kernel without exiting to the userspace QEMU.
Using the VT-d, the KVM and QEMU is completed removed from the
signaling path way. It enables the direct interrupt delivery
from the assigned device to its VM without a VM exit.
}


\subsubsection{Posted Interrupts}

{\bf Conventional Interrupt Delivery:}
\mycomment{
Our design uses the hardware-assisted posted interrupt
mechanism and achieves the direct interrupt delivery of
assigned devices and local timers without the intervention of
hypervisor. It reduces the hypervisor CPU utilization and
dedicates the CPU time to the guest. Under the normal
circumstances, the guest can not handle physical interrupts
without the hypervisor. 
}
In the conventional architecture, when an interrupt is delivered to a CPU core,
it triggers a VM exit if the CPU core is not in the root mode and control is transferred to the hypervisor. 
%Saving and loading the execution context between the root and
%non-root mode waste the CPU cycles. Second, the host needs to
Then the hypervisor examines the cause of the interrupt. 
If the interrupt is meant for a VM scheduled on the CPU core,
the hypervisor delivers this interrupt as a virtual interrupt to the target VM
next time when it is scheduled on the CPU core; otherwise the hypervisor handles
the interrupt on its own.
After a VM completes servicing an interrupt, it writes to the EOI (End of Interrupt)  register 
to signal to the hardware that the interrupt in question has been handled. Because the EOI register is 
a privileged resource, every EOI register write causes a VM exit.
Therefore, processing of each physical interrupt costs at least two VM exits.
This per-interrupt overhead is too expensive for network-intensive applications that 
are designed to process multi-million packets per second. 

\mycomment{
and handle the physical interrupt. If the physical
interrupt is meant for the guest, the host needs to deliver it
as the virtual interrupt upon the next VM entry. Otherwise,
the host handles it and schedules the next VM entry. Third,
when the guest's interrupt handler finishes, it writes to the
EOI register. Such write operation may induce the VM exit.
Since the guest is not aware of the distinction between the
physical and virtual interrupt, it signals the completion of
interrupt in the same way. After the guest handles the virtual
interrupt, its EOI update is normally emulated by the host.
Fourth, the host may need to use CPU cache and reduce the time
to handle the physical interrupt. This introduces the CPU
cache pollution.
}

%In the X86 architecture, 
A VM may experience two types of interrupts: {\em external} and {\em local} interrupts.
External interrupts originate from external I/O devices, such as network card or disk controller.
When these I/O devices generate a hardware interrupt, this signal first goes to an IOAPIC, which, through an {\em Interrupt Redirection Table}, 
converts the hardware interrupt into an interrupt message that contains a vector number and is sent to a particular CPU core.
There is a local APIC associated with each CPU core to field interrupts sent to the CPU core.
In the Intel VT-d architecture, all interrupts sent to any local APIC are intercepted by  an {\em Interrupt Remapping Table} in the IOMMU unit, 
which provides a similar functionality to that of an Interrupt Redirection Table
to those external interrupts that do not come from an IOAPIC, e.g., message signaled interrupts from PCI devices.

{\bf Posted Interrupt Delivery:}
The Posted Interrupt mechanism~\cite{intelvtd-paper,intelvtd-manual} allows a CPU core that is running in the non-root mode to receive and handle interrupts with a specific vector
(Post Interrupt Notification or PIN vector) directly without involving the hypervisor.
When a CPU core handles a PIN interrupt, it examines a bitmap data structure associated with the CPU core called {\em Virtual Interrupt Request Register} (vIRR),
which is copied from another privileged data structure also associated with the CPU core called {\em Posted Interrupt Requests} (PIR) as a side effect of an PIN interrupt delivery,  
to determine the vectors of the interrupts behind the PIN interrupt, and processes these interrupts one by one according to their vectors and priorities.
The PIR bitmap is part of a per-VCPU data structure called {\em Posted Interrupt Descriptor} (PID), which in addition contains an Outstanding Notification (ON) flag, which, when set, indicates that there is a PIN interrupt pending.  
The address of the PID and the PIN vector associated with a CPU core are both contained in the virtual machine control structure (VMCS) associated with the VCPU running on the CPU core.

%\subsubsection{Direct External Interrupt Delivery}
To deliver an external interrupt directly to a VM, KVM sets up an Interrupt Remapping Table (IRT) entry for that external interrupt as follows.
First , it sets the IRT entry's IM bit to 1, which means that any interrupt matching this entry is to be delivered via the posted interrupt mechanism.  
Then, it ses the IRT entry's PID address field to the PID address associated with this entry's target CPU core. 
When an external interrupt arrives at the IOMMU and matches an IRT entry, the 
hardware first locates the PIR bitmap in the target CPU core's PID, then sets the bit in the PIR bitmap corresponding to the external interrupt's associated vector,
converts this interrupt into an interrupt labelled with the PIN vector, and finally delivers this PIN interrupt to the target CPU core's local APIC.
Because the external interrupts ``pretend'' to be a PIN interrupt, the target CPU core processes them directly without causing any VM exit.

When a VM completes the service of an interrupt, it needs to clear the EOI register to indicate to the 
associated local APIC that it is done with the interrupt. Because the EOI register is a privileged resource, accessing the EOI register
would normally requires a VM exit, which would add to the interrupt processing overhead. 
Hence, by default, when configuring a VM, KVM disables all EOI-triggered VM exits by setting the corresponding fields in the VM's VMCS.
%To allow a VM to avoid a VM exit due to clearing the EOI register after handling an interrupt of vector X,  KVM clears X's corresponding bit in 
%the {\em EOI exit bitmap} in the VM's VMCS. As a result, at run time after this VM completes servicing an interrupt of vector X and clears the EOI register via a memory mapped 
%intreface (called APIC-access page), no VM exit occurs.   
%Helped with the posted interrupt and EOI virtualization mechanism, a HaaS VM is now able to process any external interrupt directly without triggering any VM exit.

 
\mycomment{
VT-d supports the posted-interrupt capability and deliver the
external interrupts directly from the I/O devices and external
controllers without the cost of VM exits and the hypervisor
intervention. Before utilizing such feature, the system
software needs to define the posted interrupt notification
vector. The PIN signifies the incoming external interrupt from
the assigned device is subjected to the posted-interrupt
processing. The processing is achieved by updating the
posted-interrupt descriptor dynamically. When the VMCS is
actively used by the logical processor in the non-root mode,
it is prohibited to update its data structures. The PID is the
exception. Nonetheless, there is one requirement that the PID
modifications must be done using locked read-modify-write
instructions. Here is another benefit of posted-interrupt
support. When the virtual processor is scheduled on another
VCPU, the VMM can co-migrate its interrupts from the assigned
devices by setting the corresponding bits in posted-interrupt
register of PID.

The posted-interrupt support is accomplished in three general
steps. First, the VMM programs the interrupt-remapping
hardware with the mapping between the external interrupt and
virtual interrupt. Second, when the external interrupt is
delivered to the interrupt-remapping hardware, it sets the
outstanding bit and corresponding bit of virtual interrupt in
the posted-interrupt register of PID. It generates the PIN.
The IOAPIC delivers the PIN to the appropriate LAPIC. Third,
PIN notifies the logical processor that it is the
posted-interrupt event. The logical processor starts the
posted interrupt processing and delivers the virtual interrupt
without any VM exit.

The posted-interrupt processing is described in the following
steps. First, when the external interrupt is delivered to the
guest's processor, it is acknowledged by the LAPIC. LAPIC
provides the processor core the interrupt number. Second, if
the physical interrupt is equal to the PIN, the logical
processor starts the posted interrupt processing. Third, the
processor clears the outstanding notification bit from the
posted-interrupt descriptor. Fourth, the processor
acknowledges the EOI. Fifth, the processor updates the vIRR by
synchronizing it with the PIR. Sixth, the processor acquires
the next request virtual interrupt. It updates RVI by the
maximum of previous RVI and highest index of bits set in PIR,
before it clears PIR. Seventh, the processor evaluates the
pending virtual interrupt. Eighth, the processor delivers the
virtual interrupt.
}


\subsection{Achieving Bare-metal I/O Performance}
\figw{virtualization_overhead}{9}{Virtualization overheads in I/O operations due to VM exits in (a) para-virtualized I/O, (b) pass-through I/O, and (c) optimized pass-through I/O in \sna.}
%TODO: Describe I/O challenges here
%	Why CPU utilization increases
% 	VM Exit overhead
While using the VT-d and posted-interrupt mechanisms removes 
the hypervisor of from I/O data plane path of the VM, these alone are
not sufficient to reduce hypervisor overheads. 
To ensure that a VM using VT-d and VFIO can achieve bare-metal 
I/O performance and with least overheads, \na implements 
the following overhead reduction mechanisms

{\bf Dedicated CPUs:}
First, the hypervisor is assigned one dedicated physical CPU core
(CPU 0 in our implementation)
where it executes all of VM management operations. 
The HaaS VM is assigned all the remaining physical 
CPU cores by pinning the guest VCPUs to 
dedicated physical cores one-to-one~\cite{amit:2015}. 
This prevents the hypervisor's threads 
from competing with the VM's VCPU threads and 
also reduces the need to re-route interrupts
from the VM's assigned devices.
To avoid unnecessary VM exits on the HaaS VM's CPUs,
all external interrupts which are not directly handled
by the Haas VM, are delivered to the hypervisor's physical 
CPU by the IOAPIC.

{\bf Disabling  HLT-triggered VM exits:}
The second optimization relates to reducing hypervisor's CPU utilization
under the following scenario.
When handling frequent I/O operations, 
a guest VCPU  may become idle for brief intervals, and 
run a special idle thread.
On X86, this idle thread consists of a loop of {\tt HLT} instructions, 
each of which is {\em intended} to place the CPU in the C1 energy-saving mode 
until an external interrupt occurs.
However, since {\tt HLT} is a privileged instruction, it triggers a 
VM exit when executed by the guest.
To emulate the {\tt HLT} instruction, currently, the KVM hypervisor
polls (busy waits) for external events for a short time 
(adaptively between 0-200ns). If no events occur during this 
polling period, the guest VCPU is blocked and the physical CPU 
is placed in the low-power state.

The problem we observe is that, when the VM sends or receives network packets
at a high rate, the hypervisor-level CPU utilization increases
dramatically. This is because frequent {\tt HLT}-triggered VM exits
result in frequent invocation of the event polling loop in the hypervisor. 
Often, the physical CPU re-enters the guest mode directly 
from the polling loop as soon as the next network packet 
or interrupt arrives for the VM, without ever entering the low-power mode.

Although peak network throughput is not reduced when 
using VFIO (see Figure~\ref{fig:network_bandwidth}), 
both VM exits and polling loop cause significant 
CPU utilization by the hypervisor 
(see Figure~\ref{tab:cpu_utilization_40gbps}).
{\em Ironically, this overhead occurs even when using pass-through I/O and 
posted interrupts which were meant to reduce hypervisor overheads.}
Eliminating this event polling alone is insufficient because {\tt HLT}-triggered VM exits 
are also expensive by themselves; the guest VCPU thread is blocked and
must make a trip through the hypervisor's CPU scheduler before 
re-entry into guest mode.

Since \na is designed for a single VM setting, {\em we choose to
disable {\tt HLT}-triggred exits altogether}
by modifying the corresponding execution control fields in the VM's VMCS.
As a result, whenever a CPU becomes idle, no {\tt HLT}-triggered
VM exits occur, the CPU lowers its clock frequency while in guest mode, 
the total energy consumption is reduced, and importantly, 
hypervisor code is not invoked in I/O data path. 

Another interesting side effect of disabling {\tt HLT}-triggered VM exits
is that the HaaS VM is observed to receive more interrupts
from its direct-assigned device, possibly due to 
faster EOI response and hence less interrupt coalescing by 
the hardware, similar to the behavior on a bare-metal OS.

%
%First, the direct assigned network card and disk drive under
%the VFIO framework removes the host from the forward I/O path.
%%The guest has the control over the assigned devices and avoids
%the virtualization overhead, when accessing the device control
%registers and performing the
%DMA~\cite{sdm:2018,intelvtd-manual,williamson:2016}. After the
%assigned device services the request, it delivers the
%interrupt to the guest and triggers the VM exits due to the
%external interrupts and EOI respectively. Using the VT-d
%posted-interrupt and VT-x APIC virtualization~\cite{postedinterrupt},
%the guest handles the device interrupt and updates the EOI
%without any VM exit. Thus, the host is completed removed from
%the guest I/O path for the passthrough devices. Nonetheless,
%


\subsection{Direct Local Interrupt Delivery} 
\label{subsubsec:shared_pid_dtid}
\figw{dtid}{9}{Architecture for direct delivery of timer interrupts in \sna. 
%TODO: mention disabling WRMSR VM Exits.
The SVMV hypervisor configures each guest LAPIC to transform the local timer 
interrupt into a PIN interrupt and allows the guest
to directly update the PID page.}

Unlike an external interrupt, a local interrupt is delivered to a CPU core's local APIC without going through the IOMMU's Interrupt Remapping Table.
Therefore, such local interrupts cannot be delivered to a VM via the posted interrupt mechanism.
In this section, we describe how to achieve direct delivery of two types of local interrupts, {\em timer interrupt} and {\em inter-processor interrupt} (IPI).

{\bf Timer Interrupts:}To enable a timer interrupt to be delivered directly to a HaaS VM running on a CPU core,
\na first leverages the {\em local vector table} in the local APIC to turn each timer interrupt into a PIN interrupt, and 
then programmatically sets a particular bit in the CPU core's PIR bitmap to indicate to the target VM that underlying the PIN interrupt is a timer interrupt.

If the PIR bit associated with a timer interrupt is set way ahead of the interrupt's next expiration time, 
the interrupt's target VM may receive spurious timer interrupts. 
For example, suppose the next expiration time of a timer interrupt is $T_1$, and a NIC interrupt is delivered via the posted interrupt mechanism to the target VM at $T_2$, where ${T_2} < {T_1}$.
When the target  VM processes the PIN interrupt triggered by the NIC interrupt, it finds in the vIRR bitmap that two bits are turned on, one for the timer interrupt and the other for the NIC
interrupt, {\em even though the timer interrupt is spurious} because the real local timer has not yet expired.

To solve this spurious timer interrupt problem, a HaaS VM's guest OS is modified to keep track of the next expiration time of every timer interrupt, ignore an ostensible timer interrupt when  
the current time is substantially smaller than the next expiration time, 
and set the timer interrupt bit in the PIR because the PIR bitmap is cleared 
after a PIN interrupt is delivered.
The last step is required to ensure that a timer interrupt that is considered spurious and thus ignored because of an external interrupt,  still has a chance to be delivered directly when the timer truly expires.  

To avoid VM exit when a HaaS VM modifies the PIR of the CPU core it runs on, 
\na allocates a separate page to house the PIR and makes the page accessible to the VM.
With this set-up, a HaaS VM can write to the PIR of the CPU core it runs on without triggering any VM exit.

After a timer interrupt handler services a timer interrupt, it may need to configure the next timer expiry by updating the initial counter (TMICT) register of the local APIC timer, 
which would normally cause a VM exit. To avoid this VM exit, \na leverages VT-d's hardware-assisted APIC virtualization by properly configuring the MSR bitmap in the associated VMCS. 
As a result, the KVM's  intercept of any TMICT MSR update is disabled. When a VM writes to the TMICT, the change is written to the associated register directly without triggering any VM exit.     

Finally, from a security viewpoint, control over the timer interrupt is considered important for a hypervisor or OS to maintain control over all physical CPUs.
In \na, although the HaaS VM can directly control the timer hardware on its assigned CPUs, the SVMV hypervisor can still regain 
control over the CPUs when needed, such as before live migration. 
The hypervisor always controls the CPU0 and its local timer. 
To regain control over other physical CPUs assigned to the VM, the hypervisor 
simply disables direct timer access for the VM by reconfiguring the IOMMU and VMCS to 
disable PIN and trigger VM exits for timer interrupts; the hypervisor then delivers
emulated virtual interrupts to the guest.

Combining all the above techniques, \na is able to deliver local timer interrupts 
directly to a HaaS VM and have them properly serviced, 
all without causing any VM exits and while maintaining control over all physical CPUs.


{\bf Direct IPI Delivery:} 
Just as with direct delivery of timer interrupts, the posted interrupt mechanism can be used for direct delivery of inter-core IPIs.
Here we describe the design of direct IPI delivery mechanism which is currently under development in our prototype.
When a source CPU core sends an IPI to a destination CPU core, the source CPU core configures the Interrupt Command Register (ICR) in its local APIC, 
and, after the low double-word of the ICR is written to, triggers an interrupt message to be sent to the destination CPU core's local APIC through an inter-APIC system bus.
To enable an IPI to be delivered directly to the HaaS VM, \na configures the source CPU core's ICR so that the resulting interrupt message carries a PIN vector,
and then sets a certain bit in the PIR of the destination CPU core to indicate that the interrupt behind the PIN interrupt is an IPI.

 



\mycomment{

The longer the guest stays on its CPU, the more local timer
interrupts it receives. The goal is to let the guest have its
dedicated CPUs. Our design does not only directly deliver the
timer interrupts to the guest, but also take one step further
by allowing the guest update its next timer event directly.

The local timer interrupt is delivered to the guest and
results in two scenarios. First, the timer interrupt is meant
for the guest. It induces the VM exit and the control is
transferred back to the host. The host handles the timer
interrupt and injects the virtual timer interrupt to the
guest. When the guest receives the virtual timer interrupt, it
services the timer interrupt and set up the next timer event
by updating the LAPIC timer initial count register through the
x2APIC interface. This triggers the MSR-write VM exit and the
control is transfer to the host. The host helps the guest to
set up its next timer by registering the \texttt{hrtimer}
object of guest next timer event. Second, the timer interrupt
is not meant for the guest. It induces the VM exit and
transfer the control back to the host. The host processes the
timer interrupt but does not inject the virtual timer
interrupt. Nonetheless, if the timer interrupt is not meant
for the guest, the guest should not pay the price.

The first task is to transform the local timer interrupt into
the posted interrupt, which is directly delivered to the guest
by the VT-d hardware. It requires two actions. The
timer-interrupt bit of posted-interrupt request needs to be
set, before the posted-interrupt notification is delivered to
the guest core. Since the guest is responsible for its own
timer interrupt, the guest should set the bit in the PIR.
However, such a PIR structure is embedded in the
posted-interrupt descriptor and protected by the host. The
host needs to share the PIR with the guest by isolating the
entire PID to a shared page. If the guest messes up setting
the proper bits in the PID through the EPT, it does not affect
the host normal operations. In our design, the shared PID page
is accessible by three entities: host, guest and
virtualization hardware. The second task is to allow the guest
control the timer initial count register of LAPIC timer. With
the hardware-assisted APIC virtualization, this is achieved by
updating the MSR bitmap of VM control structure. The KVM
intercept of TMICT MSR update is disabled. When the guest
configures the TMICT, the change is written to the register
directly without triggering the VM exit. The third task is to
configure the LAPIC timer chip to deliver the posted-interrupt
notification instead of the actual timer interrupt. In
summary, we reach our goal of guest having dedicated CPUs by
disabling the HLT- and timer-related VM exits.

Using the shared PID has the draw back. It induces the
spurious timer interrupts causing additional interrupt
processing in the guest. Since the guest sets the PIR
timer-interrupt bit before its next timer event, the it
induces the spurious timer interrupts. Such a fake timer
interrupt is induced in two cases. First, the guest
experiences the spurious timer interrupt when performing
I/O-bound activities with the assigned device. Let's take the
assigned network card for an example. Both the bits of timer
interrupt and network-device interrupt are set in the PIR.
Based on the Intel architecture, the timer interrupt has a
higher priority than the network interrupt does. The timer
interrupt is delivered before the network interrupt. Although
the guest should have only processed the network interrupt, it
first processes the timer and then network interrupt. Second,
upon the VM entry, the PIR is synced to the virtual
interrupt-request register because of the KVM implementation.
One of time points to evaluate the virtual interrupt delivery
is at the VM entry time. If the PIR timer-interrupt bit is
present during the copy, the fake virtual timer interrupt is
delivered into the guest after the VM entry. If the arrival of
virtual timer interrupt is earlier than the expected
expiration, the guest ignores it and processes the next
interrupt. Thus, the CPU overhead is reduced in comparison
with the full timer interrupt processing. 

}
% Currently, 3.4 is under the review.
\subsection{Seamless Device State Migration}

%The VM on an \na server is special because it is the only VM on the server and it 
%interacts directly with PCIe devices and local timers.
%We call such a VM a HaaS VM.
Because the states associated with directly accessible devices complicate the migration process~\cite{zhai:2008}, 
\na augments KVM's VM migration capability with additional mechanisms to support seamless migration of HaaS VMs. 
More specifically, \na adopts the following unified strategy to hide the states of directly accessed device states from KVM's VM migration logic:
In the normal mode, a HaaS VM directly accesses PCIe devices and timers; immediately before and during when the HaaS VM is migrated,  the VM 
accesses PCIe devices and timers indirectly; after the migration, the migrated HaaS VM accesses PCIe devices and timers directly again.

Moreover, \na assumes that the source and destination physical servers involved in a migration have an identical
set of hardware devices with which a HaaS VM interacts directly.
This way, the driver code directly interacting with these devices has a chance to properly work on 
both the source and destination server.
In this section, we focus on two types of directly accessed devices: NIC and timer.

\mycomment{
The direct device assignment makes it difficult to migrate the
guest to its destination~\cite{zhai:2008}. After the VM
migration, it is possible that the previously-assigned devices
may not be available at the destination. Even if the assigned
device is available, the internal state of device may not be
readable or still on its way to the destination. The host at
the destination has a hard time to passthrough the device
without the device-specific knowledge. Moreover, some devices
have the unique hardware information that cannot be
transferred, such as the MAC address of network interface
card. In the case of guest-controlled timer, it depends on the
VT-x availability at the destination. Our design takes the
approach of alternating the usage of passthrough and
respective virtual device with the acceptable service downtime
or number of missed time interrupts. We assume that the
devices and hardware supports are available. For the network
activity, the network traffic is switch from the assigned to
virtual network device, before the migration starts. The
network traffic is switched back after the guest starts up at
the destination. For the local timer interrupt, the direct
timer interrupt delivery is switched back to the indirect
delivery with the help of \texttt{hrtimer} object and TMICT
WRMSR VM exit is enabled, before the migration. The changes
are reverted after the guest starts up at the destination.
}

\subsubsection{NIC State Migration}
\figh{nic_bonding}{5}{NIC bonding}

We assume each HaaS physical server is equipped with an SR-IOV Ethernet NIC~\cite{dong:2008},
which provides one physical function and multiple virtual functions, each with its own MAC address.
\na sets up a directly accessed or pass-through NIC using one of the virtual functions
and an indirectly accessed or virtual NIC using another virtual function,
and then teams them up using the active-backup mode of the Linux bonding driver~\cite{bond-dri}, as shown in Figure~\ref{fig:nic_bonding}.
The HaaS VM running on every \na physical server sends and receives network packets through such a bonded interface. 


During the normal run time, the pass-through NIC is the Active slave and the virtual NIC is the Backup slave,
so that a HaaS VM could make full use of the underlying physical NIC's capability.
Before a HaaS VM is to be migrated, \na hot-unplugs the pass-through NIC so as to fail the current Active slave.
Upon detecting the Active slave's failure, the bonding driver immediately switches the Backup slave to be the Active slave,
and from this point on, all network traffic goes through the virtual NIC and KVM has now captured the NIC-related state. Then \na kicks off a VM migration transaction for the HaaS VM to transfer its NIC-related state.
%During the migration, \na . 
After the migration transaction is completed, the HaaS VM is resumed on the destination server and still continues to use the virtual NIC.
In the mean time, \na hot-plugs the pass-through NIC on the destination  server to make it the Backup slave, and then hot-unplugs the virtual NIC so that the current Backup slave or the pass-through  NIC becomes the new Active slave.


\mycomment{
In this section we describe the mechanism of migrating a guest
with direct NIC assignment. As shown in Figure
\ref{fig:nic_bonding} the guest is configured with virtio
network device backed by the vhost driver and a passthrough
network interface card~\cite{zhai:2008}. Using SR-IOV
~\cite{dong:2008} the physical NIC is presented as virtual NIC
through virtual functions. For the purpose of simplicity, we
assume that the guest has one assigned network device. The
prototype overcomes the challenge of migrating NIC assigned
guest by the following strategy. It uses the Ethernet bonding
driver to direct the network traffic between the assigned and
virtual NIC. The migration procedure is divided into two
parts. During regular operation of guest, the assigned NIC is
used for higher network bandwidth. Before the migration, the
host uses the bonding driver and shifts the network traffic
from the assigned NIC to the virtual NIC. The source host
takes the control of the assigned NIC through hot unplug event
of the assigned NIC and starts the migration. After the guest
resumes at the destination, the destination host transfers the
control back to the guest using hot plug event of the assigned
NIC and switches the network traffic from virtual NIC to
assigned NIC.

Linux provides bonding driver to present multiple network
interfaces into a single logical interface. The modes of
bonding driver define the behavior of the bonded interfaces.
To maintain higher network performance during regular
operation of guest, we configure the bonding driver in
active-backup mode where the assigned NIC is chosen as the
active interface and the virtual NIC as a backup-slave
interface. In active-backup mode only one of the interfaces is
active at any time. When the active interface fails, one of
the slave interfaces becomes active. The bonding driver always
takes the MAC of the active interface. On failure of the
active interface, the bonding driver takes the MAC address of
the next to be slave interface. The change in MAC address is
notified by broadcasting ARP packets to avoid the network
packets loss in guest. Before the migration is initiated, on
hot unplug command, the outgoing network traffic is redirected
to the virtual NIC interface by the bonding driver and the
incoming traffic is shifted to the virtual NIC by broadcasting
ARP packets. Once the assigned NIC is hot unplugged, QEMU
issues migrate command. After the migration is completed, once
the guest resumes on the destination, the NIC device is hot
plugged.
}

The most important performance metric for VM migration is the service disruption time.
The additional service disruption time that \na introduces is attributed to the transition from
the pass-through NIC to the virtual NIC on the source server, and the transition from
the virtual NIC to the pass-through NIC on the destination server.
Measurements on a earilier \na prototype suggest that the transition from the virtual to the pass-through NIC introduces non-trivial (about 0.3 second) service disruption, which results from hot-plugging the pass-through NIC on the destination server. 

A deeper analysis shows that the hot-plug operation consists of three steps:
(1) QEMU prepares a software object to represent the pass-through
NIC, (2) then It populates this software object by with parameter values from extracted 
fiom the PCIe configuarion space of the NIC, (3) and finaly it
resets the software NIC object to set up the BAR and interrupt forwarding
information. The first and third step must take place in QEMU's main event
loop when the HaaS VM requsting the hot-plug operation must be paused.

To minimize this service disruption time due to hot-plug, 
\na performs the first and second step of the operation of hot-plugging the pass-through NIC
on the destination server while the HaaS VM is being migrated, and performs the third step after the HaaS VM is migrated and resumed. This brings the service disruption time due to hot-plug to ? second.


\mycomment{
However, by forcing the migrated HaaS VM resumed on the destination server to continue using the virtual NIC,
the time required to hot-plug the directly accessed NIC on the destination server could be fully masked. 
After \na successfully brings up the directly accessed NIC and makes it the Backup slave,  it fails the virtual NIC so as 
to direct all network traffic to the directly accessed NIC.  


We observed that on hot plug event of NIC device on the
destination host, the network service in guest drops until for
0.3seconds. Further, we investigate the reason for the network
packet loss in the guest. The hot plug mechanism of assigned
NIC consists of the following three steps. First, QEMU
prepares a software object that represents the passthrough
NIC. It then realizes the QEMU software object by getting a
copy of configuarion space from the NIC device. Finally, it
resets the software NIC object and setup the BAR and interrupt
forwarding. The first and last step happen in QEMU main event
loop during which the guest remains paused. As a result, the
guest experiences downtime during hot plug operation. In \na,
to mitigate the downtime due to hot plug operation on the
destination host, the first two steps are executed during
migration. During the first phase of pre-copy live
migration~\cite{clark:2005,postcopy-osr}, all the memory pages are
transferred to the destination over the network. The dirty
memory pages are then transferred in multiple iterative
rounds. The VCPU and I/O state of guest are transferred in the
final phase to resume the guest on destination. Step one and
two are executed on the destination host during migration.
Consequently, the network service in guest does not get
affected as it runs on the source host. QEMU allows to setup
and realize the software object during migration. However, the
BAR and interrupts can be setup only after resuming the guest.
Hence, we eliminate the downtime caused during the setup of
software NIC object phase.
}



\subsubsection{Timer State Migration}

To leverage the posted interrupt mechanism to directly deliver local APIC timer interrupts,
\na allows a HaaS VM running on a CPU core to directly access the following device state:
(1) setting the timer interrupt bit in the PIR associated with the CPU core's local APIC ,  (2) 
using a WRMSR to modify the initial counter register of the local APIC timer, and (3) 
computing the next timer expiration time from conversions between clock cycles and nano-seconds, the multiplication
and shift factor of the calibrated timer, etc., which are provided by the hypervisor.

During the normal run time, timer interrupts are delivered to a HaaS VM directly. 
Before a HaaS VM is to be migrated, \na notifies the HaaS VM to stop the 
durect timer interrupt delivery (DTID) mechanism, unmaps the PIR page, 
enables the TMICT WRMSR VM exit, configures the LVT in the local APIC to 
fire timer interrupts as they are rather than as posted-interrupt notification interrupts. 
Upon receving this notification, the HaaS VM uses a different set of 
multiplication and shift parameters to compute the next timer value, 
and convey the resulting value to the hypervisor via the \texttt{hrtimer} object
when control is transferred to the hypervisor upon a TMICT WRMSR VM exit. 


After a HaaS VM is successfully migrated and resumed, both the hypervisor and the VM restart 
the DTID mechanism, by running 
the aforementioned steps in reverse order.


\mycomment{   
The direct timer interrupt delivery depends the following
factors. First, the host shares the posted-interrupt
descriptor with the guest. When the guest udpates the
timer-interrupt bit of posted-interrupt request through EPT,
it does not trigger the EPT violations and the local timer
interrupt is delivered as the posted interrupt. Second, the
guest directly configures the timer initial count register
without a WRMSR VM exit. This is achieved by updating the MSR
bitmap of VM control structure. Third, the guest needs to
correctly compute its next timer events with the
host-calibrated LAPIC timer. To compute the next timer event
from the nano-seconds to the clock cycles, the multiplication
and shift factor of the calibrated timer are required. The
host needs to convey such information to the guest. The guest
configures the TMICT with the correct timer event in clock
cycles.

Before the migration starts, both host and guest need to tear
down the DTID. In the host, it
} 

\mycomment{
It does not only support the direct PCI device
assignment to the userspace processes of VM, but also the
platform devices. Why do we need to allow the userspace
programs to gain the control of physical devices? For the
field of high performance computing, the I/O performances has
a great impact on the overall system performance. The
performance congestion comes, When the rate of data being read
is slower than the rate of data being consumed. Or it happens,
when the rate of data being written is slower than the rate of
data being computed and produced.

The VFIO needs to fulfill the three requirements of device
assignment to a userspace process. First, the userspace driver
can access to the device resources such as I/O ports. Second,
the userspace driver can perform the DMA securely. This is
provided by the IOMMU protection mechanism. Third, the device
interrupts is delivered to the device owner in the userspace.
The way how the VFIO fulfills the three requirements and
applies them with QEMU is briefly described below.
}

\mycomment{
The hardware-assisted direct device assignment helps the guest
achieve the baremetal I/O performance without the additional
virtualization overhead using the VT-x and VT-d support.
After proper VMCS and EPT configuration, the guest gains
the control of assigned device
by MMIO/PIO without the help of hypervisor. The VT-x APIC
virtualization permits the guest to write to the
end-of-interrupt register without a VM exit. With the VT-d
support, the guest performs DMA with the enhanced security and
eliminates the VM exit overhead due to the device interrupts by using 
the posted-interrupt mechanism. In addition to the hardware
support, VFIO provides the software framework for the
userspace device drivers. It works with VT-d and QEMU and sets
up the direct PCI device assignment. Although it is expected
to move the hypervisor out of the guest I/O path, the
hypervisor still induces high CPU utilization due to the
HLT emulation. This greatly deviates our goal of guest having
its own dedicated cores. Nonetheless, our CPU optimization
strategies remedy such a problem.
}


%\subsection{Implementation-specific Details}
% Low-level implementation details include the hypervisor
% modifications.
% - Disable the HLT existing: KVM
% - DTID: KVM and timer-interrupt handler
% - Seamless VM migration: QEMU

{\bf Code Changes:}
The SVMV hypervisor takes up 80-120 MB of memory of the physical machine it runs on when it is idle.
DTID requires modifications to the guest OS, and entails 387 lines of code changes: 43 lines are added to the timer subsystem, 110 lines adjusts the clock multiplication and shift factor, and 234 lines map/unmap/test the shared PID page and cleanup.

\mycomment{
%MOSTLY REDUNDANT
In addition to the existing supports from the VT-x and VT-d,
Linux and its modules, QEMU and KVM, we modify the kernel, KVM
and QEMU and reach our goal of baremetal virtual machine. The
following features are provided. First, the guest has its own
dedicated CPU resources and PCIe devices. The guest's cores
are isolated, so no other host processes or threads compete
the CPU resources with the guest. Each virtual processor is
pinned to the isolated core in a one-to-one fashion. It is
important to move the host out of the way, when the guest is
idle. The VM exit due to the HTL instruction is disabled.
Second, the interrupts from the assigned devices and LAPIC
timer are handled directly by the guest without the hypervisor
intervention. Our approach utilizes the posted-interrupt
mechanism to deliver the local timer interrupt. It has the
requirement that the guest needs to set the PIR
timer-interrupt bit before the arrival of posted-interrupt
notification. While the host protects the posted-interrupt
descriptor, it shares the only the PID from the associated
VMCS. When the DTID is enabled in the host and guest, it
introduces spurious timer interrupts. The guest ignores the
fake timer interrupt, when the timer interrupt arrives earlier
than the expected. Third, the guest updates the LAPIC TMICT
directly. This is achieved by updating the MSR bitmap of
running guest. As a result, when the guest writes its next
period to the LAPIC TMICT, it does not trigger a VM exit and
avoids the overhead of interrupt processing and complexity of
\texttt{hrtimer} subsystem in the host. Fourth, when switching
the traffic from the passthrough to the virtual network
device, the network service down time is reduced by the
Ethernet bonding driver. Furthermore, the network service down
time is eliminated when hot plugging the assigned network
device to the running guest.



\subsection{Disable HLT Exiting and Update the MSR Bitmap}
To disable HLT-related VM exit, the HLT-exiting bit of
processor-based VM-execution control of VMCS is cleared. 
Our
implementation uses the existing KVM function to set or unset
the HLT-exiting bit. The function is \texttt{vmcs\_write64}.
To update the MSR bitmap, we disable the VM exit due to TMICT
WRMSR. The function is
\texttt{vmx\_disable\_intercept\_msr\_x2apic}. We uses this
function as the prototype and enable the TMICT WRMSR VM exits,
while consulting the MSR bitmap from the Intel Software
Developer Manual.

\subsection{CPU Idleness Processing}

When a CPU on a Linux-based machine does not have any applications to run,
Linux schedules a special idle task to run on the CPU. On X86 CPU, this idle task consists of a loop of {\tt HLT} instructions, 
each of which places the CPU in the C1 energy-saving mode until an external interrupt occurs.
Because {\tt HLT} is a privileged instruction, its execution inside a guest VM triggers a VM exit and causes the control to be transferred to KVM. 
Currently, KVM emulates a {\tt HLT} instruction using a busy waiting loop, which polls the CPU until the associated external interrupt comes along. 
%One of our design goals is to let the guest stay on its CPU as
%long as it can. We encounter two different types of
%virtualization overheads. First, the idle guest issues the
%privileged HLT instruction. Such an instruction induces the VM
%exit and transfers the control to the KVM which starts to poll
%on the CPU until the event arrival. 

This polling-loop emulation of {\tt HLT} instruction incurs higher CPU utilization than that when {\tt HLT} instruction is executed natively. 
To address this issue, \na prevents VM exit when executing a {\tt HLT} instruction inside a HaaS VM by modifying the VM-execution control fields 
in the VM's virtual machine control structure (VMCS).
As a result, whenever a CPU becomes idle, no {\tt HLT}-related VM exits occur, the CPU lowers its clock frequency, and the total energy consumption is reduced. 
%It is eliminated
%by updating the primary processor-based VM execution control
%and disabling the VM exit due to the HLT instruction. It
%allows the idle guest to stay on its CPU without polling
%and results in the CPU clock frequency remains at minimal.
%Thus, disabling HLT-induced VM exit helps to reduce the CPU
%power consumption of idle guest. 
%Second, the local timer
%interrupt fires and causes the VM exit, when the guest's time
%quantum is expired. The longer the guest stays on its CPU, the
%higher number of physical timer interrupt it receives. 

%To support our objective, our work utilizes the posted-interrupt
%mechanism and directly deliver the interrupt into the guest
%without triggering any VM exit. This feature is discussed in
%Interrupts that wake a CPU up from the {\tt HLT}-induced energy-saving mode used to go through the hypervisor and thus may incur additional VM exits.
%As disucussed in the~\nameref{subsubsec:shared_pid_dtid}, we leverage X86's {\em posted interrupt} mechanism to deliver these interrupts directly to a VM in the energy-saving mode.
%Consequently, processing of CPU idleness incurs no VM exit, as is the case when the user OS runs directly on a bare metal server.
}

{\bf Gues OS access to the PIR bitmap:}
The Posted-Interrupt Descriptor (PID) contains the PIR bitmap 
which the guest OS must access for direct timer interrupt delivery in \sna.
A CPU core's PID is accessible to the IOMMU (Interrupt Remapping Table), 
the hypervisor, and (in \sna) to  the HaaS VM running on the CPU core. 
%Originally, the PID  is embedded within the HaaS VM's VMCS structure.
To make a PID accessible to the HaaS VM's guest OS,
when a VCPU is created, \na allocates a separate page for its PID, 
and places the PID's base address in the VCPU's VMCS, and 
in all the Interrupt Remapping Table entries that target at the vCPU.
%To enable the VCPU's associated VM to access this PID page, 
The guest OS provides the address of a page in the guest physical address 
space to the hypervisor via a hypercall, and the hypervisor maps this address 
to the host physical address of the PID page in QEMU's page table and 
the Extended Page Table (EPT)~\cite{ept-wiki}, and updates the 
PID page's reference count accordingly.
   
\mycomment{
and extended page table entry to the physical location of
shared page and the reference count of shared page. To have
the DTID reversible, the implementation saves the host
physical address of target GPA. When the DTID is torn down,
our implementation reverts the PTE and EPTE back to the saved
HPA and updates the reference counts of shared page. Second,
the TMICT WRMSR VM exit is disabled, so the guest is able to
update its TMICT without the additional cost. Third, the host
needs to inform the gust the multiplication and shift factor
of calibrated LAPIC timer. When the guest programs the LAPIC
TMICT, it needs the right factor to convert the duration in
time to the number of clock cycles. Fourth, we implement the
screening algorithm in the guest timer interrupt handlers,
\texttt{smp\_apic\_timer\_interrupt} and
\texttt{local\_apic\_timer\_interrupt}. If the timer interrupt
arrives than the expected expiration, it is the spurious
interrupt. Then, the guest ignores it by skipping the regular
processing of timer interrupt. Fifth, the guest updates the
PIR timer-interrupt bit, whenever it receives the timer
interrupt. 
}

%According to Intel IA-32 Intel Architecture Software Developer's Manual~\cite{sdm:2018} 
Intel IA-32 Intel Architecture~\cite{sdm:2018}  requires that all accesses to the PIR bitmap in a PID
must use the locked read-modify-write instruction to ensure mutual exclusion. 
In \sna, when network packets come in at a high rate, a HaaS VM may repeatedly turn 
on the timer interrupt bit in the PIR bitmap to ignore  spurious timer interrupts that 
accompany the NIC interrupts. However, these accesses to the PIR, being based on the
atomic test-and-set instruction, may prevent the Interrupt Remapping Table from 
accessing the PIR on behalf of incoming packets, and cause some of these 
packets to be dropped, eventually degrading the network performance.
 
\na solves this performance problem by requiring a HaaS VM to set the timer interrupt bit
in the PIR using a non-atomic instruction, which immediately cuts down the extent of lock contention
and boosts the network performance. This lockless design is safe because the PIR bits 
that the HaaS VM and the Interrupt Remapping Table access are guaranteed to be 
disjoint and thus do not need to be locked before being accessed.


{\bf NIC Bonding}
Linux's bonding driver provides a \texttt{fail\_over\_mac} option to change 
the MAC address of the bonded interface and broadcast ARP packets for the new MAC address 
when the Active slave fails. \na configures the bonding driver by setting   
\texttt{fail\_over\_mac} to 1, and cuts down the
network downtime due to hot unplug operations.

To transition a NIC between the VFIO and Vhost mode, \na hot-plugs and hot-unplugs the corresponding devcices at the right times.  It uses the QEMU command \texttt{device\_add} to implement a hot plug operation, and the QEMU command \texttt{device\_del} to implement a hot unplug operation. 



\mycomment{
\texttt{device\_add} transfers the control of NIC device from host to guest and.
on \texttt{device\_del} command the NIC device control is transferred back to host. 
In \na, the \texttt{device\_del} command itself is further broken down 
into three commands \texttt{setup\_nic}, which sets up a software NIC object, 
\texttt{realize\_vfio\_nic}, which configures a software NIC object, and \texttt{reset\_nic\_device}, which resets a software NIC object by setting the 
BARs and redirecting the associated interrrupts. 
The \texttt{setup\_nic} and \texttt{reset\_nic\_device} commands must be executed when the VM issuing them is paused. The \texttt{reset\_nic\_device} command is issued on VM resumption. 
}





%\section{Performance Evaluation}
% Performance evaluation includes the following items.
% - Evaluation testbed and methodology
% - Network bandwidth and latency for the direct NIC assignment
%   - Our CPU optimization
%   - vhost
%   - vfio
%   - NIC bonding of vhost and VFIO
% - DID Efficiency
%   - Delivery effectiveness of NIC interrupts
%   - Delivery effectiveness of local timer interrupts
%   - Time the handling of spurious timer interrupt
% - Service disruption time for the baremetal guest migration
%   - Network downtime
%   - The required time to set up or tear down the VFIO and DTID

\subsection{Evaluation testbed and methodology}

\subsection{Direct assignment of 1Gbps and 10Gbps NIC}

\subsubsection{Network Bandwidth, Latency and CPU Utilization}

\subsection{NIC bonding}

\subsubsection{Hot-plug and Hot-unplug}

\subsubsection{Migration of guest}

%% Suggested subsections
%\subsection{Evaluation Testbed and Methodology}
%% Describe the tested including the following items.
% - Hardware configuration: CPU, memory and network cards.
% - Guest: CPU, memory, bonding driver, assigned and virtual devices.
% - QEMU
% - KVM
%\figw{cpu_state_diagram}{8}{CPU State Diagram}

The experiments are run on machines equipped with the 10-core
Intel Xeon CPU E4 v4 of 2.2GHz, 32GB memory, 40Gbps Mellanox
ConnectX-3 Pro network interface and Intel Corporation
Ethernet Connection I217-LM. The Linux kernel 4.10.1 and QEMU
2.9.0 are installed in the host. The guest is configured with
1 to 9 vCPUs, 10GB of RAM, 1 Virtio and 1 pass-through network
device. The Linux kernel of 4.10.1 and the Ethernet bonding
driver are installed in the guest. The bonding driver operates
in active-backup mode.

The tools to measure the CPU, memory and network I/O
performance are listed as follows. iPerf 2.0.5~\cite{iperf}
measures the network bandwidth. Ping~\cite{ping} measures the
round-trip delay. Atopsar 2.3.0~\cite{atopsar} measures the
CPU utilization. Free 3.3.10~\cite{free} measures the memory
consumption. Perf 4.10.1~\cite{perf} measures the number of VM
exits. Cyclictest 0.93~\cite{cyclictest} benchmarks the timer
interrupt latency.
%Kernbench 0.42~\cite{kernbench} benchmarks the CPU throughput.
The following configurations are evaluated:
%depending on the physical or virtual network device, CPU
%optimization and DTID.
\mycomment{
\begin{enumerate}[(a)]
 \item The guest uses the Virtio network device backed by the
  vHost driver (Guest + vHost).
  \item The guest uses the assigned network device (Guest +
  VFIO).
  \item The guest uses the assigned network device. We also
  apply the CPU optimization (OPTI Guest). There are no VM
  exits due to the network interrupt and HLT instruction.
  \item The guest uses the assigned network device. We apply
  both the CPU optimization and DTID (DTID Guest). guest.
  There are no VM exits due the network interrupts, HLT
  instruction, local timer interrupts, direct timer updates or
  EPT violations when accessing the shared PID page.
\end{enumerate}
}

\begin{itemize}
\parskip 0mm
\itemsep 0mm
\item {\bf Bare-metal}: A machine without virtualization.

\item {\bf VHOST}: A HaaS VM accessing I/O devices using the
                   vHost interface.

\item {\bf VFIO}: A HaaS VM accessing I/O devices using the
                  VFIO interface without incurring VM exits
                  due to network interrupts.

\item {\bf OPTI}: A HaaS VM accessing I/O devices using the
                  VFIO interface without incurring VM exits
                  due to network interrupts or HLT
                  instructions.

\item{\bf  DTID}: A HaaS VM accessing I/O devices using the
                  VFIO interface without incurring VM exits
                  due to network interrupts, HLT instructions
                  or local timer interrupts.

\item{\bf  DID}: A HaaS VM accessing I/O devices using the
                 VFIO interface without incurring VM exits due
                 to network interrupts, HLT instructions or
                 local timer interrupts or IPIs.
\end{itemize}


\mycomment{
In Figure~\ref{fig:cpu_state_diagram}, it shows the transition
among host and different guest configurations. The control is
transferred to the host upon a VM exit. After the host has
done it emulation, the control is return back to the guest. In
the case of live migration, OPTI or DTID guest are reverted
back to the unmodified guest before the migration starts.
After the migration ends, the unmodified guest is again
transformed to the OPTI or DTID guest.
}


In our experiment, it is necessary to use two CPU cores to
saturate a 40Gbps Infiniband link for all configurations. One
core is handling the interrupts and soft IRQs, while the other
is running the network performance testing workload. We use a third
core to monitor the CPU utilization, which does not
affect the network performance. In contrast, only one
core is needed to saturate a 1 Gbps Ethernet link.

%
%\subsection{CPU and Performance of Assigned Network Device}
%% Excel: cpu_network_io_performance.xlsx
%% network bandwidth performance: 1 gbps and 40 gbps.
%% CPU utilization when guest performs the network I/O.
%% number of VM exits during the network I/O.
%
%\subsection{Direct Interrupt Delivery Efficiency}
%% Excel: did_efficiency.xlsx
%% number of interrupts from the network card.
%% number of timer interrupts.
%% number of spurious interrupts.
%% time to handle the spurious interrupts.
%% cyclic test and cumulative probability distribution.
%
%\subsection{Seamless Live Migration}
%% Excel: migration_performancy.xlsx
%% migration downtime
%% network downtime
%% time to set up and tear down the VFIO NIC.
%% time to set up and tear down the DTID.

%% Acknowledge our team members, fans and sponsors.

%% USENIX program committees give extra points to submissions
% that are backed by artifacts that are publicly available. If
% you made your code or data available, it's worth mentioning
% this fact in a dedicated section.


%-------------------------------------------------------------------------------
\nocite{*}
\bibliographystyle{plain}
\bibliography{reference}

%%%%%%%%%%%%%%%%%%%%%%%%%%%%%%%%%%%%%%%%%%%%%%%%%%%%%%%%%%%%%%%%%%%%%%%%%%%%%%%%
\end{document}
%%%%%%%%%%%%%%%%%%%%%%%%%%%%%%%%%%%%%%%%%%%%%%%%%%%%%%%%%%%%%%%%%%%%%%%%%%%%%%%%

%%  LocalWords:  endnotes includegraphics fread ptr nobj noindent
%%  LocalWords:  pdflatex acks

%%-------------------------------------------------------------------------------
%\section{Footnotes, Verbatim, and Citations}
%%-------------------------------------------------------------------------------
%
%Footnotes should be places after punctuation characters, without any
%spaces between said characters and footnotes, like so.%
%\footnote{Remember that USENIX format stopped using endnotes and is
%  now using regular footnotes.} And some embedded literal code may
%look as follows.
%
%\begin{verbatim}
%int main(int argc, char *argv[])
%{
%    return 0;
%}
%\end{verbatim}
%
%Now we're going to cite somebody. Watch for the cite tag. Here it
%comes. Arpachi-Dusseau and Arpachi-Dusseau co-authored an excellent OS
%book, which is also really funny~\cite{arpachiDusseau18:osbook}, and
%Waldspurger got into the SIGOPS hall-of-fame due to his seminal paper
%about resource management in the ESX hypervisor~\cite{waldspurger02}.
%
%The tilde character (\~{}) in the tex source means a non-breaking
%space. This way, your reference will always be attached to the word
%that preceded it, instead of going to the next line.
%
%And the 'cite' package sorts your citations by their numerical order
%of the corresponding references at the end of the paper, ridding you
%from the need to notice that, e.g, ``Waldspurger'' appears after
%``Arpachi-Dusseau'' when sorting references
%alphabetically~\cite{waldspurger02,arpachiDusseau18:osbook}.
%
%It'd be nice and thoughtful of you to include a suitable link in each
%and every bibtex entry that you use in your submission, to allow
%reviewers (and other readers) to easily get to the cited work, as is
%done in all entries found in the References section of this document.
%
%Now we're going take a look at Section~\ref{sec:figs}, but not before
%observing that refs to sections and citations and such are colored and
%clickable in the PDF because of the packages we've included.

%%-------------------------------------------------------------------------------
%\section{Floating Figures and Lists}
%\label{sec:figs}
%%-------------------------------------------------------------------------------
%
%%---------------------------
%\begin{figure}
%\begin{center}
%\begin{tikzpicture}
%  \draw[thin,gray!40] (-2,-2) grid (2,2);
%  \draw[<->] (-2,0)--(2,0) node[right]{$x$};
%  \draw[<->] (0,-2)--(0,2) node[above]{$y$};
%  \draw[line width=2pt,blue,-stealth](0,0)--(1,1)
%        node[anchor=south west]{$\boldsymbol{u}$};
%  \draw[line width=2pt,red,-stealth](0,0)--(-1,-1)
%        node[anchor=north east]{$\boldsymbol{-u}$};
%\end{tikzpicture}
%\end{center}
%\caption{\label{fig:vectors} Text size inside figure should be as big as
%  caption's text. Text size inside figure should be as big as
%  caption's text. Text size inside figure should be as big as
%  caption's text. Text size inside figure should be as big as
%  caption's text. Text size inside figure should be as big as
%  caption's text. }
%\end{figure}
%%% %---------------------------
%
%
%Here's a typical reference to a floating figure:
%Figure~\ref{fig:vectors}. Floats should usually be placed where latex
%wants then. Figure\ref{fig:vectors} is centered, and has a caption
%that instructs you to make sure that the size of the text within the
%figures that you use is as big as (or bigger than) the size of the
%text in the caption of the figures. Please do. Really.
%
%In our case, we've explicitly drawn the figure inlined in latex, to
%allow this tex file to cleanly compile. But usually, your figures will
%reside in some file.pdf, and you'd include them in your document
%with, say, \textbackslash{}includegraphics.
%
%Lists are sometimes quite handy. If you want to itemize things, feel
%free:
%
%\begin{description}
%
%\item[fread] a function that reads from a \texttt{stream} into the
%  array \texttt{ptr} at most \texttt{nobj} objects of size
%  \texttt{size}, returning returns the number of objects read.
%
%\item[Fred] a person's name, e.g., there once was a dude named Fred
%  who separated usenix.sty from this file to allow for easy
%  inclusion.
%\end{description}
%
%\noindent
%The noindent at the start of this paragraph in its tex version makes
%it clear that it's a continuation of the preceding paragraph, as
%opposed to a new paragraph in its own right.

%\subsection{LaTeX-ing Your TeX File}
%%-----------------------------------
%
%People often use \texttt{pdflatex} these days for creating pdf-s from
%tex files via the shell. And \texttt{bibtex}, of course. Works for us.
