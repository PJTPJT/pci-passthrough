% Briefly talk about the implementation.
In addition to the existing supports from the VT-x and VT-d,
Linux and its modules, QEMU and KVM, we modify the kernel, KVM
and QEMU and reach our goal of baremetal virtual machine. The
following features are provided. First, the guest has its own
dedicated CPU resources and PCIe devices. The guest's cores
are isolated, so no other host processes or threads compete
the CPU resources with the guest. Each virtual processor is
pinned to the isolated core in a one-to-one fashion. It is
important to move the host out of the way, when the guest is
idle. The VM exit due to the HTL instruction is disabled.
Second, the interrupts from the assigned devices and LAPIC
timer are handled directly by the guest without the hypervisor
intervention. Our approach utilizes the posted-interrupt
mechanism to deliver the local timer interrupt. It has the
requirement that the guest needs to set the PIR
timer-interrupt bit before the arrival of posted-interrupt
notification. While the host protects the posted-interrupt
descriptor, it shares the only the PID from the associated
VMCS. When the DTID is enabled in the host and guest, it
introduces spurious timer interrupts. The guest ignores the
fake timer interrupt, when the timer interrupt arrives earlier
than the expected. Third, the guest updates the LAPIC TMICT
directly. This is achieved by updating the MSR bitmap of
running guest. As a result, when the guest writes its next
period to the LAPIC TMICT, it does not trigger a VM exit and
avoids the overhead of interrupt processing and complexity of
\texttt{hrtimer} subsystem in the host. Fourth, when switching
the traffic from the passthrough to the virtual network
device, the network service down time is reduced by the
Ethernet bonding driver. Furthermore, the network service down
time is eliminated when hot plugging the assigned network
device to the running guest.
