% Migration performance.
% - Describe what we have done.
% - Include tables and figures.
% - Evaluate the performance and see if it match our goal.

\figw{seamless_migration}{10}{Place holder for seamless
migration. Top: Guest sends the outgoing traffic. Bottom:
guest receives the incoming traffic.}

\begin{table}[]
\begin{tabular}{|l|l|l|l|}
\hline
& UNPLUG & PLUG & OPTI. PLUG \\ \hline
UDP SENDER & 4.6 & 300 & 0 \\ \hline
UDP RECEIVER & 1.8 & 300 & 0 \\ \hline
\end{tabular}
\caption{Network Downtime}
\label{tab:migration_network_downtime}
\end{table}

In this experiment, we show the network downtime when
switching between the assigned and Virtio network device. We
also demonstrate the average number of missed local timer
interrupt in guest, when we disable the DTID.

In Table~\ref{tab:migration_network_downtime}, when switching
from the assigned to Virtio network device, the network is --
and -- ms for the UDP sender and receiver respectively. When
switching back from the Virtio to assigned network device, the
network is -- and -- ms for the UDP sender and receiver
respectively. After we hide the overhead of assigned NIC
creation during the VM migration, the network downtime is
reduced to -- and -- ms for the UDP sender and receiver
respectively.

When disabling the DTID, the timer interrupts received by the
guest per seconds matches the expected frequency of timer
interrupt received by the unmodified guest. Since KVM delivers
the virtual interrupts with or without DTID, the guest can
still receives its interrupt during the transition. The longer
the transition takes, the later the migration starts. The host
communicates with the guest by the TCP transmission. We expect
most of the transition time is due to the packet transmission
and processing. The average transition time is -- $\mu$s.

% TODO: Analysis of DTID algorithm in the guest shows the
% guest does not miss the next timer interrupt.
