% Migration performance.
% - Describe what we have done.
% - Include tables and figures.
% - Evaluate the performance and see if it match our goal.

\begin{table}[]
\begin{tabular}{|l|l|l|l|}
\hline
& UNPLUG & PLUG & OPTI. PLUG \\ \hline
UDP SENDER & 4.6 & 300 & 0 \\ \hline
UDP RECEIVER & 1.8 & 300 & 0 \\ \hline
\end{tabular}
\caption{Network Downtime}
\label{tab:migration_network_downtime}
\end{table}

\begin{table}[]
\begin{tabular}{|l|l|}
\hline
& LOCAL TIMER INTERRUPT \\ \hline
HOST & 255 \\ \hline
DTID GUEST & 255 \\ \hline
DTID GUEST - DTID &  \\ \hline
\end{tabular}
\caption{Missed Local Timer Interrupt}
\label{tab:migration_missed_loc}
\end{table}

In this experiment, we show the network downtime when
switching between the assigned and Virtio network device. We
also demonstrate the average number of missed local timer
interrupt in guest, when we disable the DTID.

In Table~\ref{tab:migration_network_downtime}, when switching
from the assigned to Virtio network device, the network is --
and -- ms for the UDP sender and receiver respectively. When
switching back from the Virtio to assigned network device, the
network is -- and -- ms for the UDP sender and receiver
respectively. After we hide the overhead of assigned NIC
creation during the VM migration, the network downtime is
reduced to -- and -- ms for the UDP sender and receiver
respectively.

In Table~\ref{tab:migration_missed_loc}, the average number of
local timer interrupts in host and DTID guest are -- and --
respectively. When disabling the DTID in both host and guest,
the average number of timer interrupts in guest is --.
