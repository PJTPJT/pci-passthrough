% Migration performance.
% - Describe what we have done.
% - Include tables and figures.
% - Evaluate the performance and see if it match our goal.

%\figw{seamless_migration_default}{8}{Place holder for seamless
%migration. Top: Guest sends the outgoing traffic. Bottom:
%guest receives the incoming traffic.}
%\figw{seamless_migration_haas}{8}{Place holder for seamless
%migration. Top: Guest sends the outgoing traffic. Bottom:
%guest receives the incoming traffic.}

\figw{seamless_migration_default}{9}{Performance of iPerf TCP benchmark during live migration of HaaS VM without fail-over-mac and hot-plug optimizations.}

\begin{table}[]
\begin{tabular}{|l|l|l|l|}
\hline
 & \begin{tabular}[c]{@{}l@{}}Unplug\\ (ms)\end{tabular} & \begin{tabular}[c]{@{}l@{}}Plug\\ (ms)\end{tabular} & \begin{tabular}[c]{@{}l@{}}Optimized Plug\\ (ms)\end{tabular} \\ \hline
UDP Sender   & 4.6 & 300 & 0 \\ \hline
UDP Receiver & 1.8 & 300 & 0 \\ \hline
\end{tabular}
\caption{Network downtime experienced by iPerf UDP traffic during VM migration due to Unplug, Plug, and Optimized Plug operations of the HaaS VM's VFIO network device.}
\label{tab:downtime}
\end{table}

In this section, we show the network service disruption
due to transition from VFIO (pass-through) NIC to the 
Virtio (para-virtual) NIC 
at the source server and from the Virtio NIC to the VFIO
NIC at the destination server.
%We also demonstrate the average 
%number of missed local timer interrupts in guest, 
%when we disable the DTID.

\figw{seamless_migration_haas}{9}{Performance of iPerf TCP benchmark during live migration of HaaS VM using fail-over-mac and hot-plug optimizations.}

The guest is configured with a bonding driver in 
active-backup mode where the VFIO NIC acts as 
active slave and Virtio NIC as 
backup slave. The bonding driver accepts 
\texttt{fail\_over\_mac} parameter, which
ensures that the
MAC address of the bond is always the same as the active
slave. Before migrtion, \na hot-unplugs the assigned
NIC. The bonding driver detects hot-unplug event as failure
of pass-through NIC and choses
Virtio NIC as the active slave. 
\texttt{fail\_over\_mac} switches the network 
traffic to the Virtio NIC by broadcasting
gratituous ARP packets notifying the change in MAC address of the bond.

In Figure~\ref{fig:seamless_migration_default}, we measure 
the bandwidth achieved by a HaaS VM 
when running the iPerf TCP benchmark over a 1Gbps network;
The VM sends the network traffic at 940Mbps bandwidth 
during regular execution. Before migration, when the network traffic is 
switched from the VFIO NIC to Virtio NIC using hot-unplug of the VFIO device, 
{\em without using the \texttt{fail\_over\_mac} option},
iPerf experiences 100ms network disruption.
As shown in Figure~\ref{fig:seamless_migration_haas},
when the switch from VFIO to Virtio is performed
using \texttt{fail\_over\_mac} option, 
the downtime due to hot unplug reduces to 80ms.

During live migration, the iPerf benchmark continues 
to send traffic over the Virtio network interface
with the network bandwidth of 940Mbps.
The VM experiences another network dowtime of 
0.1 seconds during the last phase of migration 
when the guest is paused on source server
to transfer the CPU and I/O state to destination server 
for guest resumption. This downtime is typical for pre-copy live migration.

Finally, after the VM resumes at the destination, the
destination hypervisor hot-plugs the VFIO network device. 
\na sets the VFIO network interface as the active slave and
switches the network traffic from Virtio network interface
back to the VFIO network interface. 
In Figure~\ref{fig:seamless_migration_default}, by default, 
the hot-plug operation introduces an additional downtime 
of 300 ms.
To reduce the downtime introduced by
the hot-plug event, it is broken down to three steps.
As shown in Figure~\ref{fig:seamless_migration_haas}, 
the step (a) of creating software NIC object 
and step (b) of extracting the values from the 
configuration space are executed before the guest 
is resumed at the destination. The last step (c) 
of resetting the software NIC object to set up the BAR and 
interrupt forwarding is executed after 
the live migration completes. 
Using the above  approach, as shown in Figure~\ref{fig:seamless_migration_haas},
the downtime due to hot-plug operation reduces to zero
since the first two steps are executed in advance 
before the guest resumes.
The total migration time remains unaffected.

In Table~\ref{tab:downtime}, when switching
from the VFIO (pass-through) network device to Virtio network device
at the source machine, the network downtime (in the ``Unplug'' column)
is observed to be 4.6ms for the UDP sender case and 1.8ms for the  UDP receiver case. 
When switching back from the Virtio to  VFIO network device at the destination
machine, the network downtime (in the ``Plug'' column)
is measured to be 300ms for both the UDP sender and receiver cases. 
After we mask the overhead of VFIO NIC
initialization at the destination machine,
the network downtime (in the ``Optimized Plug'' column) reduces to zero for both sender and receiver cases.

% TODO: Analysis of DTID algorithm in the guest shows the
% guest does not miss the next timer interrupt.
%When disabling the DTID, the timer interrupts received by the
%guest per seconds matches the expected frequency of timer
%interrupt received by the unmodified guest. Since KVM delivers
%the virtual interrupts with or without DTID, the guest can
%still receives its interrupt during the transition. The longer
%the transition takes, the later the migration starts. The host
%communicates with the guest by the TCP transmission. We expect
%most of the transition time is due to the packet transmission
%and processing. The average transition time is -- $\mu$s.
