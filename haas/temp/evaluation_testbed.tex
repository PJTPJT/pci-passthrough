% Describe the tested including the following items.
% - Hardware configuration: CPU, memory and network cards.
% - Guest: CPU, memory, bonding driver, assigned and virtual devices.
% - QEMU
% - KVM
%\figw{cpu_state_diagram}{8}{CPU State Diagram}

The experiments are run on machines equipped with the 10-core
Intel Xeon CPU E4 v4 of 2.2GHz, 32GB memory, 40Gbps Mellanox
ConnectX-3 Pro network interface and Intel Corporation
Ethernet Connection I217-LM. The Linux kernel 4.10.1 and QEMU
2.9.0 are installed in the host. The guest is configured with
1 to 9 vCPUs, 10GB of RAM, 1 Virtio and 1 pass-through network
device. The Linux kernel of 4.10.1 and the Ethernet bonding
driver are installed in the guest. The bonding driver operates
in active-backup mode.

The tools to measure the CPU, memory and network I/O
performance are listed as follows. iPerf 2.0.5~\cite{iperf}
measures the network bandwidth. Ping~\cite{ping} measures the
round-trip delay. Atopsar 2.3.0~\cite{atopsar} measures the
CPU utilization. Free 3.3.10~\cite{free} measures the memory
consumption. Perf 4.10.1~\cite{perf} measures the number of VM
exits. Cyclictest 0.93~\cite{cyclictest} benchmarks the timer
interrupt latency. 
%Kernbench 0.42~\cite{kernbench} benchmarks the CPU throughput. 
The following configurations are evaluated:
%depending on the physical or virtual network device, CPU
%optimization and DTID.
\mycomment{
\begin{enumerate}[(a)]
 \item The guest uses the Virtio network device backed by the
  vHost driver (Guest + vHost).
  \item The guest uses the assigned network device (Guest +
  VFIO).
  \item The guest uses the assigned network device. We also
  apply the CPU optimization (OPTI Guest). There are no VM
  exits due to the network interrupt and HLT instruction.
  \item The guest uses the assigned network device. We apply
  both the CPU optimization and DTID (DTID Guest). guest.
  There are no VM exits due the network interrupts, HLT
  instruction, local timer interrupts, direct timer updates or
  EPT violations when accessing the shared PID page.
\end{enumerate}
}

\begin{itemize}
\setlength\itemsep{-0.04in}
  \item {\bf Bare-metal}: A physical machine without virtualization.
  
  \item {\bf VHOST}: A HaaS VM accessing I/O devices using the vHost interface.
  
  \item {\bf VFIO}: A HaaS VM accessing I/O devices using the VFIO interface without incurring VM exits due to network interrupts.
  
  \item {\bf OPTI}: A HaaS VM accessing I/O devices using the VFIO interface without incurring VM exits due to network interrupts or HLT instructions.  
  
  \item{\bf  DTID}: A HaaS VM accessing I/O devices using the VFIO interface without incurring VM exits due to network interrupts, HLT instructions or local timer interrupts.  
 
\end{itemize}


\mycomment{
In Figure~\ref{fig:cpu_state_diagram}, it shows the transition
among host and different guest configurations. The control is
transferred to the host upon a VM exit. After the host has
done it emulation, the control is return back to the guest. In
the case of live migration, OPTI or DTID guest are reverted
back to the unmodified guest before the migration starts.
After the migration ends, the unmodified guest is again
transformed to the OPTI or DTID guest.
}


In our experiment, it is necessary to use two CPU cores to
saturate a 40Gbps Infiniband link for all configurations. One
core is handling the interrupts and soft IRQs, while the other
is running the network performance testing workload. We use a third
core to monitor the CPU utilization, which does not
affect the network performance. In contrast, it needs only one
core to saturate a gigabit Ethernet link.
