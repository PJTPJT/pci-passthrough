% Describe the tested including the following items.
% - Hardware configuration: CPU, memory and network cards.
% - Guest: CPU, memory, bonding driver, assigned and virtual devices.
% - QEMU
% - KVM
\figw{cpu_state_diagram}{10}{CPU State Diagram}

The experiments are run on machines equipped with the 10-core
Intel Xeon CPU E4 v4 of 2.2GHz, 32GB memory, 40Gbps Mellanox
ConnectX-3 Pro network interface and Intel Corporation
Ethernet Connection I217-LM. The Linux kernel 4.10.1 and QEMU
2.9.0 are installed in the host. The guest is configured with
1 to 9 vCPUs, 10GB of RAM, 1 Virtio and 1 passthrough network
device. The Linux kernel of 4.10.1 and the Ethernet bonding
driver are installed in the guest. The bonding driver operates
in active-backup mode.

The tools to measure the CPU, memory and network I/O
performance are listed as follows. iPerf 2.0.5~\cite{iperf}
measures the network bandwidth. Ping~\cite{ping} measures the
round-trip delay. Atopsar 2.3.0~\cite{atopsar} measures the
CPU utilization. Free 3.3.10~\cite{free} measures the memory
consumption. Cyclictest 0.93~\cite{cyclictest} benchmarks the
timer interrupt latency. Kernbench 0.42~\cite{kernbench}
benchmarks the CPU throughput.

In our study, we have the following guest configurations
depending on the assigned network device, CPU optimization and
DTID.
\begin{enumerate}[(a)]
  \item The guest uses the Virtio network device backed by the
  vHost driver.
  \item The guest uses the assigned network device.
  \item The guest uses the assigned network device. We also
  apply the CPU optimization. Thus, there are no VM exits due
  to the network interrupt and HLT instruction.
  \item The guest uses the assigned network device. We also
  apply both the CPU optimization and DTID. Thus, there are no
  VM exits due the network interrupts, HLT instruction, local
  timer interrupts, direct timer updates or EPT violations
  when accessing the shared PID page.
  \end{enumerate}
In Figure~\ref{fig:cpu_state_diagram}, it shows the transition
among host and different guest configuration. The control is
transferred to the host upon a VM exit. After the host has
done it emulation, the control is return back to the guest. In
the case of live migration, OPTID or DTID guest are reverted
back to the unmodified guest before the migration starts.
After the migration ends, the unmodified guest is again
transformed to the OPTI or DTID guest.

In our experiment, it is necessary to use two cores to
saturate the 40Gbps infiniband for all configurations. One
core is handling the interrupts and soft IRQs, while the other
is performing the network-intensive workload. There is an
additional core to monitor the CPU utilization which does not
disturb the network activity. In contrast, it needs only one
core to saturate the gigabit link.
