% Describe the design choice of direct device assignment and
% its benefits.
The hardware-assisted direct device assignment helps the guest
achieve the baremetal I/O performance without the additional
virtualization overhead. Using the VT-x and VT-d support
fulfills our design goal. After proper VMCS and EPT
configuration, the guest gain the control of assigned device
by MMIO/PIO without the help of hypervisor. The VT-x APIC
virtualization permits that the guest writes to the
end-of-interrupt register without an VM exit. With the VT-d
support, the guest performs DMA with the enhanced security and
zero the VM-exit overhead due to the device interrupts by the
posted-interrupt mechanism. Under the normal circumstances,
the guest can not handle physical interrupts without the
hypervisor. When a physical interrupt is delivered to the
guest, it induces the following overheads. First, it triggers
the VM exit and transfer the control to the host. Saving and
loading the execution context between the root and non-root
mode waste the CPU cycles. Second, the host needs to examine
and handle the physical interrupt. If the physical interrupt
is meant for the guest, the host needs to deliver it as the
virtual interrupt upon the next VM entry. Otherwise, the host
handles it and schedules the next VM entry. Third, when the
guest's interrupt handler finishes, it writes to the EOI
register. Such write operation may induce the VM exit. Since
the guest is not aware of the distinction between the physical
and virtual interrupt, it signals the completion of interrupt
in the same way. After the guest handles the virtual
interrupt, its EOI update is normally emulated by the host.
Fourth, the host may need to use CPU cache and reduce the time
to handle the physical interrupt. This introduces the CPU
cache pollution.
