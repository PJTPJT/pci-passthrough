% Describe the design choice of direct device assignment and
% its benefits.
The hardware-assisted direct device assignment helps the guest
achieve the baremetal I/O performance without the additional
virtualization overhead. Using the VT-x and VT-d support
fulfills our design goal. After proper VMCS and EPT
configuration, the guest gain the control of assigned device
by MMIO/PIO without the help of hypervisor. The VT-x APIC
virtualization permits that the guest writes to the
end-of-interrupt register without an VM exit. With the VT-d
support, the guest performs DMA with the enhanced security and
zero the VM-exit overhead due to the device interrupts by the
posted-interrupt mechanism. In addition to the hardware
support, VFIO provides the software framework for the
userspace device drivers. It works with VT-d and QEMU and sets
up the direct PCI device assignment. Although it is expected
to move the hypervisor out of the guest I/O path, the
hypervisor still induces the high CPU utilization due to the
HLT emulation. This greatly deviates our goal of guest having
its own dedicated cores. Nonetheless, our CPU optimization
strategies remedy such a problem.
