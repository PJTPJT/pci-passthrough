The direct device assignment makes it difficult to migrate the
guest to its destination~\cite{zhai:2008}. After the VM
migration, it is possible that the previously-assigned devices
may not be available at the destination. Even if the assigned
device is available, the internal state of device may not be
readable or still on its way to the destination. The host at
the destination has a hard time to passthrough the device
without the device-specific knowledge. Moreover, some devices
have the unique hardware information that cannot be
transferred, such as the MAC address of network interface
card. In the case of guest-controlled timer, it depends on the
VT-x availability at the destination. Our design takes the
approach of alternating the usage of passthrough and
respective virtual device with the acceptable service downtime
or number of missed time interrupts. We assume that the
devices and hardware supports are available. For the network
activity, the network traffic is switch from the assigned to
virtual network device, before the migration starts. The
network traffic is switched back after the guest starts up at
the destination. For the local timer interrupt, the direct
timer interrupt delivery is switched back to the indirect
delivery with the help of \texttt{hrtimer} object and TMICT
WRMSR VM exit is enabled, before the migration. The changes
are reverted after the guest starts up at the destination.
