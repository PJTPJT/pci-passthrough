% CPU clock rate and implication to the power saving.
One of our design goals is to let the guest stay on its CPU as
long as it can. We encounter two different types of
virtualization overheads. First, the idles guest issues the
privileged HLT instruction. Such an instruction induces the VM
exit and transfers the control to the KVM which starts to poll
on the CPU until the event arrival. This incurs the high CPU
utilization due to the HLT-related VM exit. It is eliminated
by updating the primary processor-based VM execution control
and disabling the VM exit due to the HLT instruction. It
allows the idle guest to stay on its CPU without the polling
and results in the CPU clock frequency remains at minimal.
Thus, disabling HLT-induced VM exit helps to reduce the CPU
power consumption of idle guest. Second, the local timer
interrupt fires and causes the VM exit, when the guest's time
quantum is expired. The longer the guest stays on its CPU, the
higher number of physical timer interrupt it receives. To
support our objective, our work utilizes the posted-interrupt
mechanism and directly deliver the interrupt into the guest
without triggering any VM exit. This feature is discussed in
the~\nameref{subsubsec:shared_pid_dtid}.
