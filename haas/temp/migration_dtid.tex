% DTID during the migration.
The direct timer interrupt delivery depends the following
factors. First, the host shares the posted-interrupt
descriptor with the guest. When the guest udpates the
timer-interrupt bit of posted-interrupt request through EPT,
it does not trigger the EPT violations and the local timer
interrupt is delivered as the posted interrupt. Second, the
guest directly configures the timer initial count register
without a WRMSR VM exit. This is achieved by updating the MSR
bitmap of VM control structure. Third, the guest needs to
correctly compute its next timer events with the
host-calibrated LAPIC timer. To compute the next timer event
from the nano-seconds to the clock cycles, the multiplication
and shift factor of the calibrated timer are required. The
host needs to convey such information to the guest. The guest
configures the TMICT with the correct timer event in clock
cycles.

Before the migration starts, both host and guest need to tear
down the DTID. In the host, it enables the TMICT WRMSR VM exit
by updating the MSR bitmap of VMCS and configures the LAPIC
time to fire the timer interrupt rather than the
posted-interrupt notification. It needs to notify the guest to
tear down its DTID. Upon receving the notification, the guest
requests the host to unmap the shared PID and restores the
timer multiplication and shift back in order to correctly
compute the timer duration. Such a timer duration is conveyed
to the host \texttt{hrtimer} object when the host handles
TMICT WRMSR VM exit. After the guest starts at the
destination, both the guest and host rebuild the DTID. The
process of rebuilding the DTID is the reverse of
aforementioned steps.
