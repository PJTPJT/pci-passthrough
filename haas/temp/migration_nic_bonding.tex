In this section we describe the mechanism of migrating a guest
with direct NIC assignment. The guest is configured with
virtio network device backed by the vhost driver and a
passthrough network interface card~\cite{zhai:2008}. For the
purpose of simplicity, we assume that the guest has one
assigned network device. The prototype overcomes the challenge
of migrating NIC assigned guest by the following strategy. It
uses the Ethernet bonding driver to direct the network traffic
between the assigned and virtual NIC. The migration procedure
is divided into two parts. During regular operation of guest,
the assigned NIC is used for higher network bandwidth. Before
the migration, the host uses the bonding driver and shifts the
network traffic from the assigned NIC to the virtual NIC. It
hot unplugs the assigned NIC and starts the migration. After
the guest resumes at the destination, the destination host hot
plugs the assigned NIC and switches the network traffic from
virtual NIC to assigned NIC.
%It announces the MAC address of assigned NIC by sending
%the gracious ARP packets to its local network before
%shifting the network traffic back to the
%assigned NIC.

% Insert the Guest's network interface configuration.
%
% Seamless VFIO hotplug at the destination. This is
% about the QEMU modification.
% Explain why hotplugging of NIC at the destination cause the
% network downtime?
% Three general steps:
%   - construct a QEMU software object that represent the
%     passthrough NIC. This is done during migration.
%   - realize the QEMU software object. This is done during
%     the migration.
%   - configure the QEMU software object, when the VM wakes
%     up. This is done when the VM wakes up.


% COMMENT
% - We may need to explain the reason why we choose to use the
%   active-backup mode among other different modes such as
%   round robin, transmission-load balancing or adaptive load
%   balancing.
% - The active backup mode only handles the outgoing traffic
%   instead of the incoming traffic. For the outgoing traffic,
%   the bonding driver deposits the outgoing packet requests
%   to the target outgoing ring buffer with the respective to
%   the assigned or virtual NIC. For the incoming traffic, it
%   uses the gratuitous ARP to redirect the incoming traffic.
% - We may need to explain the reason why we would like to
%   shift the network traffic before hot unplugging. What if
%   we do not shift the network traffic and hot unplug the
%   assigned NIC. In the other words, how does shifting the
%   traffic helps with the objective of exitless guest?
% - We may need to first highlight what was the problem we
%   have encountered when hot plugging the assigned device at
%   the destination, so the QEMU modification would make
%   sense. In the other words, the long time we spend inside
%   the main event loop on adding a I/O device, the longer it
%   takes to pause the guest. This is the QEMU implementation
%   restriction.Then, we would explain the QEMU modification
%   which helps to resolve the network service downtime during
%   the unmodified hotplugging event. Probably just need to
%   restructure the paragraphs a bit.
% - What would be the benefit to hide the first two steps of
%   hot plug during the migration? Or will it introduce the
%   addition migration downtime at all? If we do the first two
%   steps of configuration, the effects are taken at the
%   destination. It does not really seem to affect the QEMU
%   migrating the memory pages and CPU context from the source
%   to the destination.

Linux provides bonding driver to present multiple network
interfaces into a single logical interface. In \na the virtual
NIC and the assigned NIC are enslaved in active-backup mode
where only one of the interfaces is active at any time. When
any of the active interface fails, one of the slave interfaces
becomes active. Before the migration is initiated, the network
traffic is gradually shifted to the slave interface using ARP
packets before hot unplugging the assigned NIC. Once the
assigned NIC is hot unplugged, QEMU issues migrate command.
After the migration is completed, once the guest resumes on
the destination, the NIC device is hot plugged.

The hot plug mechanism of assigned NIC consists of the
following three steps. First, QEMU prepares a software object
that represents the passthrough NIC. It then realizes the QEMU
software object by getting a copy of configuarion space from
the NIC device. Finally, it resets the software NIC object and
setup the BAR and interrupt forwarding. The first and last
steps happen in QEMU main event loop during which the guest
remains paused. As a result, the guest experiences downtime
during hot plug operation. In \na, to mitigate the downtime
due to hot plug operation on the destination host, the first
two steps are executed during migration. QEMU allows to setup
and realize the software object during migration. However, the
BAR and interrupts can be setup only after resuming the guest.
Hence, we eliminate the downtime caused during the setup of
software NIC object phase.
