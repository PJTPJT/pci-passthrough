To prove the concept, our design focuses on the case that
guest has a virtio network device backed by the vhost driver
and a passthrough network interface card~\cite{zhai:2008}. For
the purpose of simplicity, the guest NIC is the only assigned
device. The prototype overcomes the challenge by the following
two strategies. First, it uses the Linux bonding driver to
direct the network traffic between the assigned and virtual
NIC. The migration procedure is divided into two parts. Before
the migration, the host uses the bonding driver and shifts the
network traffic from the assigned NIC to the virtual NIC. It
hot unplug the assigned NIC and starts the migration. After
the guest resumes at the destination, the destination host hot
plugs the assigned NIC. It announces the MAC address of
assigned NIC by sending the gracious ARP packets to its local
network before shifting the network traffic back to the
assigned NIC.

% Insert the Guest's network interface configuration.
%
% Seamless VFIO hotplug at the destination. This is
% about the QEMU modification.
% Explain why hotplugging of NIC at the destination cause the
% network downtime?
% Three general steps:
%   - construct a QEMU software object that represent the
%     passthrough NIC. This is done during migration.
%   - realize the QEMU software object. This is done during
%     the migration.
%   - configure the QEMU software object, when the VM wakes
%     up. This is done when the VM wakes up.
