% DID efficiency.
% - Describe what we have done.
% - Include tables and figures.
% - Evaluate the performance and see if it match our goal.

\figw{cyclictest}{10}{Place holder for cyclictest performance}

\begin{table}[]
\begin{tabular}{|l|l|l|l|l|l|}
\hline
& IB & CORE & NIC & SPU & LOC \\ \hline
\multirow{2}{*}{HOST} & \multirow{2}{*}{37.39} & 0 & 118592 & 0 & 256 \\ \cline{3-6} 
&  & 1 & 0 & 0 & 348 \\ \hline
\multirow{2}{*}{GUEST} & \multirow{2}{*}{37.37} & 0 & 123024 & 0 & 256 \\ \cline{3-6} 
&  & 1 & 0 & 0 & 347 \\ \hline
\multirow{2}{*}{DTID GUEST} & \multirow{2}{*}{37.35} & 0 & 110856 & 100114 & 255 \\ \cline{3-6} 
&  & 1 & 0 & 3933 & 348 \\ \hline
\end{tabular}
\caption{Spurious Timer Interrupt}
\label{tab:spurious_timer_interrupt}
\end{table}

In this experiment, we demonstrate that the guest improve its
timer interrupt latency using our DTID mechanism. Typically,
the host virtualizes the timer and the time interrupt by its
high-resolution timer subsystem and the delivery of virtual
timer interrupts. DTID reduces the overheads by directly
delivering the timer interrupts and allowing the guest to
configure TMICT directly. We measure the latency using the
cylictest in (a) host, (b) guest and (c) DTID guest. For (b)
and (c), we apply our CPU optimization. (c) has DTID enabled
whereas (b) has DTID disabled. We also show the additional
cost to handle the spurious timer interrupts in the DTID
guest.

In Figure~\ref{fig:cyclictest}, we observe that the median of
interrupt latency for the host, guest and DTID are --, -- and
--ns respectively. Using the DTID algorithm, we improve the
timer interrupt latency by --\%.

In Table~\ref{tab:spurious_timer_interrupt}, the DTID guest
matches the baremetal network bandwidth performance over the
infiniband, while handling both the expected timer interrupts
and spurious timer interrupts. The number of network and
spurious timer interrupts are -- and -- per second
respectively. We measure the overhead of handling spurious
interrupt in the kernel. It takes --$\mu$s, while the typical
handling of timer interrupt is --$\mu$s. The data suggests our
algorithm works efficiently to deliver the timer interrupt and
ignore the spurious timer interrupts.
