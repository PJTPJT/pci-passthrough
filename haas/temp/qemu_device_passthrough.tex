% Implementation details about the modified device passthrough
% in QEMU in order to minimize the network downtime during the
% hot-plugging event.

% COMMENT
% - Here we have explained the configuration of bonding driver
%   and QEMU modification. To have a full story, we would want
%   to describe what the hot-plug and unplug operation mean
%   and how they are done.
% - We may want to explain how we determine if it is the right
%   time to hot unplug the assigned NIC that does not cause
%   the network downtime. For example, 0 byte of
%   outgoing/incoming traffic are used as the criteria. What
%   would be the duration for us to determin if is the good
%   time to hot unplug the assigned NIC.
% - In the design, we only mentioned the hot plug/unplug
%   operations, but we have not mention how they are done. It
%   is a good place to describe how the operations are done.
%   How the NIC ownership is transferred between the host and
%   guest in our implementation, so the procedures do not
%   disturb the guest work or induce the network downtime.

For the migration of guest with NIC passthrough device, we setup the 
guest with both virtual NIC and assigned NIC. The two interfaces 
are enslaved to the bonding driver in active-backup mode. The 
active-backup mode provides fault-tolerance by switching the
network traffic from failed active interface to slave interface.
In active-backup mode, the MAC address of bond interface 
is the same as the MAC address of active interface. On failure 
of active interface, the MAC address of bond interface has to 
be changed to the MAC address of the slave interface. On changing
the MAC address of the bond interface, the ARP packets have to be
broadcast through newly setup bond interface to switch-over the
traffic from active to slave interface. The bonding driver 
provides \texttt{fail\_over\_mac} option to change 
the MAC address and broadcast the ARP packets on fail-over of
active interface. In \na the bonding driver is configured 
with \texttt{fail\_over\_mac} set to one. This results in minimum
network downtime during hot unplug operation which is performed
before initiating the migration process.

The NIC device control is transferred from host to guest 
(hot plug) using QEMU command \texttt{device\_add} 
and the control is transferred back (hot unplug) to host 
using \texttt{device\_del}. In \na the hot unplug operation 
or \texttt{device\_del} command is further broken down 
into three commands \texttt{setup\_nic}, 
\texttt{realize\_vfio\_nic} and \texttt{reset\_nic\_device}.
\texttt{setup\_nic} command sets up the software NIC object, 
\texttt{realize\_vfio\_nic} configures the software NIC object
and \texttt{reset\_nic\_device} resets the device by setting the 
BARs and redirecting the interrrupts. The commands \texttt{setup\_nic}
and \texttt{realize\_vfio\_nic} are executed on the destination host
during the migration. The destination host executes 
\texttt{reset\_nic\_device} command on migration completion.    
