% Working on the shared-pid DTID.
% Direct timer-interrupt delivery
%   - periodic vs aperiodic timer interrupt
%   - one-shot vs periodic hardware timer interrupt
%   - guest-level access to the PIR page
%   - Eliminate the hardware lock, when the guest accesses
%     the PIR bit
%   - Spurious timer interrupts
Both the host and guest requires to receive timer interrupts
from the LAPIC or virtual APIC respectively. If the timer
interrupt is delivered to the processor core in the root mode,
the logical processor services the timer interrupt through the
host's timer interrupt handler. Then, the host sets up the
next timer event and execute the previously scheduled work
from the bottom halves. If the timer interrupt is delivered,
while the guest is running, it induces the VM exit. The host
handles the timer interrupt and injects the virtual timer
interrupt to the guest. When the guest receives the virtual
timer interrupt, it services the timer interrupt and set up
the next timer event by updating the LAPIC timer initial count
register through the x2APIC interface. This triggers the
MSR-write VM exit and the control is transfer to the host. The
host helps the guest to set up its next timer by the $hrtimer$
subsystem. If the interrupt is meant for the guest, the guest
should not pay the additional price due to the interrupt
processing by the host.

The task is to transform the timer interrupts into the virtual
interrupt and deliver through the posted-interrupt mechanism.
Since the virtual timer interrupt is injected through the
posted-interrupt mechanism, the corresponding bit in the
posted-interrupt register and outstanding notification bit
need to be set beforehand. When the logical processor receives
the posted interrupt notification, it activates the
posted-interrupt processing and delivers the timer interrupt
as the virtual timer interrupt. The transformation is done by
the two procedures. First, the ON bit and timer-interrupt bit
in the PIR are set by the host. Second, the LAPIC timer on the
guest's dedicated core is configured to deliver the PIN. When
the LAPIC timer is up, it delivers the PIN and activates the
posted-interrupt processing and virtual-interrupt delivery.
