
\begin{abstract}
Bare-metal cloud computing or Hardware as a Service (HaaS) allows customers
to rent remote physical servers so as to install their preferred operating system (OS)
and make the best of the server's raw hardware capabilities. 
Common management functions available on virtualized servers offered 
by cloud platforms, such as workload migration and introspection-based 
application performance management, are difficult to 
duplicate on HaaS servers, because HaaS service providers typically 
do not install any software on these bare-metal servers.
To address this manageability gap of existing HaaS services, 
we propose a special form of virtualization called Single Virtual Machine 
Virtualization (SVMV) that is optimized to run a single VM on a 
physical server and allow the VM's  guest OS to directly interact 
with the I/O devices and timer hardware, as if it runs right on top 
of the physical server. 
This paper describes the design, implementation and evaluation of 
a HaaS management system called the \fullname (\sna) that supports 
SVMV without incurring any I/O performance penalty, and enables 
seamless live migration of VMs with direct I/O device access and 
direct timer access with low migration downtime.
\end{abstract}

%\begin{abstract}
%Bare-metal cloud is an emerging cloud service model in which users
%rent dedicated physical servers from a remote cloud provider
%to execute their applications natively, without any virtualization 
%overheads inherent in multi-tenant clouds.
%However, building a bare-metal cloud service management system is technically challenging 
%due to the constraint that no software agent can be installed on the 
%physical servers being rented out. Such a constraint particularly limits
%the cloud provider's ability to live migrate customers' workloads for 
%system maintenance or monitor application performance to 
%meet service-level agreements.
%The \fullname, or \sna, is a bare metal cloud service management system 
%that eliminates this constraint by installing on each physical server
%a thin single-VM hypervisor that combines the benefits of 
%bare-metal execution and virtualization --
%customer workloads have complete and direct access to all the hardware 
%devices while the cloud operator retains the 
%serviceability and manageability of the physical machine.
%We describe a prototype implementation of \na 
%that provides direct device access to the customer's VM, 
%direct delivery of timer and network interrupts without VM Exits,
%and supports live migration.
%\end{abstract}


%--------Prof. Chiueh'ss abstract
%Bare metal cloud service is an emerging form of cloud service in which users 
%rent physical servers from a cloud operator because they want to make full 
%use of the server’s underlying hardware without paying any virtualization overhead. 
%Building a bare metal cloud service management system is technically challenging 
%because of the constraint that no software agent could be installed on the 
%physical servers to be rent out. Such a constraint is particularly limiting 
%when it comes to the support for migration of physical machine state and 
%performance monitoring for applications running on physical machines. 
%The ITRI HaaS OS or IHO, is a bare metal cloud service management system 
%that removes this constraint by installing on each physical server a 
%single-VM virtualization hypervisor, which affords the user complete and 
%direct access to all the devices on the server, and at the same time 
%significantly enhances each server’s serviceability and manageability.
%
%--------Our abstract -- TODO: merge above -----
%
%%Motivation
%Hardware-as-a-service (HaaS) enables customers to rent physical
%machines on cloud platforms to execute their 
%applications with near bare-metal performance.
%% Problem
%However, traditional hypervisor platforms used to host system virtual machines (VMs) 
%are heavyweight and impose significant overheads 
%in interrupt and I/O processing, making them unsuitable for use in HaaS platforms.
%On the other hand, native execution of OS and applications limits 
%a cloud provider's ability to migrate customer workloads for system maintenance
%and failure recovery.
%% Our contributions
%In this paper, we propose the \fullname (\sna) to reduce
%key virtualization overheads on HaaS platforms.
%\na enables a traditional system VM to run atop a thin hypervisor 
%and achieve near baremetal performance while retaining the 
%administrator's flexibility to live migrate the VM.
%A key feature of \na includes
%direct delivery of timer and network interrupts to the VM without VM Exits, 
%thus eliminating hypervisor-level emulation overheads in interrupt delivery.
%% Implementation summary
%We describe a prototype implementation on the KVM/QEMU hypervisor
%that leverages Intel VT-d hardware support to achieve direct network and timer 
%interrupt delivery with no hypervisor intervention, while
%supporting live VM migration.

