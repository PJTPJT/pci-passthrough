
\begin{abstract}

--------Prof. Chiueh'ss abstract
Bare metal cloud service is an emerging form of cloud service in which users 
rent physical servers from a cloud operator because they want to make full 
use of the server’s underlying hardware without paying any virtualization overhead. 
Building a bare metal cloud service management system is technically challenging 
because of the constraint that no software agent could be installed on the 
physical servers to be rent out. Such a constraint is particularly limiting 
when it comes to the support for migration of physical machine state and 
performance monitoring for applications running on physical machines. 
The ITRI HaaS OS or IHO, is a bare metal cloud service management system 
that removes this constraint by installing on each physical server a 
single-VM virtualization hypervisor, which affords the user complete and 
direct access to all the devices on the server, and at the same time 
significantly enhances each server’s serviceability and manageability.

--------Our abstract -- TODO: merge above -----

%Motivation
Hardware-as-a-service (HaaS) enables customers to rent physical
machines on cloud platforms to execute their 
applications with near bare-metal performance.
% Problem
However, traditional hypervisor platforms used to host system virtual machines (VMs) 
are heavyweight and impose significant overheads 
in interrupt and I/O processing, making them unsuitable for use in HaaS platforms.
On the other hand, native execution of OS and applications limits 
a cloud provider's ability to migrate customer workloads for system maintenance
and failure recovery.
% Our contributions
In this paper, we propose the \fullname (\acro) to reduce
key virtualization overheads on HaaS platforms.
\name enables a traditional system VM to run atop a thin hypervisor 
and achieve near baremetal performance while retaining the 
administrator's flexibility to live migrate the VM.
A key feature of \name includes
direct delivery of timer and network interrupts to the VM without VM Exits, 
thus eliminating hypervisor-level emulation overheads in interrupt delivery.
% Implementation summary
We describe a prototype implementation on the KVM/QEMU hypervisor
that leverages Intel VT-d hardware support to achieve direct network and timer 
interrupt delivery with no hypervisor intervention, while
supporting live VM migration.
\end{abstract}

