% Describe the tested including the following items.

\begin{table*}[t]
\renewcommand{\arraystretch}{1.2}
\centering
\resizebox{\textwidth}{!}{%
\begin{tabular}{|l|c|c|c|c|c|c|}
\hline
                    & \multicolumn{6}{c|}{\textbf{VM Exit Happens When}}        \\ \cline{2-7}
\diagbox{\textbf{Configuration}}{\textbf{Optimization}} & \textbf{Access NIC} & \textbf{NIC Interrupt} & \textbf{HLT Instruction} & \textbf{VMX Preemption Timer} & \textbf{Timer Interrupt} & \textbf{IPI} \\ \hline
%\textbf{Bare-metal}   & --                  & --                     & --                       & --                            & --                       & --           \\ \hline
\textbf{VHOST}        & Yes                 & Yes                    & Yes                      & Yes                           & Yes                      & Yes          \\ \hline
\textbf{VFIO/VANILLA} & No                  & No                     & Yes                      & Yes                           & Yes                      & Yes          \\ \hline
\textbf{OPTI}         & No                  & No                     & No                       & No                            & Yes                      & Yes          \\ \hline
\textbf{DTID}         & No                  & No                     & No                       & No                            & No                       & Yes          \\ \hline
\textbf{DID}          & No                  & No                     & No                       & No                            & No                       & No           \\ \hline
\end{tabular}%
}
\caption{Comparison of features for each virtualization
configuration. We optimize the configurations by disabling
different types of VM exits. For DID, we disable all the VM
exits of listed conditions.}
\label{tab:configuration}
\end{table*}

The experiments are run on machines equipped with the 10-core
Intel Xeon CPU E4 v4 of 2.2GHz, 32GB memory, 40Gbps Mellanox
ConnectX-3 Pro network interface and Intel Corporation
Ethernet Connection I217-LM. The Linux kernel 4.10.1 and QEMU
2.9.0 are installed in the host. The guest is configured with
1 to 9 vCPUs, 10GB of RAM, 1 Virtio and 1 pass-through network
device. The Linux kernel of 4.10.1 and the Ethernet bonding
driver are installed in the guest. The bonding driver operates
in active-backup mode.

The tools to measure the CPU, memory and network I/O
performance are listed as follows. iPerf 2.0.5~\cite{iperf}
measures the network bandwidth. Ping~\cite{ping} measures the
round-trip delay. Atopsar 2.3.0~\cite{atopsar} measures the
CPU utilization. Free 3.3.10~\cite{free} measures the memory
consumption. Perf 4.10.1~\cite{perf} measures the number of VM
exits. Cyclictest 0.93~\cite{cyclictest} benchmarks the timer
interrupt latency. The virtualization configurations in


Table~\ref{tab:configuration} are evaluated along with the
Bare-metal configuration, which is a machine without
virtualization. In contrast to the VHOST configurations, other
configurations uses the passthroughed network card.

We found it necessary to use two CPU cores to saturate a
40Gbps Infiniband link for all configurations except the VHOST
configuration (as opposed to just one core for saturating a 1
Gbps link). One core handled the NIC interrupts, while another
handled the soft IRQs and ran network performance testing
workload. We used the third core to monitor CPU utilization,
which did not affect the network performance. For the VHOST
configuration, we used one addition core for the VHOST worker
thread.


%%%%%%%%%%%%%%%%%%%%%%%%%%%%%%%%%%%%%%%%%%%%%%%%%%%%%%%%%%%%%%%%%%%%%%%%%%%%%%%%%%%%%%%%%%%%%%%%%%%%%%%%%%%%%%%%%%%%%%%%
\mycomment{
In Figure~\ref{fig:cpu_state_diagram}, it shows the transition
among host and different guest configurations. The control is
transferred to the host upon a VM exit. After the host has
done it emulation, the control is return back to the guest. In
the case of live migration, OPTI or DTID guest are reverted
back to the unmodified guest before the migration starts.
After the migration ends, the unmodified guest is again
transformed to the OPTI or DTID guest.

}
\mycomment{
\begin{enumerate}[(a)]
 \item The guest uses the Virtio network device backed by the
  vHost driver (Guest + vHost).
  \item The guest uses the assigned network device (Guest +
  VFIO).
  \item The guest uses the assigned network device. We also
  apply the CPU optimization (OPTI Guest). There are no VM
  exits due to the network interrupt and HLT instruction.
  \item The guest uses the assigned network device. We apply
  both the CPU optimization and DTID (DTID Guest). guest.
  There are no VM exits due the network interrupts, HLT
  instruction, local timer interrupts, direct timer updates or
  EPT violations when accessing the shared PID page.
\end{enumerate}
}

\mycomment{
\begin{itemize}
\parskip 0mm
\itemsep 0mm
\item {\bf Bare-metal}: A machine without virtualization.

\item {\bf VHOST}: A HaaS VM accessing I/O devices using the
                   vHost interface.

\item {\bf VFIO}: A HaaS VM accessing I/O devices using the
                  VFIO interface without incurring VM exits
                  due to network interrupts.

\item {\bf OPTI}: A HaaS VM accessing I/O devices using the
                  VFIO interface without incurring VM exits
                  due to network interrupts, HLT
                  instructions or VMX preemption timers.

\item{\bf  DTID}: A HaaS VM accessing I/O devices using the
                  VFIO interface without incurring VM exits
                  due to network interrupts, HLT instructions,
                  VMX preemption timers or local timer
                  interrupts.

\item{\bf  DID}: A HaaS VM accessing I/O devices using the
                 VFIO interface without incurring VM exits due
                 to network interrupts, HLT instructions, VMX
                 preemption timers or local timer interrupts
                 or IPIs.
\end{itemize}
}
