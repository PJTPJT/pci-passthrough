% CPU and network performance of assigned NIC
% - Describe what we have done.
% - Include tables and figures.
% - Evaluate the performance and see if it match our goal.

% For some reason, putting the figures next to each other
% gives me the compilation error. We temporarily comment it
% out.
%\figw{iperf}{10}{Place holder for iperf performance}
%\figw{cpu_util_iperf}{10}{Place holder for iperf CPU utilization}



\mycomment{
\begin{table}[tbp]
\begin{tabular}{|l|l|l|l|}
\hline
& \begin{tabular}[c]{@{}l@{}}Outbound\\ (Gbps)\end{tabular} & \begin{tabular}[c]{@{}l@{}}Inbound\\ (Gbps)\end{tabular} & \begin{tabular}[c]{@{}l@{}}RTT\\ ($\mu$s)\end{tabular} \\ \hline
\begin{tabular}[c]{@{}l@{}} {\bf Bare-metal}\end{tabular}     & 37.39 & 37.52 & 12\\ \hline
\begin{tabular}[c]{@{}l@{}} {\bf VHOST} \end{tabular} & 37.39 & 37.10 & 18\\ \hline
\begin{tabular}[c]{@{}l@{}} {\bf VFIO}\end{tabular}  & 37.45 & 37.58 & 13\\ \hline
\begin{tabular}[c]{@{}l@{}} {\bf OPTI} \end{tabular}    & 37.37 & 37.52 & 13\\ \hline
\begin{tabular}[c]{@{}l@{}} {\bf DTID} \end{tabular}    & 37.35 & 37.50 & 12\\ \hline
%%\begin{tabular}[c]{@{}l@{}}GUEST\\ + vHOST\end{tabular}                          & $37.39 \pm 0.04$ & $19.02 \pm 0.50$ & $24 \pm 5$\\ \hline
%%\begin{tabular}[c]{@{}l@{}}GUEST\\ + VFIO\end{tabular}                           & $37.45 \pm 0.08$ & $37.58 \pm 0.15$ & $13 \pm 2$\\ \hline
%%\begin{tabular}[c]{@{}l@{}}GUEST\\ + VFIO\\ + OPTIMIZATION\end{tabular}          & $37.37 \pm 0.11$ & $37.52 \pm 0.15$ & $13 \pm 3$\\ \hline
%%\begin{tabular}[c]{@{}l@{}}GUEST\\ + VFIO\\ + OPTIMIZATION\\ + DTID\end{tabular} & $37.35 \pm 0.08$ & $37.50 \pm 0.23$ & $12 \pm 5$\\ \hline
\end{tabular}
\caption{Comparison of network bandwidth and network packet latency over a
40Gbps Infiniband link for the five configurations evaluated.}
\label{tab:network_performance}
\end{table}
}

\begin{table}
\renewcommand{\arraystretch}{1.2}
\small
\begin{center}
\begin{tabular}{|c|c|c|c|} \hline
{\bf Configuration} & {\bf Outbound} & {\bf Inbound}  & {\bf Round-Trip} \\ 
 & {\bf (Gbps)} & {\bf (Gbps)} & {\bf Delay ($\mu$s)} \\ \hline
 {\bf Bare-metal}  & 37.39 & 37.52 & 12\\ \hline
 {\bf VHOST} & 37.39 & 19.02/37.10 & 24/18\\ \hline
{\bf VFIO}  & 37.45 & 37.58 & 13\\ \hline
 {\bf OPTI} & 37.37 & 37.52 & 13\\ \hline
 {\bf DTID} & 37.35 & 37.50 & 12\\ \hline
\end{tabular}
\end{center}
\vspace{-0.1in}
\caption{Comparison of network throughput and round-trip packet delay over a
40Gbps Infiniband link among the five evaluated configurations}
\label{tab:network_performance}
\vspace{-0.1in}
\end{table}


In this experiment, we demonstrate that, with all  the proposed optimizations applied,  a HaaS VM performs essentially the same as a bare-metal server in terms of network throughput, network packet latency and CPU utilization.


%the guest using our
%optimization achieves near bare-metal performance with minimum
%CPU utilization and minimal hypervisor involvement. We use the
%macro benchmark and measure the performance in the following
%three metrics: network bandwidth, latency and CPU utilization.

We use  iPerf~\cite{iperf} to measure the network throughput of
incoming and outgoing TCP traffic on a 40Gbps Infiniband link.
As shown in Table~\ref{tab:network_performance}, for outgoing TCP traffic, 
all five evaluated configurations are able to saturate the Infiniband link and 
have almost identical network throughput. 
%performance for the gigabit link is not shown here. All the
%configurations are able to saturate the 1Gbps NIC and 40 Gbps
%Infiniband for the outgoing TCP throughput. 
For incoming TCP  traffic, all five configurations except the VHOST configuration have roughly the same network throughput; however, the VHOST configuration's network throughput 
is about 52\% of that of the other four configurations when two CPU cores are used, and is comparable to that of the other four configurations only when three CPU cores are used. 



We use ping~\cite{ping} to measure the round-trip delays over a 40Gbps Infiniband link and use them to compare the network packet latency of the evaluated configuration. 
As shown in Table~\ref{tab:network_performance}, when two CPU cores are used, the round-trip delay over an Infiniband link in the 
VHOST configuration is 24$\mu$s, which is about twice as much as that of the other four configurations, between 12-13$\mu$s.
The extra delay arises because, when a HaaS VM runs in the VHOST configuration, VM exists occur whenever it access virtual I/O devices and receives interrupts from these devices.




%For the gigabit link, the RTT for the host
%and guest using VFIO configuration is about 160$\mu$s, whereas
%the RTT of guest using VHOST as the backend driver is
%173$\mu$s. For the 40Gbps Infiniband, the guest using the VFIO
%out performs the guest using the VHOST configuration by 50\%.
%As shown in 
%Figure~\ref{fig:network_latency}, 
%the RTT for the
%host and guest using VFIO configuration is about 13$\mu$s,
%whereas the RTT of guest using VHOST as the backend driver is
%24$\mu$s.

% TODO: Put them together as the sub-figures.
% TODO: May need to redo the figure and make the label clear.
%\figw{network_bandwidth}{9}{Comparison of Network Bandwidth over 40Gbps Link}
%\figw{network_latency}{9}{Comparison of Network Latency over 40Gbps Link}

\mycomment{
\begin{table}[tbp]
\begin{tabular}{|l|l|l|l|}
\hline
& \%User & \% System & \%Guest \\ \hline
\begin{tabular}[c]{@{}l@{}} {\bf Bare-metal}\end{tabular}     & 0.64   & 80.98 & -- \\ \hline
\begin{tabular}[c]{@{}l@{}} {\bf vHost} \end{tabular} & 68.6   & 84.36 & 68.6 \\ \hline
\begin{tabular}[c]{@{}l@{}} {\bf VFIO} \end{tabular}  & 90.48  & 85.08 & 90.48 \\ \hline
\begin{tabular}[c]{@{}l@{}} {\bf OPTI}\end{tabular}    & 199.76 & 0.24  & 199.76 \\ \hline \hline
\begin{tabular}[c]{@{}l@{}}In OPTI Guest\end{tabular} & 0.66   & 81.7  & -- \\ \hline
\begin{tabular}[c]{@{}l@{}}In DTID Guest\end{tabular} & 0.68   & 82.9  & -- \\ \hline
\end{tabular}
\caption{Comparison of CPU Utilization. When the guest
generates the TCP outgoing traffic, we measure the CPU
utilization in the host. We also measure the CPU utilization
in the OPTI and DTID guest. The total CPU utilization is 200\%
for two working cores. \%User is the \%CPU time consumed in
the user mode, \%System is the \%CPU time consumed in the
kernel mode and \%Guest is the \%CPU time consumed by the
guest. \%Idle is not shown.}
\label{tab:cpu_utilization_40gbps}
\end{table}
}




\begin{table}
\renewcommand{\arraystretch}{1.2}
\small
\begin{center}
\begin{tabular}{|c|c|c|c|c|} \hline
{\bf Exit Cause} & {\bf VHOST } & {\bf VFIO} & {\bf OPTI} & {\bf DTID} \\ \hline
 {\bf HLT Inst}    & 4362  & 79765 & 0    & 0    \\ \hline
 {\bf EPT Fault}  & 55071 & 0     & 0    & 0    \\ \hline
{\bf Interrupt}   & 15702 & 219   & 498  & 1    \\ \hline
\end{tabular}
\end{center}
\vspace{-0.1in}
\caption{Comparison of number of VM exits per second when the evaluated 
configurations send out TCP traffic over a
40Gbps Infiniband link using iPerf.}
\label{tab:vm_exit}
\vspace{-0.1in}
\end{table}

Because the number of VM exits significantly impacts the CPU utilization, 
let's first examine the number of VM exits per second for each of the five 
evaluated configurations. 
Table \ref{tab:vm_exit}  lists the number of VM exists per second a HaaS VM experiences 
when it sends out TCP traffic over an Infiniband link in one of the four virtualized 
configurations. 
The VHOST configuration incurs a VM exit whenever a HaaS VM executes the HLT instruction,
accesses a NIC and triggers an EPT fault, or gets interrupted. 

In contrast, a HaaS VM running in the VFIO configuration does not trigger any VM exit when
accessing a NIC or receiving a NIC interrupt.
The only interrupts that cause a VM exit for such a HaaS VM are internal interrupts, such
as timer and IPI interrupts.
This is why the numbers of VM exits per second due to EPT faults and interrupts  for the VFIO 
configuration are significantly lower than those of the VHOST configuration.
On the other hand, the number of VM exits per second due to HLT instructions for the VFIO configuration
is drastically larger than that of the VHOST instruction, because 
the percentage of run time spent inside a HaaS VM running in the VFIO configuration 
is higher than that inside a HaaS VM running in the VHOST configuration.
When a HaaS VM occupies the CPU longer, it is more likely for the VM to be idle and issue HLT instructions.

 Compared with the VFIO configuration, the OPTI configuration further eliminates VM exits due to HLT instructions.
 The only VM exits for a HaaS VM running in the OPTI configuration are caused by local interrupts.
 In our test, because both the hypervisor and the test HaaS VM set a timer 
 resolution of 4 msec or 250Hz and their expiration times
 are not synchronized, about 500 hardware timer interrupts are generated every second.
 These interrupts are all delivered through the hypervisor to the HaaS VM. That's why the OPTI configuration 
 experiences roughly 498 VM exits per second, most of which are attributed to timer interrupts.
 In the VFIO configuration , because of the large number of VM exits due to HLT instructions, a timer interrupt could arrive when the hypervisor is in control of the interrupted CPU, and triggers no additional VM exit.
 This is why the VFIO configuration's number of  VM exits  due to interrupts (219) is substantially lower than 500. 
  
In the DTID configuration, the only interrupt that triggers a VM exit for a HaaS VM is IPI interrupt. 
During our test, because IPIs do not occur frequently,  the resulting interrupt rate becomes very low, about 1 per second. 



\mycomment{
we show the most relevant VM exits
that has the significant impact on the CPU utilization. With
the NIC assignment, the number of HLT exits per second
increases from 4362 to 79765. Both the number of VM exits per
second due to the EPT misconfiguration and external interrupts
goes down. After disabling the HLT exit, we observes the
increases number of VM exits frequency due to the timer
interrupts. This is due to the time-slice expiration from both
the host and guest. In our experiment, the host and guest has
the time-slice of 4ms which triggers 250 timer interrupts per
second. We examine the VM-exit qualification and find 498
numbers of VM exits per second are due to the timer
interrupts. This matches what we have expected with the
sampling error. In contrast, if the HLT exiting is not
disabled, some of the host timer interrupts are hidden by the
time when the host is emulating HLT instruction. Furthermore,
the timer interrupts are directly delivered to the guest
without causing any VM exit.
}
\begin{table}
\renewcommand{\arraystretch}{1.2}
\small
\begin{center}
\begin{tabular}{|c|c|c|c|c|c|} \hline
{\bf } & {\bf Host } & {\bf Host } & {\bf Guest} & {\bf Guest} & {\bf Total}\\ 
{\bf } & {\bf  User } & {\bf System } & {\bf  User} & {\bf System} & \\ \hline
 {\bf Bare-metal}    & 0.64\%   & 80.98\% & NA & NA & 81.62\%\\ \hline
 {\bf VHOST} & 68.32\%   & 83.85\% & 0.23\% & 31.20\% & 115.28\%\\ \hline
{\bf VFIO}   & 83.35\%  & 99.25\% & 0.26\% & 34.92\% & 134.43\% \\ \hline
 {\bf OPTI}  & 199.76\% & 0.24\%  & 0.65\% & 81.62\% & 82.51\% \\ \hline 
\end{tabular}
\end{center}
\vspace{-0.1in}
\caption{Comparison of the total CPU utilization percentage and its breakdown 
when the evaluated configurations send out TCP traffic over a
40Gbps Infiniband link at full speed using iPerf. The total is 200\% because two CPU cores are used.}
\label{tab:cpu_utilization_40gbps}
\vspace{-0.1in}
\end{table}

We measured the CPU utilization percentage by the host OS ({\em Host System}),
by the HaaS VM ({\em Host User}), by the kernel of the HaaS VM ({\em Guest System})
and by the user programs of the HaaS VM ({\em Guest User}).
Because  the Bare-metal configuration is not virtualized, its Host User percentage is due to 
the user-level test program, in this case iPerf, and it does not have Guest System or 
Guest User percentage.
The total CPU utilization is equal to $Guest System + Guest User + Host System$.
As shown in Table~\ref{tab:cpu_utilization_40gbps}, 
the Bare-metal configuration consumes the least CPU resource,  
the OPTI configuration's CPU resource consumption comes very close to that of the Bare-metal configuration,
and both the VHOST and VFIO configuration consume at least 40\% more CPU resource in comparison.
%compared with the Bare-metal configuration, 
%the VHOST and VFIO configuration both have a higher Host System percentage because of additional 
%processing costs associated with VM exits and entries that result from HLT instructions, EPT faults and interrupts.

Even though the VFIO configuration enables direct access to and interrupt delivery for NICs,
surprisingly, the total CPU utilization of the VFIO configuration is actually higher than that of 
the VHOST configuration. The main cause is the large number of VM exits due to HLT instructions in the VFIO configuration, as shown in Table~\ref{tab:vm_exit}. When KVM processes emulates the HLT instruction, it uses a busy waiting loop, and the CPU cycles burned by these busy waiting loops bump up the Host System mode's CPU utilization and eventually the total CPU utilization of the VFIO configuration.


%The VHOST configuration's Guest User percentage is significantly higher than 
%that of  the Bare-metal configuration, because the kernel of the HaaS VM is responsible for {\em virtual} NIC 
%interaction and interrupt processing, and TCP protocol processing.
%The VFIO configuration's Guest User percentage is significantly higher than 
%that of  the VHOST configuration, because the kernel of the HaaS VM is responsible for {\em physical} NIC 
%interaction and interrupt processing, and TCP protocol processing.

In the OPTI configuration, the Host User percentage is close to 0\%, indicating that the hypervisor is almost 
completely out of the picture. Although the Host User percentage is 199.76\%, the sum of the Guest User and
Guest System percentage is only 82.27\%. The gap between the two is due to HLT instructions.
That is, when a HaaS VM becomes idle and issues HLT instructions, these instructions put the VM in a lower power
idle mode. However, to the hypervisor, the HaaS VM remains active because it takes up the CPU.
In the end, the fact that the Guest User and Guest System percentage of the OPTI configuration are almost 
identical to the Host User and Host System percentage of the Bare-metal configuration suggests that \na is successful in enabling a HaaS server
to perform very closely to a bare-metal server.

\mycomment{
Although the outgoing TCP throughput is about 37.4Gbps for the
guest using the vHost as the backend driver or assigned NIC,
the guest using the assigned NIC has higher CPU utilization.
The VFIO configuration consumes $175.56/200$ of total CPU
utilization, while the vHost configuration consumes
$152.96/200$ of total CPU utilization. This is due to the
higher number of HLT instruction issued by the guest using the
assigned NIC. The NIC assignment by VT-d allows the guest to
handle the network interrupts directly without a VM exit. At
the same time, it keeps the guest on its CPU for longer time.
The guest issues the HLT instruction more frequently and waits
for its network I/O, when it is idle. The vCPUs burn CPU
cycles when they poll for sometime before executing the HLT
instruction. The vCPU polling mechanism keeps the CPU busy
leading to higher CPU utilization.

To reduce the CPU utilization in host, we disable the HLT
exits by modifying the VMCS structure in KVM. To avoid other
system processes to compete with guest, the vCPUs are pinned
on isolated CPUs. As shown in
\ref{tab:cpu_utilization_40gbps}, we notice that disabling HLT
exits along with dedicating cores to guest, reduces the CPU
utilization in system mode to $0.24/200$ and increases the CPU
Utilization in guest mode to $199.76/200$. It indicates that
the guest occupies the CPUs for most of the time.
}


\mycomment{
\begin{table}[tbp]
\begin{tabular}{lllll}
\hline
& \begin{tabular}[c]{@{}l@{}}Guest\\ + vHost\end{tabular} & \begin{tabular}[c]{@{}l@{}}Guest\\ + VFIO\end{tabular} & \begin{tabular}[c]{@{}l@{}}OPTI\\ Guest\end{tabular} & \begin{tabular}[c]{@{}l@{}}DTID\\ Guest\end{tabular}\\ \hline
HLT                & 4362  & 79765 & 0    & 0    \\ \hline
EPT Misconfig.     & 55071 & 0     & 0    & 0    \\ \hline
External Interrupt & 15702 & 219   & 498  & 1    \\ \hline
%Preemption Timer   & 406   & 271   & 601  & 0    \\ \hline
%IO Instruction     & 18    & 19    & 18   & 19   \\ \hline
%MSR Read           & 2     & 2     & 2    & 2    \\ \hline
%MSR Write          & 2248  & 3919  & 4495 & 2499 \\ \hline
%Pause Instruction  & 1266  & 0     & 0    & 0    \\ \hline
%Pending Interrupt  & 273   & 0     & 0    & 0    \\ \hline
%Total              & 79438 & 84195 & 5614 & 2521 \\ \hline
\end{tabular}
\caption{Comparison of VM Exits per Second Among Busy Guests.
The VM exits are recorded when the guest generates the TCP
outgoing traffic.}
\label{tab:vm_exit}
\end{table}
}
