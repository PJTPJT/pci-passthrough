% Multi-threaded Computation Performance

\figw{parsec_barchart}{8.5}{Percentage of CPU performance
slowdown compared to the bare-metal performance when running
the PARSEC benchmark programs. \textbf{VANILLA}: unmodified
guest configuration without our optimizations.}

\figw{fpspeed_barchart}{8.5}{Percentage of CPU performance
slowdown for the floating-point computation compared to the
bare-metal performance. The x-axis is the IDs for the
benchmark programs of SPEC CPU 2017~\cite{bucek:2018}.
\textbf{VANILLA}: unmodified guest configuration without our
optimizations. \textbf{603}: explosion modeling. \textbf{607}:
relativity. \textbf{619}: fluid dynamics. \textbf{621}:
weather forecasting. \textbf{627}: atmosphere modeling.
\textbf{628}: wide-scale ocean modeling. \textbf{638}: image
manipulation. \textbf{644}: molecular dynamics. \textbf{649}:
electromagnetic. \textbf{654}: regional ocean modeling.}

\figw{intrate_barchart}{8.5}{Percentage of CPU performance
slowdown for the integer computation compared to the
bare-metal performance. The x-axis is the IDs for the
benchmark programs of SPEC CPU 2017~\cite{bucek:2018}.
\textbf{VANILLA}: unmodified guest configuration without our
optimizations.  \textbf{500}: perl scripts. \textbf{502}: GCC.
\textbf{505}: route planning. \textbf{520}: OMNeT simulation.
\textbf{523}: XML conversion. \textbf{525}: video compression.
\textbf{531}: chess. \textbf{541}: GO. \textbf{548}: sudoku.
\textbf{57}: data compression.}

To evaluate how well a HaaS VM performed for the CPU-bound
applications, we measured the percentage slowdown of CPU,
cache and memory all together by the SPEC CPU
2017~\cite{bucek:2018} and PARSEC~\cite{bienia:2008} benchmark.
In our experiment, the HaaS VM had dedicated 8 cores, and 27GB
RAM, whereas \na had the remaining two core, include physical
CPU core 0 and the remaining 5GB RAM. The CPU L1/L2/L3 cache
topology was exposed to the VM. The guest kernel utilized this
information for the scheduling optimization. The bare-metal
was the baseline and had the same configuration as the VM. We
compared the percent slowdown between the Haas and vanilla VM
which did not use our optimizations except the dedicated
cores. Thus, the vanilla VM experienced much more
virtualization overhead than the HaaS VM.

For the multi-threaded workloads, two popular thread models
were considered: Pthreads~\cite{lewis:1998} and
OpenMP~\cite{dagum:1998}. Pthreads~\cite{lewis:1998} provided
the finer-grained control over the thread management, while
openMP~\cite{dagum:1998} was the industry standard, more
portable and easier to scaled than Pthreads. 
