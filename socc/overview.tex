\section{\na Overview}

\figw{architecture}{8.75}{Architecture of \na for single-VM
virtualization: The \na runs as a thin hypervisor that
provides manageability services such as live migration and VM
monitoring. The HaaS VM runs on all the remaining CPU cores
with direct access to I/O devices, local timers and interrupt
command register and handles the direct delivery of timer,
device and inter-VCPU interrupts.}

Figure~\ref{fig:architecture} shows the high-level
architecture of \na. \na runs as a thin hypervisor on each
physical server and supports a single VM, called the HaaS VM,
on which a users installs preferred OS and applications. \na
primarily provides value-added functionalities, such as live
migration, introspection, and performance monitoring, while
staying out of normal execution path of a HaaS VM.

I/O operations and interrupt processing using traditional VMs
incur higher overheads than bare-metal execution. I/O
operations issued by a VM typically are trapped into the
hypervisor via VM exits for emulation. Likewise, external
device interrupts, local timer interrupts and IPIs to the CPU
running a VM result in VM exits for emulating virtual
interrupt delivery. Each VM exit is expensive, since it
requires saving the VM's execution context upon exit,
emulation of the exit reason in hypervisor mode, and finally
restoration of the VM's context before re-entry into the guest
mode.

The main technical challenge of providing the guest OS with
the illusion of running on a bare-metal server is that its
interactions with I/O devices, such as network interface card
(NIC) and disk controller, must be direct without going
through any intermediary. \na allows the sole VM on the
physical machine to directly interact with PCIe I/O devices,
timer hardware and the interrupt command register by
leveraging Intel VT-d~\cite{intelvtd-paper} and Linux Virtual
Function I/O (VFIO)~\cite{vfio} mechanisms.

To match bare-metal I/O performance, \na implements mechanisms
for reducing the hypervisor's CPU, I/O, and memory footprint
during runtime and minimizing interference with guest
operations. In addition, \na enables direct delivery of device
interrupts, timer interrupts and inter-VCPU IPIs to a HaaS VM
without any VM exit by leveraging the posted interrupt
mechanism in Intel VT-d. Direct interrupt delivery greatly
reduces the interrupt processing latency and is particularly
useful for real-time applications running on bare-metal
servers. When the VM must be live migrated, \na instructs the
VM to use para-virtual I/O, virtualized timer and indirect
IPIs and completes the migration. At the destination, it
switches the VM back to direct I/O device and direct timer
access and direct inter-VCPU IPIs with minimal disruption of
VM's workload.
