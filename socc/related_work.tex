
\section{Related Work}
% Bare-metal service
% - IBM SoftLayer
% - Oracle
% - Zenlayer
% - Vultr
In recent years, virtualization overheads and security concerns on 
multi-tenant IaaS platforms have led to the rise of a 
number of bare-metal or HaaS~\cite{softlayer,oracle,zenlayer,vultr,m2}
clouds which provision dedicated physical machines for customers
to provide native execution and I/O performance.
However, the absence of a virtualization layer 
limits the cloud provider's ability to manage, monitor, 
and live migrate~\cite{clark:2005,postcopy-osr}
customer workloads, leaving the customer responsible for 
these functions.
% Reducing virtualization overheads
On the other hand, virtualized IaaS cloud platforms~\cite{gcp,azure,ec2}
retain the cloud provider's ability to manage and live migrate customers'
VMs but also introduce overheads  due to
hypervisor-level emulation of hardware access by guests. To mitigate I/O-related 
virtualization costs, customers can provision VMs with 
direct device assignment~\cite{intelvtd-paper,intelvtd-manual}. 
However, doing so has meant sacrificing the ability 
to live migrate VMs for load balancing and fault-tolerance
due to the difficulty of migrating the state of physical I/O devices.
\na aims to  support bare-metal server performance
while retaining the benefits of virtualization via a 
thin hypervisor.

%TODO container/process migration cannot migrate the OS state
% and limit the type of processes that can be migrated, such as 
% those without active network connections.
{\bf Live migration with direct device access:} 
On-demand virtualization~\cite{ondemand} presented an approach
to insert a hypervisor that deprivileges the OS at the 
source server just before migration and re-privileges it
at the destination after migration. However, this approach 
requires the hypervisor to fully trust the OS being migrated, 
the OS having full access to the hypervisor's memory and code.
Such lack of isolation rules out the use of many hypervisor-level
cloud services that require strong isolation, such as VM introspection
and SLA enforcement. In contrast, a HaaS VM 
cannot access or modify \sna's code or memory.
Recent developments~\cite{vfio-live-migration,blmvisor-journal} 
have proposed frameworks to support live migration 
of VMs using passthrough I/O devices. These approaches 
require guest device drivers to save and reconstruct device-specific 
I/O states for VM migration; the hypervisor to track changes to
and transfer individual device states during migration.
%BLMVisor~\cite{blmvisor,blmvisor-journal} proposes live migration of VMs 
%having direct access to physical devices by requiring the hypervisor
%to regularly intercept and capture low-level physical device states
%during normal execution; the captured state is then reconstructed
%at the destination after migration. 
Such approaches require
implementing device-specific state capture and reconstruction code 
in the VM by each device vendor besides tracking of each device's
state changes by the hypervisor. 
In contrast to the above approaches, 
\sna's live migration mechanism is device agnostic; 
device drivers in Haas VM do not need
device-specific state capture and reconstruction code 
and the \na does not need to intercept guest I/O state changes.
Finally, unlike the above approaches,
\na also supports live migration of guests accessing non-PCIe 
hardware such as local timer and IPI hardware.


{\bf Reducing virtualization overheads:}
A number of techniques have been proposed to reduce virtualization
overheads in interrupt delivery by eliminating VM Exits to the hypervisor.
ELI~\cite{amit:2015} and DID~\cite{tu:2015} presented techniques
for direct delivery of I/O interrupts to the VM. 
Intel VT-d~\cite{intelvtd-paper,intelvtd-manual} supports a posted interrupt delivery mechanism~\cite{postedinterrupt}.
However, these techniques do not fully eliminate hypervisor's role
in the delivery of device interrupt.
Specifically, the hypervisor must intercept and deliver device interrupts
(using either virtual interrupts or inter-processor interrupts)
when the target VM's VCPU is not scheduled on the CPU
that receives the device interrupt. Further, idling guest VCPUs
waiting for I/O completion will trap to the hypervisor, where a well-intended
halt-polling optimizations can inadvertently end up increasing CPU utilization.
In contrast, \na eliminates these 
overheads in device interrupt delivery by 
combining the use of Intel VT-d with optimizations to
dedicate physical CPU cores to the guest and disable VM exits when a guest VCPU idles.
%Intel VT-d posted interrupts, but also by isolating  hypervisor execution 
%to a single physical core, and ensuring that physical 
%cores assigned to the VM remain in guest (non-root) 
%mode whenever the guest idles while waiting for I/O completion.

Finally, a key distinction of \na lies in local APIC timer
access by the guest, timer interrupt delivery to the guest, and IPI 
delivery between guest VCPUs.
None of the existing techniques eliminate hypervisor overheads 
when a VM interacts with its local APIC timer hardware and nor do they 
support direct delivery of timer interrupts and inter-VCPU IPIs.
\na provides these features for a HaaS VM through a novel 
use of Intel VT-d posted interrupt mechanism and without 
incurring VM Exit overheads.

% TODO: Single VM virtualization
%   Unikernels
%   Clean OS

%TODO: Summarize Jailhouse

%\subsection{Jailhouse}
%By far, Xen and KVM are the two Linux de facto hypervisors.
%The are versatile and cooperate with QEMU to virtualize the
%entire system. They utilize the modern hardware-assisted
%virtualization and match the bare-metal CPU, memory and I/O
%performance with the significantly reduced CPU overhead.
%Nonetheless, there are some areas that these two
%general-purpose solutions need further improvements. One such
%area deals with the real time applications. A real-time
%application needs to meet the minimal response time and/or the
%worst latency. For example, the vehicular break system needs
%to meet the minimal response time , when the driver presses
%the break. Failing the requirement leads to a detrimental
%consequence.
%
%The jailhouse hypervisor is a partitioning hypervisor that
%runs on the bare-metal and works closely with the
%Linux\cite{sinitsyn:2015, ramsauer:2017}. It is not trying to
%be the full-fledged KVM. Its responsibility is to partition
%and assign the available hardware resource to the guests and
%prevent a guest from interfering with the jailhouse or another
%guest. Each guest has its own set of dedicated hardware
%resource and do not share them. In the other words, there is
%no overcommitment of resources. Moreover, the jailhouse is an
%example of asymmetric multiprocessing design, which treats one
%processor core differently from another. For example, one
%processor can access the hard disk, while another accesses the
%serial port. The jailhouse uses this design to create an
%isolated environment, called a cell. When the jailhouse boots
%up, it creates a Linux cell or root cell, containing all the
%processor cores, memory and hardware resources. Before the
%jailhouse boots up a guest, it creates a new cell and allocate
%the requested hardware resources to the guest or inmate. This
%set of hardware resources is dedicated to the guest. This
%partitioning design indicates the jailhouse does not emulate
%the devices or manage resources for the guests.
%
%To pass-through the devices, the jailhouse requires the
%hardware-assisted virtualization. On the x86, it is the VT-x
%and VT-d. For the LAPIC registers, the jailhouse handles the
%accesses differently depending on whether the host supports
%the xAPIC or x2APIC mode. If the host only supports the xAPIC
%mode, the jailhouse traps all the guest's accesses to the
%LAPIC registers. Even if the guest would like to access the
%LAPIC using the MSR interface in the x2APIC mode, the
%jailhouse traps and emulates it on the top of xAPIC mode. If
%the host supports the x2APIC, the jailhouse only traps the
%access to the interrupt command register. ICR is used to send
%IPIs to other process cores. The trap is required, so the
%jailhouse can prevent the malicious guest from disturbing
%other guests. In terms of the interrupt handling, the external
%interrupts are delivered directly to the guest, which handles
%them through the guest IDT. One exception is the non-maskable
%interrupt. The jailhouse uses NMI to regain the controls of
%guest CPUs.
%
%Thus, the interrupts from the assigned devices and timer
%interrupts are delivered directly into the guest without any
%indirection, while the guest can have a fully control over its
%assigned devices and LAPIC timer. These features help to meet
%the real-time application requirement for the latency and
%response time or the long-running computations. Furthermore,
%the jailhouse confines the guest in its own cell and results
%in security enhancement and no resource overcommitment.
%
%% Previous work
%\subsection{Exitless Interrupt}
%Exitless Interrupt (ELI) is based on the following
%conditions~\cite{amit:2015}. First, the guest has its own set
%of dedicated cores. Second, the guest runs the I/O intensive
%workload with the directly assigned SR-IOV devices. Third, the
%number of interrupts that the guest receives from the assigned
%devices is proportional to the guest execution time. Thus, the
%ELI delivers the assigned interrupts to the guest directly and
%non-assigned interrupts to the VMM. This is achieved by the
%shadow interrupt descriptor table.
%
%When the guest runs in the guest mode, it runs with this
%shadow IDT prepared by the ELI instead of its own IDT. When
%the logical processor in the non-root mode receives an
%external interrupt, it screens the interrupt by the shadow
%IDT. If the interrupt is from the assigned device, it
%dereferences the corresponding table entry and invokes the
%guest's ISR. Otherwise, it traps to the host, which handles
%the non-assigned interrupts. To make such a distinction, the
%ELI copies the guest's IDT, including the guest's exceptions
%and assigned devices, to the shadow IDT. The ELI preserves
%the device interrupt priorities and keeps the interrupt vector
%numbers of each device the same between the corresponding
%guest and host interrupt handlers. It marks guest entries as
%present and the rest of entries as non-present. Moreover, the
%ELI configures the logical processor to force exit on the
%non-present exception. When the host handles the non-present
%exception, it needs to inspect the exit reason. If it is due
%to the non-assigned physical interrupt, it converts the
%exception back to the original interrupt vector and invoke the
%respective ISR. For the virtual interrupts from the emulated
%device, the ELI marks them as the non-assigned interrupts.
%After the trap, the host enters the special injection mode
%that configures the logical processor to exit on any physical
%interrupts and the guest to use its own IDT. The host injects
%the virtual interrupt to the guest.
%
%When the guest's ISR finishes handling its interrupt, it
%updates the LAPIC EOI register and triggers the VM exit. The
%VM exit can be disabled through the MSR bitmap, when
%configuring the VMX module with the x2APIC programming
%interface. Since the guest does not distinguish between the
%injected virtual interrupt or the assigned interrupt, it
%updates the EOI LAPIC register for all cases. This should not
%be the case for the virtual interrupt from the host emulated
%device. When the host operates in the special injection mode,
%it traps the EOI write to the host. Once the guest finishes
%all the pending EOI writes for the virtual interrupts, the
%host leaves the special injection mode.
%
%Thus, the ELI delivers the interrupts from the assigned
%devices directly and virtual interrupts indirectly, while
%preserving the interrupt priorities. It effectively reduces
%the VM exits due to the assigned interrupts to 0 and handles
%the EOI signals properly.
%
%\subsection{Direct Interrupt Delivery}
%Direct Interrupt Delivery (DID) solves the two
%challenges~\cite{tu:2015}. First, the interrupts are directly
%delivered to the guest without the hypervisor intervention on
%the delivery path. Second, the guest completes the end of
%interrupts without the VM exit. The two challenges are divided
%into the following sub-tasks. First, if a guest is running,
%the interrupts from the SR-IOV devices, timers and emulated
%devices are delivered directly. Second, when the target guest
%is not running, its interrupts are delivered through the
%hypervisor. Third, the priority of physical and virtual
%interrupts are preserved. Fourth, the number of interrupts
%that the host needs to complete is zero. In addition, the DID
%supports the unmodified guests.
%
%The DID routes the interrupts to the guest or host
%appropriately by configuring the interrupt routing and
%remapping table from the IOAPIC and IOMMU respectively. For
%the SR-IOV device interrupts, the DID disables the VM-exit due
%to external interrupts. Consequently, the interrupts are
%delivered normally through the guest interrupt descriptor
%table. If the virtual processor is running, the device
%interrupt is directly delivered. If the virtual processor is
%rescheduled by the host, the interrupt is delivered to the
%host through the non-maskable interrupt. The host injects the
%corresponding virtual interrupt, when the virtual processor is
%re-scheduled on the logical processor.
%
%On the modern x86 architecture, each processor has its own
%LAPIC. LAPIC generates timer interrupts, which are not routed
%by the IOAPIC and IOMMU. Instead, the LAPIC delivers the timer
%interrupts to its associated processor. After the guest
%handles the timer interrupt, it sets up the next timer event
%by configuring the LAPIC timer. This requires the host's help.
%The DID ensures that the guest's timer interrupt only delivers
%to guest instead of other user-level processes. The DID
%installs the software timer on the host's dedicated core.
%After the host receives the guest's timer interrupt, it
%delivers the physical timer interrupt through the IPI. The
%host delivers the timer IPI, only when the virtual processor
%is active. If the virtual processor is preempted, the host
%delivers the timer IPI, when the virtual processor is
%scheduled on another logical processor.
%
%The DID delivers the virtual interrupts as the IPIs, which are
%treated as the external interrupts. Each QEMU virtual device
%is represented by a thread and runs on its dedicate core.
%Before delivering the virtual device interrupt, the device
%thread need to check if the virtual processor is active. If
%the virtual processor is running, the thread delivers the
%virtual device interrupt through the IPI. Since the DID
%disables the VM-exit due to external interrupts, the guest
%receives the device interrupt directly. If the vCPU is not
%running, the host receives the interrupt on the behalf of
%guest. The host injects it to the appropriate guest as the
%virtual interrupt, when the virtual-processor is scheduled on
%the logical processor.
%
%When the ISR completes, it instructs the logical processor to
%update the LAPIC end-of-interrupt register. In the DID scheme,
%the direct EOI write has the following considerations. First,
%if the handler of virtual interrupts directly updates the EOI
%register, the LAPIC may thinks there is no pending interrupt.
%Second, it may also think the pending interrupt is completed,
%which is still ongoing. Third, LAPIC may dispatch the lower
%priority interrupt to preempt a higher-priority interrupt. The
%root cause is that the virtual interrupts are not visible to
%the hardware LAPIC, when they are injected via IRR and ISR.
%The DID solves this problem by converting the virtual
%interrupts as the IPIs, while disabling the VM-exit due to the
%EOI writes. If there are multiple virtual interrupts, the host
%issues them as the IPIs one at a time.
%
%Thus, the DID delivers the external interrupts from the
%assigned devices, timer and emulated devices directly into the
%guest, while preserving the interrupt priorities. The DID
%zeroes the number of VM exits due to the interrupts. Moreover,
%it is not required to modify the guest kernel. This is one of
%the advantages over ELI and ELVIS.
