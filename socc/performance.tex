\section{Performance Evaluation}
\vspace{-0.05in}

% Describe the tested including the following items.
% - Hardware configuration: CPU, memory and network cards.
% - Guest: CPU, memory, bonding driver, assigned and virtual devices.
% - QEMU
% - KVM
%\figw{cpu_state_diagram}{8}{CPU State Diagram}

The experiments are run on machines equipped with the 10-core
Intel Xeon CPU E4 v4 of 2.2GHz, 32GB memory, 40Gbps Mellanox
ConnectX-3 Pro network interface and Intel Corporation
Ethernet Connection I217-LM. The Linux kernel 4.10.1 and QEMU
2.9.0 are installed in the host. The guest is configured with
1 to 9 vCPUs, 10GB of RAM, 1 Virtio and 1 pass-through network
device. The Linux kernel of 4.10.1 and the Ethernet bonding
driver are installed in the guest. The bonding driver operates
in active-backup mode.

The tools to measure the CPU, memory and network I/O
performance are listed as follows. iPerf 2.0.5~\cite{iperf}
measures the network bandwidth. Ping~\cite{ping} measures the
round-trip delay. Atopsar 2.3.0~\cite{atopsar} measures the
CPU utilization. Free 3.3.10~\cite{free} measures the memory
consumption. Perf 4.10.1~\cite{perf} measures the number of VM
exits. Cyclictest 0.93~\cite{cyclictest} benchmarks the timer
interrupt latency.
%Kernbench 0.42~\cite{kernbench} benchmarks the CPU throughput.
The following configurations are evaluated:
%depending on the physical or virtual network device, CPU
%optimization and DTID.
\mycomment{
\begin{enumerate}[(a)]
 \item The guest uses the Virtio network device backed by the
  vHost driver (Guest + vHost).
  \item The guest uses the assigned network device (Guest +
  VFIO).
  \item The guest uses the assigned network device. We also
  apply the CPU optimization (OPTI Guest). There are no VM
  exits due to the network interrupt and HLT instruction.
  \item The guest uses the assigned network device. We apply
  both the CPU optimization and DTID (DTID Guest). guest.
  There are no VM exits due the network interrupts, HLT
  instruction, local timer interrupts, direct timer updates or
  EPT violations when accessing the shared PID page.
\end{enumerate}
}

\begin{itemize}
\parskip 0mm
\itemsep 0mm
\item {\bf Bare-metal}: A machine without virtualization.

\item {\bf VHOST}: A HaaS VM accessing I/O devices using the
                   vHost interface.

\item {\bf VFIO}: A HaaS VM accessing I/O devices using the
                  VFIO interface without incurring VM exits
                  due to network interrupts.

\item {\bf OPTI}: A HaaS VM accessing I/O devices using the
                  VFIO interface without incurring VM exits
                  due to network interrupts or HLT
                  instructions.

\item{\bf  DTID}: A HaaS VM accessing I/O devices using the
                  VFIO interface without incurring VM exits
                  due to network interrupts, HLT instructions
                  or local timer interrupts.

\item{\bf  DID}: A HaaS VM accessing I/O devices using the
                 VFIO interface without incurring VM exits due
                 to network interrupts, HLT instructions or
                 local timer interrupts or IPIs.
\end{itemize}


\mycomment{
In Figure~\ref{fig:cpu_state_diagram}, it shows the transition
among host and different guest configurations. The control is
transferred to the host upon a VM exit. After the host has
done it emulation, the control is return back to the guest. In
the case of live migration, OPTI or DTID guest are reverted
back to the unmodified guest before the migration starts.
After the migration ends, the unmodified guest is again
transformed to the OPTI or DTID guest.
}


In our experiment, it is necessary to use two CPU cores to
saturate a 40Gbps Infiniband link for all configurations. One
core is handling the interrupts and soft IRQs, while the other
is running the network performance testing workload. We use a third
core to monitor the CPU utilization, which does not
affect the network performance. In contrast, it needs only one
core to saturate a gigabit Ethernet link.


\vspace{-0.1in}
\subsection{Network I/O Performance}
\vspace{-0.05in}
% CPU and network performance of assigned NIC
% - Describe what we have done.
% - Include tables and figures.
% - Evaluate the performance and see if it match our goal.

% For some reason, putting the figures next to each other
% gives me the compilation error. We temporarily comment it
% out.
%\figw{iperf}{10}{Place holder for iperf performance}
%\figw{cpu_util_iperf}{10}{Place holder for iperf CPU utiliztion}


\begin{table}[tbp]
\begin{tabular}{|l|l|l|}
\hline
& \begin{tabular}[c]{@{}l@{}}BEFORE\\ OPTI\end{tabular} & \begin{tabular}[c]{@{}l@{}}AFTER\\ OPTI\end{tabular} \\ \hline
HLT & 4785929 & 0 \\ \hline
EXTERNAL INTERRUPT & 13114 & 29881 \\ \hline
IO INSTRUCTION & 1110 & 1073 \\ \hline
MSR READ & 120 & 120 \\ \hline
MSR WRITE & 235121 & 269695 \\ \hline
PAUSE INSTRUCTION & 23 & 0 \\ \hline
PREEMPTION TIMER & 16260 & 36063 \\ \hline
TOTAL & 5051677 & 336832 \\ \hline
\end{tabular}
\caption{Comparison of VM Exit Before and After the CPU
Optimization. The VM exits are recorded for 60 seconds when
the guest with the passthrough NIC generates the TCP outgoing
traffic. The HLT exiting is eliminated.}
\label{tab:vm_exit}
\end{table}


\begin{table}[tbp]
\begin{tabular}{|l|l|l|}
\hline
& \begin{tabular}[c]{@{}l@{}}OUTGOING\\ TRAFFIC\end{tabular} & \begin{tabular}[c]{@{}l@{}}INCOMING\\ TRAFFIC\end{tabular} \\ \hline
BAREMETAL & $37.39 \pm 0.08$ & $37.52 \pm 0.10$ \\ \hline
\begin{tabular}[c]{@{}l@{}}GUEST\\ + vHOST\end{tabular} & $37.39 \pm 0.04$ & $19.02 \pm 0.50$ \\ \hline
\begin{tabular}[c]{@{}l@{}}GUEST\\ + VFIO\end{tabular} & $37.45 \pm 0.08$ & $37.58 \pm 0.15$ \\ \hline
\begin{tabular}[c]{@{}l@{}}GUEST\\ + VFIO\\ + OPTIMIZATION\end{tabular} & $37.37 \pm 0.11$ & $37.52 \pm 0.15$ \\ \hline
\end{tabular}
\caption{Comparison of Network Bandwith for the Outgoing and
Incoming TCP Traffic. We measured the network bandwidth for
the (a) baremetal, (b) guest using the Virtio frontend backed
by the host vHost driver, (c) guest using the assigned NIC and
(d) guest using the assigned NIC with CPU optimization.}
\label{tab:network_bandwidth}
\end{table}

\begin{table}[tbp]
\begin{tabular}{|l|l|}
\hline
& RTT ($\mu$s) \\ \hline
BAREMETAL & $12 \pm 3$ \\ \hline
\begin{tabular}[c]{@{}l@{}}GUEST\\ + vHOST\end{tabular} & $24 \pm 5$ \\ \hline
\begin{tabular}[c]{@{}l@{}}GUEST\\ + VFIO\end{tabular} & $13 \pm 2$ \\ \hline
\begin{tabular}[c]{@{}l@{}}GUEST\\ + VFIO\\ + OPTIMIZATION\end{tabular} & $13 \pm 5$ \\ \hline
\end{tabular}
\caption{Comparison of Network Latency. We measured the
network latency for the (a) baremetal, (b) guest using the
Virtio frontend backed by the host vHost driver, (c) guest
using the assigned NIC and (d) guest using the assigned NIC
with CPU optimization.}
\label{tab:network_latency}
\end{table}

\begin{table}[tbp]
\begin{tabular}{|l|l|l|l|}
\hline
& \%USER & \%SYSTEM & \%GUEST \\ \hline
BAREMETAL & 0.64 & 80.98 & -- \\ \hline
\begin{tabular}[c]{@{}l@{}}GUEST\\ + vHOST\end{tabular} & 68.6 & 84.36 & 68.6 \\ \hline
\begin{tabular}[c]{@{}l@{}}GUEST\\ + VFIO\end{tabular} & 90.48 & 85.08 & 90.48 \\ \hline
\begin{tabular}[c]{@{}l@{}}GUEST\\ + VFIO\\ + OPTIMIZATION\end{tabular} & 199.76 & 0.24 & 199.76 \\ \hline
IN GUEST & 0.66 & 81.62 & -- \\ \hline
\end{tabular}
\caption{Comparison of CPU Utilization. When the guest
generates the TCP outgoing traffic, we measured the CPU
utilization of 2 cores in the host with the following
configuration: (a) baremetal, (b) guest using the Virtio
frontend backed by the host vHost driver, (c) guest using the
assigned NIC and (d) guest using the assigned NIC with CPU
optimization.  We also measure the CPU utilization in the
guest using (d). \%USER is the \%CPU time consumed in the user
mode, \%SYSTEM is the \%CPU time consumed in the kernel mode
and \%GUEST is the \%CPU time consumed by the guest.}
\label{tab:cpu_utilization_40gbps}
\end{table}

In this experiment, we demonstrate that the \na achieves near
bare-metal performance with minimum CPU utilization and
minimal hypervisor involvement in the following three metrics.
\begin{itemize}
  \item Network latency.
  \item Network bandwidth.
  \item CPU utilization.
\end{itemize}

We measure the performance using macro benchmark in (a) Host
(b) VM with virtio-net interface and (c) VFIO interface.
Additionally, we also show that we eliminate the VM exits due
to HLT instruction.

The network latency is measured by the round-trip delay using
the ping tool. In Table~\ref{tab:network_latency}, we observe
that the network latency in host for 1Gbps and 40Gbps NIC is
-- and --$\mu$s. With virtio network front-end device and
vhost back-end driver, the network latency in guest for 1Gbps
and 40Gbps is -- and --$\mu$s. The network latency in guest
using the assigned 1Gbps and 40Gbps NIC is -- and --$\mu$s.
(c) is out performed (b) by --\%, while matching the network
latency in host.

We measure the TCP throughput in (a)-(c) using iperf benchmark
\cite{iperf}. In Table~\ref{tab:network_bandwidth}, we observe that
the network bandwidth in host for a 1Gbps and 40Gbps NIC is
--Mbps and --Gbps respectively. With virtio network front-end
device and vhost back-end driver, the network bandwidth in
guest for 1Gbps and 40Gbps is --Mbps and --Gbps. The guest
achieves near bare metal performance using configuration (b)
and (c).  However, the CPU utilization differs between (b) and
(c). With virtio network device backed with vhost driver, the
CPU utilization in system mode is --\% and --\% in guest mode.
With NIC passthrough we observe that the CPU utilization is in
system mode is --\% and --\% in guest mode. The CPU overhead
with passthrough NIC is high compared to virtio network backed
by vhost driver. The vCPUs in (c) execute HLT instruction by
--\% more frequently than the vCPUs in (b) when the guest is
idle. The vCPUs burn CPU cycles when they poll for sometime
before executing the HLT instruction. The vCPU polling
mechanism keeps the CPU busy leading to higher CPU
utilization.

As explained before, to reduce the CPU utilization in host, we
disable the HLT exits by modifying the VMCS structure in KVM.
To avoid other system processes to compete with guest, the
VCPUs are pinned on isolated CPUs. As shown in \ref{}, we
notice that disabling HLT exits along with dedicating cores to
guest, reduces the CPU utilization in system mode and
increases the CPU Utilization in guest mode indicating that
the guest occupies the CPUs for most of the time. In \ref{} we
show that the number of VM exits due to HLT instruction is
zero.


\vspace{-0.1in}
\subsection{Direct Interrupt Delivery Efficiency}
\vspace{-0.05in}
% DID efficiency.

\figw{cyclictest_in_kernel}{8.5}{The cumulative distribution
function of timer interrupt latency measured for the
Bare-metal, DTID, DID, and unmodified-guest (VANILLA)
configuration which does not use our optimizations.}

\figw{ipi_latency}{8.5}{The cumulative distribution function
of IPI latency measured for the Bare-metal, DTID, DID, and
unmodified-guest (VANILLA) configuration which does not use our
optimizations.}

\begin{table}
\renewcommand{\arraystretch}{1.2}
\small
\begin{center}
\begin{tabular}{|l|c|c|c|c|}
\hline
&
\multicolumn{1}{c|}{\begin{tabular}[c]{@{}c@{}}\textbf{B/W}\\ \textbf{(Gbps)}\end{tabular}} & \multicolumn{1}{c|}{\begin{tabular}[c]{@{}c@{}}\textbf{CPU}\\ \textbf{Core}\end{tabular}} & \multicolumn{1}{c|}{\begin{tabular}[c]{@{}c@{}}\textbf{NIC INTR}\\ \textbf{(per sec)}\end{tabular}} & \multicolumn{1}{c|}{\begin{tabular}[c]{@{}c@{}}\textbf{TMR INTR}\\ \textbf{(per sec)}\end{tabular}} \\ \hline
\multirow{2}{*}{\textbf{Bare-Metal}} & \multirow{2}{*}{37.39} & 0 & 118592 & 256 \\ \cline{3-5}
&  & 1 & 0 & 348 \\ \hline
\multirow{2}{*}{\textbf{DTID}} & \multirow{2}{*}{37.35} & 0 & 110856 & 255 \\ \cline{3-5}
&  & 1 & 0 & 348 \\ \hline
\end{tabular}
\end{center}
\vspace{-0.05in}
\caption{Comparison of the number of network and timer
interrupts per second experienced between a Bare-metal server
and a DTID HaaS VM, when they send out TCP traffic over a
40Gbps Infiniband link at full speed using iPerf.
\textbf{B/W}: Bandwidth. \textbf{TMR}: Timer. \textbf{INTR}:
Interrupt.}
\label{tab:network_interrupts}
\vspace{-0.06in}
\end{table}

To evaluate the direct local interrupt delivery, we measured
both the timer interrupt and IPI latency. The timer interrupt
latency was the timing difference between the expected and
actual wakeup time. We ran the experiment with the sleep time
of 200$\mu$s for 10 million iterations. For the IPI, our own
IPI vector was registered in the kernel. When such an IPI was
sent or received, the timestamps were recorded. The latency
was the difference between the tranmission and receipt time.
In our experiment, IPI was sent from the core A to B with the
interval of 200$\mu$s for 10 million iterations.

As shown in Figure~\ref{fig:cyclictest_in_kernel}, the median
timer interrupt latency for the Bare-metal, DID, DTID and
VANILLA configuration, which did not use our optimizations
except the dedicated cores, were 1.51, 2.03, 2.04 and 9.20
$\mu$s, respectively. Both the a DTID and DID HaaS VM came
close to the bare-metal OS with the additional 0.52$\mu$s
latency and reduced the median timer interrupt latency by
$78\%$ over a VANILLA HaaS VM. For the IPI latency in
Figure~\ref{fig:ipi_latency}, the median IPI latency for
Bare-metal, DID, DTID and VANILLA configuration were 1.80,
2.89, 3.33 and 9.35, respectively. DID HaaS VM had 1$mu$s
extra latency than Bare-metal and reduced the median latency
by $69\%$ over a VANILLA HaaS VM. Because a DTID or DID HaaS
VM still experienced VM exits due to reasons such as
privileged instructions and required to set the correct bit in
the PID page, its median timer interrupt latency is 0.52$\mu$s
more than the bare-metal OS. Reducing this difference further
by identifying the underlying VM exit reasons is part of our
future work.

Since both the NIC and timer-interrupt delivery used the
posted-interrupt mechanism, it was possible that the timer
interrupts was triggered by the network interrupts and arrived
earlier than the expected time. Then, the early timer
interrupt was serviced and the timer was updated. Nonetheless,
the more NIC interrupts occured, the more early timer
interrupts needed to be handled. When the DTID HaaS VM
performed the network I/O over the 40Gbps Infiniband link,
spurious timer interrupts did not only reduce the network
bandwidth performanc, but also kept CPU busy handling
additional timer interrupts. The filtering mechansim was
implemented to ignore early timer interrupts. Its
effectiveness was determined by measuring the frequency of NIC
and timer interrupts for the DTID haas VM, when it transmitted
the TCP traffic over a 40Gbps Infiniband link by
iPerf~\cite{iperf}.

In our experiment, the CPU core 0 was configured to process
NIC interrupts and 250Hz local timer interrupts, whereas the
CPU core 1 ran the TCP benchmark and processes 250Hz local
timer interrupts. As shown in
Table~\ref{tab:network_interrupts}, the timer interrupt rate
of CPU core 0 of the DTID configuration was almost the same as
that of the Bare-metal configuration. It demonstrated that
\sna's spurious timer interrupt filtering mechanism worked
correctly. Although the DTID configuration incurred extra
filter processing overhead for every NIC interrupt, this
overhead had negligible impact on its network bandwidth
performance. The NIC interrupt rate of CPU core 0 of the DTID
configuration was slightly lower than that of the Bare-metal
configuration, because some of NIC interrupts were processed
as part of timer interrupt processing. The timer interrupt
rate of CPU core 1 was higher than that of CPU core 0 in both
configurations because CPU core 1 runs the iPerf benchmark,
which set up aperiodic timers in addition to the system's
periodic 250Hz timers.


%%%%%%%%%%%%%%%%%%%%%%%%%%%%%%%%%%%%%%%%%%%%%%%%%%%%%%%%%%%%%%%%%%%%%%%%%%%%%%%%%%%%%%%%%%%%%%%%%%%%%%%%%%%%%%%%%%%%%%%%%%
\mycomment{
\begin{table}[tbp]
\begin{tabular}{|l|l|l|l|l|}
\hline
& IB & CPU & \begin{tabular}[c]{@{}l@{}}NIC\\ INTR\end{tabular} & \begin{tabular}[c]{@{}l@{}}TMR\\ INTR\end{tabular} \\ \hline
\multirow{2}{*}{HOST} & \multirow{2}{*}{37.39} & 0 & 118592 & 256 \\ \cline{3-5}
&  & 1 & 0 & 348 \\ \hline
\multirow{2}{*}{OPTI Guest} & \multirow{2}{*}{37.37} & 0 & 123024 & 256 \\ \cline{3-5}
&  & 1 & 0 & 347 \\ \hline
\multirow{2}{*}{\begin{tabular}[c]{@{}l@{}}DTID Guest \end{tabular}} & \multirow{2}{*}{37.35} & 0 & 110856 & 255 \\ \cline{3-5}
&  & 1 & 0 & 348 \\ \hline
\end{tabular}
\caption{Comparison of Network Interrupts Between Host and
Guest. The interrupts are reported as the number of interrupts
per second. For the DTID guest, the numbers of spurious timer
interrupts per second for core 0 and 1 are 100114 and 3933
respectively. OPTI guest uses the assigned NIC with the CPU
optimization. IB: 40Gbps Infiniband. NIC: network interface
card. INTR: interrupt. TMR: timer.}
\label{tab:network_interrupts}
\end{table}
}

\mycomment{
We measure the expected frequency of timer and network
interrupts, when running the iPerf benchmark over the 40Gbps
Infiniband. Both the host and guest uses two cores. One core
handles the network interrupts, while the other core runs the
iPerf benchmark. In Table~\ref{tab:network_interrupts}, the
host and guest both receive the expected frequency of timer
and network interrupts and saturate the 40Gbps link.

Nonetheless, DTID guest needs to handle the spurious timer
interrupt before processing the network interrupts. This is
the addition CPU overhead. Since the PIR timer-interrupt is
almost always set, for each network interrupt, there is a
spurious timer interrupt. The frequency of network interrupt
and spurious timer interrupts are 110856 and 100114
respectively. The DTID algorithm simply ignores all the
spurious timer interrupts. The DTID guest is able to match the
bare-metal network bandwidth and latency, while having 1.2\%
additional CPU overhead to handle the spurious timer
interrupts as shown in Figure~\ref{fig:network_bandwidth},
Figure~\ref{fig:network_latency}, and
Table~\ref{tab:cpu_utilization_40gbps}.

Moreover, we observe that the spurious interrupt occurs on the
core running the iPerf benchmark, but there is no network
interrupts. Since the spurious timer interrupt also happens
after the VM entry, the  analysis suggests it is due to
other type of VM exits such as IO instructions, MSR writes or
IPIs.

% TODO: atopsar and micro-benchmark
Furthermore, We measure the overhead of handling spurious
interrupt. It takes --$\mu$s, while the typical handling of
timer interrupt is --$\mu$s. The data suggests our algorithm
works efficiently to deliver the timer interrupt and ignore
the spurious timer interrupts.

% DTID scalability parallel processing Kernbench
We are able to scale the DTID algorithm to all the 9 cores,
while the host has 1 dedicated core. Each vCPU receives the
expected number of timer interrupts. When the DTID guest uses
the periodic timer of 250Hz, each vCPU receives around 250
interrupts per second. To see how well the DTID guest performs
the parallel processing, we run Kernbench~\cite{kernbench} or
PARSEC~\cite{bienia:2008} benchmark. The result shows
~\ref{fig:kernbench}....
}

\mycomment{
\begin{table}[tbp]
\begin{tabular}{|l|l|l|l|l|l|}
\hline
& IB & CPU & \begin{tabular}[c]{@{}l@{}}NIC\\ INTR\end{tabular} & \begin{tabular}[c]{@{}l@{}}SPU\\ TMR\\ INTR\end{tabular} & \begin{tabular}[c]{@{}l@{}}TMR\\ INTR\end{tabular} \\ \hline
\multirow{2}{*}{\begin{tabular}[c]{@{}l@{}}DTID Guest \end{tabular}} & \multirow{2}{*}{37.35} & 0 & 110856 & 100114 & 255 \\ \cline{3-6}
&  & 1 & 0 & 3933 & 348 \\ \hline
\end{tabular}
\caption{Analysis of Spurious Timer Interrupts. The interrupts
are reported as the number of interrupts per second. DTID
guest uses the assigned NIC with the CPU optimization and DTID
enabled. IB: 40Gbps Infiniband. NIC: network interface card,
INTR: interrupt, SPU TMR INTR: spurious timer interrupt.}
\label{tab:spurious_timer_interrupt}
\end{table}
}

\mycomment{
\begin{table}
\renewcommand{\arraystretch}{1.2}
\small
\begin{center}
\begin{tabular}{|l|c|c|c|c|} \hline
& {\bf B/W} & {\bf CPU} & {\bf NIC INTR} & {\bf Tmr INTR} \\
& {\bf (Gbps)} & {\bf Core} & {\bf (per sec)} & {\bf (per sec)}  \\ \hline
\multirow{2}{*}{\bf Bare-metal} & \multirow{2}{*}{37.39} & 0 & 118592 & 256 \\ \cline{3-5}
&  & 1 & 0 & 348 \\ \hline
%\multirow{2}{*}{\bf OPTI} & \multirow{2}{*}{37.37} & 0 & 123024 & 256 \\ \cline{3-5}
%&  & 1 & 0 & 347 \\ \hline
\multirow{2}{*}{\begin{tabular}[c]{@{}l@{}} {\bf DTID} \end{tabular}} & \multirow{2}{*}{37.35} & 0 & 110856 & 255 \\ \cline{3-5}
&  & 1 & 0 & 348 \\ \hline
\end{tabular}
\end{center}
\vspace{-0.05in}
\caption{Comparison of the number of network and timer
interrupts per second experienced between a Bare-metal server
and a DTID HaaS VM, when they send out TCP traffic over a
40Gbps Infiniband link at full speed using iPerf.
\textbf{B/W}: Bandwidth. \textbf{Tmr}: Timer. \textbf{INTR}:
Interrupt.}
\label{tab:network_interrupts}
\vspace{-0.06in}
\end{table}
}


\vspace{-0.1in}
\subsection{Parallel Application Performance}
\vspace{-0.05in}
% Multi-threaded Computation Performance

\figw{parsec_barchart}{8.5}{Slowdown of PARSEC benchmark
programs for VANILLA and DID guests compared to bare-metal.}

\figw{fpspeed_barchart}{8.5}{Slowdown of floating point
computation benchmarks in SPEC CPU 2017~\cite{bucek:2018} for
VANILLA and DID guests compared to bare-metal.}

\figw{intrate_barchart}{8.5}{Throughput reduction of integer
computation benchmarks in SPEC CPU 2017~\cite{bucek:2018} for
VANILLA and DID guests compared to bare-metal.}

To evaluate how well a HaaS VM performed for concurrent
CPU-bound applications, we measured the percentage of slowdown
or throughput reduction. Two popular thread models were
considered: Pthreads~\cite{lewis:1998} and
OpenMP~\cite{dagum:1998}. Pthreads~\cite{lewis:1998} provided
the finer-grained control over the thread management, while
OpenMP~\cite{dagum:1998} was the industry standard and easier
to scale than Pthreads~\cite{lewis:1998}. To measure the
slowdown, we used SPECspeed 2017 Floating
Point~\cite{bucek:2018} and PARSEC~\cite{lewis:1998}. To
measure the throughput reduction, we used SPECrate 2017
Integer~\cite{bucek:2018}.

In our experiment, the HaaS VM had 8 dedicated cores and 27GB
RAM whereas \na had the two cores, including physical core 0,
and the remaining 5GB RAM. Moreover, the topology of L1/L2/L3
CPU cache was exposed to the VM. The bare-metal was the
baseline and had the same configuration as the VM. We compared
the percent slowdown between the VMs of DID and vanilla
configuration, which did not use our optimizations except the
dedicated cores and cache information. To To measure the
slowdown, we used 8 threads in each in SPEC
CPU~\cite{bucek:2018} benchmark programs. To measure the
throughput reduction, we used 8 processes in each SPEC
CPU~\cite{bucek:2018} and PARSEC~\cite{lewis:1998} benchmark
programs.

For the cases of OpenMP~\cite{dagum:1998} in
Figure~\ref{fig:fpspeed_barchart}, the maximum performance
slowdown was 3\%. DID improved the performance further and did
not perform worse than the vanilla VM for the wide-scale ocean
modeling. For the cases of Pthreads~\cite{lewis:1998} in
Figure~\ref{fig:parsec_barchart}, most cases had the
percentage slowdown less than 5\% except Canneal and
Streamcluster. DID reduced the slowdown even further.

Especially for Dedup, DID trimmed 11.45\% of slowdown to
3.85\%. Although the performance of DID in Vips seemed to be
worse than Vanilla, the absolute values differed less than 0.1
seconds. For Canneal and Streamcluster, even after tracking
and eliminating all VM exits triggered by EPT violations and
I/O instructions from certain virtual devices (virtual floppy
and CD-ROM), the performance of Canneal and Streamcluster did
not significantly improve. We suspect that the performance
difference for these two cases might be due to VTx-related
architecture issues; we are further investigating the
underlying causes.

% After analyzing the VM Exit reasons, we eliminated
%the two expensive VM exits: (1) I/O instructions and (2) EPT
%violations. To eliminate VM exits triggered by I/O
%instructions, we removed the virtual devices for floppy disk
%and CD-ROM, which accounted for about 20 I/O instructions per
%second. To eliminate VM exits triggered by EPT violation, the
%HaaS agent in the VM touched almost all guest pages at
%initialization time, which forced the \na to populate the
%corresponding EPT entries ahead of time. These two
%optimizations reduced the rate of VM exits to around 11 VM
%exits per second. We re-ran the two benchmark programs and
%observed it had only marginal performance improvement. 

%As shown in Figure~\ref{fig:intrate_barchart}, DID improved
%the CPU throughput in all cases which had less than 10\% of
%throughput reductions. For 


\vspace{-0.1in}
\subsection{Live Migration Performance}
\vspace{-0.05in}
% Migration performance.
% - Describe what we have done.
% - Include tables and figures.
% - Evaluate the performance and see if it match our goal.

\figw{seamless_migration}{10}{Place holder for seamless
migration. Top: Guest sends the outgoing traffic. Bottom:
guest receives the incoming traffic.}

\begin{table}[tbp]
\begin{tabular}{|l|l|l|l|}
\hline
& UNPLUG & PLUG & OPTI. PLUG \\ \hline
UDP SENDER & 4.6 & 300 & 0 \\ \hline
UDP RECEIVER & 1.8 & 300 & 0 \\ \hline
\end{tabular}
\caption{Network Downtime}
\label{tab:migration_network_downtime}
\end{table}

In this experiment, we show the network downtime when
switching between the assigned and Virtio network device. 
The guest is configured with bonding driver in active-
backup mode where the assigned device acts as 
active interface and Virtio network device as 
backup interface. We also demonstrate the average 
number of missed local timer interrupts in guest, 
when we disable the DTID.

In Figure~\ref{fig:default_seamless_migration}, we measure 
the bandwidth of 1Gbps network card using iperf. 
The guest sends the network traffic at 940Mbps bandwidth 
during regular runtime. When the network traffic is 
switched by bonding driver from assigned device to 
Virtio network device on hot unplug of assigned device, 
the guest experiences 0.3seconds network downtime. 
During the migration of the guest, the network
bandwidth remains to be 940Mbps using Virtio network
device. The guest experiences network dowtime of 
0.1seconds during the last phase of migration 
when the guest is paused on source host
to transfer the CPU and I/O state to destination host
for guest resumption. After the guest resumes on the
destination host, the guest is setup to use the assigned
device through hot plug event. After the assigned device
is hot plugged, the network traffic is switched from Virtio
network device to assigned device by setting the assigned
device as the active device.


In Table~\ref{tab:migration_network_downtime}, when switching
from the assigned to Virtio network device, the network is --
and -- ms for the UDP sender and receiver respectively. When
switching back from the Virtio to assigned network device, the
network is -- and -- ms for the UDP sender and receiver
respectively. After we hide the overhead of assigned NIC
creation during the VM migration, the network downtime is
reduced to -- and -- ms for the UDP sender and receiver
respectively.

When disabling the DTID, the timer interrupts received by the
guest per seconds matches the expected frequency of timer
interrupt received by the unmodified guest. Since KVM delivers
the virtual interrupts with or without DTID, the guest can
still receives its interrupt during the transition. The longer
the transition takes, the later the migration starts. The host
communicates with the guest by the TCP transmission. We expect
most of the transition time is due to the packet transmission
and processing. The average transition time is -- $\mu$s.

% TODO: Analysis of DTID algorithm in the guest shows the
% guest does not miss the next timer interrupt.



%%%%%%%%%%%%%%%%%%%%%%%%%%%%%%%%%%%%%%%%%%%%%%%%%%%%%%%%%%%%%%%%%%%%%%%%%%%
\mycomment{
In this section we present our solutions using macro
and micro-benchmarks. We demonstrate the following.
\begin{itemize}
  \item The guest using our optimization matches the baremetal
  network performance with the minimum CPU overhead.
  \item The guest with DTID improves the timer interrupt
  latency by directly handling the timer interrupt and
  updating the timer.
  \item The guest with DTID ignores the NIC-induced timer
  interrupts while achieving the baremetal network bandwidth
  and latency performance with low CPU overhead.
  \item DTID is scalable to all CPUs, while the host has its
  dedicated CPU.
  \item The guest with DTID matches the baremetal CPU
  throughput.
  \item The guest using the bonding driver and our QEMU
  optimization does not experience the apparent network
  downtime during the migration.
  \item The total memory consumption of idle hypervisor is
  between 80 to 120MB of RAM.
\end{itemize}

In addition, the total number lines changed in the guest OS is
387. First, 43 lines of code are added to the timer interrupt
handler to set the PIR timer-interrupt bits and ignore the
spurious timer interrupt. Second, the kernel module of 110
lines is used to adjust for the LAPIC timers. Since the timers
are calibrated by the host at boot time, the guest needs to
adjust its multiplication and shift factor of tick calculation
to the host configuration. The module consumes 16KB of RAM.
Third, the kernel module of 234 lines is used to map or unmap
the shared PID page. The module consumes 16KB of RAM plus $n$
shared PID pages, where each page is 4KB.
}
