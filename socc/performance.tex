\section{Performance Evaluation}
\vspace{-0.05in}

\mycomment{
In this section we present our solutions using macro
and micro-benchmarks. We demonstrate the following.
\begin{itemize}
  \item The guest using our optimization matches the baremetal
  network performance with the minimum CPU overhead.
  \item The guest with DTID improves the timer interrupt
  latency by directly handling the timer interrupt and
  updating the timer.
  \item The guest with DTID ignores the NIC-induced timer
  interrupts while achieving the baremetal network bandwidth
  and latency performance with low CPU overhead.
  \item DTID is scalable to all CPUs, while the host has its
  dedicated CPU.
  \item The guest with DTID matches the baremetal CPU
  throughput.
  \item The guest using the bonding driver and our QEMU
  optimization does not experience the apparent network
  downtime during the migration.
  \item The total memory consumption of idle hypervisor is
  between 80 to 120MB of RAM.
\end{itemize}

In addition, the total number lines changed in the guest OS is
387. First, 43 lines of code are added to the timer interrupt
handler to set the PIR timer-interrupt bits and ignore the
spurious timer interrupt. Second, the kernel module of 110
lines is used to adjust for the LAPIC timers. Since the timers
are calibrated by the host at boot time, the guest needs to
adjust its multiplication and shift factor of tick calculation
to the host configuration. The module consumes 16KB of RAM.
Third, the kernel module of 234 lines is used to map or unmap
the shared PID page. The module consumes 16KB of RAM plus $n$
shared PID pages, where each page is 4KB.
}

% Describe the tested including the following items.
% - Hardware configuration: CPU, memory and network cards.
% - Guest: CPU, memory, bonding driver, assigned and virtual devices.
% - QEMU
% - KVM
%\figw{cpu_state_diagram}{8}{CPU State Diagram}

The experiments are run on machines equipped with the 10-core
Intel Xeon CPU E4 v4 of 2.2GHz, 32GB memory, 40Gbps Mellanox
ConnectX-3 Pro network interface and Intel Corporation
Ethernet Connection I217-LM. The Linux kernel 4.10.1 and QEMU
2.9.0 are installed in the host. The guest is configured with
1 to 9 vCPUs, 10GB of RAM, 1 Virtio and 1 pass-through network
device. The Linux kernel of 4.10.1 and the Ethernet bonding
driver are installed in the guest. The bonding driver operates
in active-backup mode.

The tools to measure the CPU, memory and network I/O
performance are listed as follows. iPerf 2.0.5~\cite{iperf}
measures the network bandwidth. Ping~\cite{ping} measures the
round-trip delay. Atopsar 2.3.0~\cite{atopsar} measures the
CPU utilization. Free 3.3.10~\cite{free} measures the memory
consumption. Perf 4.10.1~\cite{perf} measures the number of VM
exits. Cyclictest 0.93~\cite{cyclictest} benchmarks the timer
interrupt latency.
%Kernbench 0.42~\cite{kernbench} benchmarks the CPU throughput.
The following configurations are evaluated:
%depending on the physical or virtual network device, CPU
%optimization and DTID.
\mycomment{
\begin{enumerate}[(a)]
 \item The guest uses the Virtio network device backed by the
  vHost driver (Guest + vHost).
  \item The guest uses the assigned network device (Guest +
  VFIO).
  \item The guest uses the assigned network device. We also
  apply the CPU optimization (OPTI Guest). There are no VM
  exits due to the network interrupt and HLT instruction.
  \item The guest uses the assigned network device. We apply
  both the CPU optimization and DTID (DTID Guest). guest.
  There are no VM exits due the network interrupts, HLT
  instruction, local timer interrupts, direct timer updates or
  EPT violations when accessing the shared PID page.
\end{enumerate}
}

\begin{itemize}
\parskip 0mm
\itemsep 0mm
\item {\bf Bare-metal}: A machine without virtualization.

\item {\bf VHOST}: A HaaS VM accessing I/O devices using the
                   vHost interface.

\item {\bf VFIO}: A HaaS VM accessing I/O devices using the
                  VFIO interface without incurring VM exits
                  due to network interrupts.

\item {\bf OPTI}: A HaaS VM accessing I/O devices using the
                  VFIO interface without incurring VM exits
                  due to network interrupts or HLT
                  instructions.

\item{\bf  DTID}: A HaaS VM accessing I/O devices using the
                  VFIO interface without incurring VM exits
                  due to network interrupts, HLT instructions
                  or local timer interrupts.

\item{\bf  DID}: A HaaS VM accessing I/O devices using the
                 VFIO interface without incurring VM exits due
                 to network interrupts, HLT instructions or
                 local timer interrupts or IPIs.
\end{itemize}


\mycomment{
In Figure~\ref{fig:cpu_state_diagram}, it shows the transition
among host and different guest configurations. The control is
transferred to the host upon a VM exit. After the host has
done it emulation, the control is return back to the guest. In
the case of live migration, OPTI or DTID guest are reverted
back to the unmodified guest before the migration starts.
After the migration ends, the unmodified guest is again
transformed to the OPTI or DTID guest.
}


In our experiment, it is necessary to use two CPU cores to
saturate a 40Gbps Infiniband link for all configurations. One
core is handling the interrupts and soft IRQs, while the other
is running the network performance testing workload. We use a third
core to monitor the CPU utilization, which does not
affect the network performance. In contrast, only one
core is needed to saturate a 1 Gbps Ethernet link.


\vspace{-0.1in}
\subsection{Network I/O Performance}
\vspace{-0.05in}
% CPU and network performance of assigned NIC
% - Describe what we have done.
% - Include tables and figures.
% - Evaluate the performance and see if it match our goal.

% For some reason, putting the figures next to each other
% gives me the compilation error. We temporarily comment it
% out.
%\figw{iperf}{10}{Place holder for iperf performance}
%\figw{cpu_util_iperf}{10}{Place holder for iperf CPU utilization}



\mycomment{
\begin{table}[tbp]
\begin{tabular}{|l|l|l|l|}
\hline
& \begin{tabular}[c]{@{}l@{}}Outbound\\ (Gbps)\end{tabular} & \begin{tabular}[c]{@{}l@{}}Inbound\\ (Gbps)\end{tabular} & \begin{tabular}[c]{@{}l@{}}RTT\\ ($\mu$s)\end{tabular} \\ \hline
\begin{tabular}[c]{@{}l@{}} {\bf Bare-metal}\end{tabular}     & 37.39 & 37.52 & 12\\ \hline
\begin{tabular}[c]{@{}l@{}} {\bf VHOST} \end{tabular} & 37.39 & 37.10 & 18\\ \hline
\begin{tabular}[c]{@{}l@{}} {\bf VFIO}\end{tabular}  & 37.45 & 37.58 & 13\\ \hline
\begin{tabular}[c]{@{}l@{}} {\bf OPTI} \end{tabular}    & 37.37 & 37.52 & 13\\ \hline
\begin{tabular}[c]{@{}l@{}} {\bf DTID} \end{tabular}    & 37.35 & 37.50 & 12\\ \hline
%%\begin{tabular}[c]{@{}l@{}}GUEST\\ + vHOST\end{tabular}                          & $37.39 \pm 0.04$ & $19.02 \pm 0.50$ & $24 \pm 5$\\ \hline
%%\begin{tabular}[c]{@{}l@{}}GUEST\\ + VFIO\end{tabular}                           & $37.45 \pm 0.08$ & $37.58 \pm 0.15$ & $13 \pm 2$\\ \hline
%%\begin{tabular}[c]{@{}l@{}}GUEST\\ + VFIO\\ + OPTIMIZATION\end{tabular}          & $37.37 \pm 0.11$ & $37.52 \pm 0.15$ & $13 \pm 3$\\ \hline
%%\begin{tabular}[c]{@{}l@{}}GUEST\\ + VFIO\\ + OPTIMIZATION\\ + DTID\end{tabular} & $37.35 \pm 0.08$ & $37.50 \pm 0.23$ & $12 \pm 5$\\ \hline
\end{tabular}
\caption{Comparison of network bandwidth and network packet latency over a
40Gbps Infiniband link for the five configurations evaluated.}
\label{tab:network_performance}
\end{table}
}

\begin{table}
\renewcommand{\arraystretch}{1.2}
\small
\begin{center}
\begin{tabular}{|c|c|c|c|} \hline
{\bf Configuration} & {\bf Outbound} & {\bf Inbound}  & {\bf Round-Trip} \\ 
 & {\bf (Gbps)} & {\bf (Gbps)} & {\bf Delay ($\mu$s)} \\ \hline
 {\bf Bare-metal}  & 37.39 & 37.52 & 12\\ \hline
 {\bf VHOST} & 37.39 & 19.02/37.10 & 24/18\\ \hline
{\bf VFIO}  & 37.45 & 37.58 & 13\\ \hline
 {\bf OPTI} & 37.37 & 37.52 & 13\\ \hline
 {\bf DTID} & 37.35 & 37.50 & 12\\ \hline
\end{tabular}
\end{center}
\vspace{-0.1in}
\caption{Comparison of network throughput and round-trip packet delay over a
40Gbps Infiniband link among the five evaluated configurations}
\label{tab:network_performance}
\vspace{-0.1in}
\end{table}


In this experiment, we demonstrate that, with all  the proposed optimizations applied,  a HaaS VM performs essentially the same as a bare-metal server in terms of network throughput, network packet latency and CPU utilization.


%the guest using our
%optimization achieves near bare-metal performance with minimum
%CPU utilization and minimal hypervisor involvement. We use the
%macro benchmark and measure the performance in the following
%three metrics: network bandwidth, latency and CPU utilization.

We use  iPerf~\cite{iperf} to measure the network throughput of
incoming and outgoing TCP traffic on a 40Gbps Infiniband link.
As shown in Table~\ref{tab:network_performance}, for outgoing TCP traffic, 
all five evaluated configurations are able to saturate the Infiniband link and 
have almost identical network throughput. 
%performance for the gigabit link is not shown here. All the
%configurations are able to saturate the 1Gbps NIC and 40 Gbps
%Infiniband for the outgoing TCP throughput. 
For incoming TCP  traffic, all five configurations except the VHOST configuration have roughly the same network throughput; however, the VHOST configuration's network throughput 
is about 52\% of that of the other four configurations when two CPU cores are used, and is comparable to that of the other four configurations only when three CPU cores are used. 



We use ping~\cite{ping} to measure the round-trip delays over a 40Gbps Infiniband link and use them to compare the network packet latency of the evaluated configuration. 
As shown in Table~\ref{tab:network_performance}, when two CPU cores are used, the round-trip delay over an Infiniband link in the 
VHOST configuration is 24$\mu$s, which is about twice as much as that of the other four configurations, between 12-13$\mu$s.
The extra delay arises because, when a HaaS VM runs in the VHOST configuration, VM exists occur whenever it access virtual I/O devices and receives interrupts from these devices.




%For the gigabit link, the RTT for the host
%and guest using VFIO configuration is about 160$\mu$s, whereas
%the RTT of guest using VHOST as the backend driver is
%173$\mu$s. For the 40Gbps Infiniband, the guest using the VFIO
%out performs the guest using the VHOST configuration by 50\%.
%As shown in 
%Figure~\ref{fig:network_latency}, 
%the RTT for the
%host and guest using VFIO configuration is about 13$\mu$s,
%whereas the RTT of guest using VHOST as the backend driver is
%24$\mu$s.

% TODO: Put them together as the sub-figures.
% TODO: May need to redo the figure and make the label clear.
%\figw{network_bandwidth}{9}{Comparison of Network Bandwidth over 40Gbps Link}
%\figw{network_latency}{9}{Comparison of Network Latency over 40Gbps Link}

\mycomment{
\begin{table}[tbp]
\begin{tabular}{|l|l|l|l|}
\hline
& \%User & \% System & \%Guest \\ \hline
\begin{tabular}[c]{@{}l@{}} {\bf Bare-metal}\end{tabular}     & 0.64   & 80.98 & -- \\ \hline
\begin{tabular}[c]{@{}l@{}} {\bf vHost} \end{tabular} & 68.6   & 84.36 & 68.6 \\ \hline
\begin{tabular}[c]{@{}l@{}} {\bf VFIO} \end{tabular}  & 90.48  & 85.08 & 90.48 \\ \hline
\begin{tabular}[c]{@{}l@{}} {\bf OPTI}\end{tabular}    & 199.76 & 0.24  & 199.76 \\ \hline \hline
\begin{tabular}[c]{@{}l@{}}In OPTI Guest\end{tabular} & 0.66   & 81.7  & -- \\ \hline
\begin{tabular}[c]{@{}l@{}}In DTID Guest\end{tabular} & 0.68   & 82.9  & -- \\ \hline
\end{tabular}
\caption{Comparison of CPU Utilization. When the guest
generates the TCP outgoing traffic, we measure the CPU
utilization in the host. We also measure the CPU utilization
in the OPTI and DTID guest. The total CPU utilization is 200\%
for two working cores. \%User is the \%CPU time consumed in
the user mode, \%System is the \%CPU time consumed in the
kernel mode and \%Guest is the \%CPU time consumed by the
guest. \%Idle is not shown.}
\label{tab:cpu_utilization_40gbps}
\end{table}
}




\begin{table}
\renewcommand{\arraystretch}{1.2}
\small
\begin{center}
\begin{tabular}{|c|c|c|c|c|} \hline
{\bf Exit Cause} & {\bf VHOST } & {\bf VFIO} & {\bf OPTI} & {\bf DTID} \\ \hline
 {\bf HLT Inst}    & 4362  & 79765 & 0    & 0    \\ \hline
 {\bf EPT Fault}  & 55071 & 0     & 0    & 0    \\ \hline
{\bf Interrupt}   & 15702 & 219   & 498  & 1    \\ \hline
\end{tabular}
\end{center}
\vspace{-0.1in}
\caption{Comparison of number of VM exits per second when the evaluated 
configurations send out TCP traffic over a
40Gbps Infiniband link using iPerf.}
\label{tab:vm_exit}
\vspace{-0.1in}
\end{table}

Because the number of VM exits significantly impacts the CPU utilization, 
let's first examine the number of VM exits per second for each of the five 
evaluated configurations. 
Table \ref{tab:vm_exit}  lists the number of VM exists per second a HaaS VM experiences 
when it sends out TCP traffic over an Infiniband link in one of the four virtualized 
configurations. 
The VHOST configuration incurs a VM exit whenever a HaaS VM executes an HLT instruction,
accesses a NIC and triggers an EPT fault, or gets interrupted. 

In contrast, a HaaS VM running in the VFIO configuration does not trigger any VM exit when
accessing a NIC or receiving a NIC interrupt.
The only interrupts that cause a VM exit for such a HaaS VM are internal interrupts, such
as timer and IPI interrupts.
This is why the numbers of VM exits per second due to EPT faults and interrupts  for the VFIO 
configuration are significantly lower than those of the VHOST configuration.
On the other hand, the number of VM exits per second due to HLT instructions for the VFIO configuration
is drastically larger than that of the VHOST instruction, because 
the percentage of run time spent inside a HaaS VM running in the VFIO configuration 
is higher than that inside a HaaS VM running in the VHOST configuration.
When a HaaS VM occupies the CPU longer, it is more likely for the VM to be idle and issue HLT instructions.

 Compared with the VFIO configuration, the OPTI configuration further eliminates VM exits due to HLT instructions.
 The only VM exits for a HaaS VM running in the OPTI configuration are caused by local interrupts.
 In our test, because both the hypervisor and the test HaaS VM set a timer 
 resolution of 4 msec or 250Hz and they expiration times
 are not synchronized, about 500 hardware timer interrupts are generated every second.
 These interrupts are all delivered through the hypervisor to the HaaS VM. That's why the OPTI configuration 
 experiences roughly 498 VM exits per second, most of which are attributed to timer interrupts.
 In the VFIO configuration , because of the large number of VM exits due to HLT instructions, a timer interrupt could arrive when the hypervisor is in control of the interrupted CPU, and triggers no additional VM exit.
 This is why the VFIO configuration's number of  VM exits  due to interrupts (219) is substantially lower than 500. 
  
In the DTID configuration, the only interrupt that triggers a VM exit for a HaaS VM is IPI interrupt. 
During our test, because IPIs do not occur frequently,  the resulting interrupt rate becomes very low, about 1 per second. 



\mycomment{
we show the most relevant VM exits
that has the significant impact on the CPU utilization. With
the NIC assignment, the number of HLT exits per second
increases from 4362 to 79765. Both the number of VM exits per
second due to the EPT misconfiguration and external interrupts
goes down. After disabling the HLT exit, we observes the
increases number of VM exits frequency due to the timer
interrupts. This is due to the time-slice expiration from both
the host and guest. In our experiment, the host and guest has
the time-slice of 4ms which triggers 250 timer interrupts per
second. We examine the VM-exit qualification and find 498
numbers of VM exits per second are due to the timer
interrupts. This matches what we have expected with the
sampling error. In contrast, if the HLT exiting is not
disabled, some of the host timer interrupts are hidden by the
time when the host is emulating HLT instruction. Furthermore,
the timer interrupts are directly delivered to the guest
without causing any VM exit.
}
\begin{table}
\renewcommand{\arraystretch}{1.2}
\small
\begin{center}
\begin{tabular}{|c|c|c|c|c|c|} \hline
{\bf } & {\bf Host } & {\bf Host } & {\bf Guest} & {\bf Guest} & {\bf Total}\\ 
{\bf } & {\bf  User } & {\bf System } & {\bf  User} & {\bf System} & \\ \hline
 {\bf Bare-metal}    & 0.64\%   & 80.98\% & NA & NA & 81.62\%\\ \hline
 {\bf VHOST} & 68.32\%   & 83.85\% & 0.23\% & 31.20\% & 115.28\%\\ \hline
{\bf VFIO}   & 83.35\%  & 99.25\% & 0.26\% & 34.92\% & 134.43\% \\ \hline
 {\bf OPTI}  & 199.76\% & 0.24\%  & 0.65\% & 81.62\% & 82.51\% \\ \hline 
\end{tabular}
\end{center}
\vspace{-0.1in}
\caption{Comparison of the total CPU utilization percentage and its breakdown 
when the evaluated configurations send out TCP traffic over a
40Gbps Infiniband link at full speed using iPerf. The total is 200\% because two CPU cores are used.}
\label{tab:cpu_utilization_40gbps}
\vspace{-0.1in}
\end{table}

We measured the CPU utilization percentage by the host OS ({\em Host System}),
by the HaaS VM ({\em Host User}), by the kernel of the HaaS VM ({\em Guest System})
and by the user programs of the HaaS VM ({\em Guest User}).
Because  the Bare-metal configuration is not virtualized, its Host User percentage is due to 
the user-level test program, in this case iPerf, and it does not have Guest System or 
Guest User percentage.
The total CPU utilization is equal to $Guest System + Guest User + Host System$.
As shown in Table~\ref{tab:cpu_utilization_40gbps}, 
the Bare-metal configuration consumes the least CPU resource,  
the OPTI configuration's CPU resource consumption comes very close to that of the Bare-metal configuration,
and both the VHOST and VFIO configuration consume at least 40\% more CPU resource in comparison.
%compared with the Bare-metal configuration, 
%the VHOST and VFIO configuration both have a higher Host System percentage because of additional 
%processing costs associated with VM exits and entries that result from HLT instructions, EPT faults and interrupts.

Even though the VFIO configuration enables direct access to and interrupt delivery for NICs,
surprisingly, the total CPU utilization of the VFIO configuration is actually higher than that of 
the VHOST configuration. The main cause is the large number of VM exits due to HLT instructions in the VFIO configuration, as shown in Table~\ref{tab:vm_exit}. When KVM processes emulates an HLT instruction, it uses a busy waiting loop, and the CPU cycles burned by these busy waiting loops bump up the Host System mode's CPU utilization and eventually the total CPU utilization of the VFIO configuration.


%The VHOST configuration's Guest User percentage is significantly higher than 
%that of  the Bare-metal configuration, because the kernel of the HaaS VM is responsible for {\em virtual} NIC 
%interaction and interrupt processing, and TCP protocol processing.
%The VFIO configuration's Guest User percentage is significantly higher than 
%that of  the VHOST configuration, because the kernel of the HaaS VM is responsible for {\em physical} NIC 
%interaction and interrupt processing, and TCP protocol processing.

In the OPTI configuration, the Host User percentage is close to 0\%, indicating that the hypervisor is almost 
completely out of the picture. Although the Host User percentage is 199.76\%, the sum of the Guest User and
Guest System percentage is only 82.27\%. The gap between the two is due to HLT instructions.
That is, when a HaaS VM becomes idle and issues HLT instructions, these instructions put the VM in a lower power
idle mode. However, to the hypervisor, the HaaS VM remains active because it takes up the CPU.
In the end, the fact that the Guest User and Guest System percentage of the OPTI configuration are almost 
identical to the Host User and Host System percentage of the Bare-metal configuration suggests that \na is successful in enabling a HaaS server
to perform very closely to a bare-metal server.

\mycomment{
Although the outgoing TCP throughput is about 37.4Gbps for the
guest using the vHost as the backend driver or assigned NIC,
the guest using the assigned NIC has higher CPU utilization.
The VFIO configuration consumes $175.56/200$ of total CPU
utilization, while the vHost configuration consumes
$152.96/200$ of total CPU utilization. This is due to the
higher number of HLT instruction issued by the guest using the
assigned NIC. The NIC assignment by VT-d allows the guest to
handle the network interrupts directly without a VM exit. At
the same time, it keeps the guest on its CPU for longer time.
The guest issues the HLT instruction more frequently and waits
for its network I/O, when it is idle. The vCPUs burn CPU
cycles when they poll for sometime before executing the HLT
instruction. The vCPU polling mechanism keeps the CPU busy
leading to higher CPU utilization.

To reduce the CPU utilization in host, we disable the HLT
exits by modifying the VMCS structure in KVM. To avoid other
system processes to compete with guest, the vCPUs are pinned
on isolated CPUs. As shown in
\ref{tab:cpu_utilization_40gbps}, we notice that disabling HLT
exits along with dedicating cores to guest, reduces the CPU
utilization in system mode to $0.24/200$ and increases the CPU
Utilization in guest mode to $199.76/200$. It indicates that
the guest occupies the CPUs for most of the time.
}


\mycomment{
\begin{table}[tbp]
\begin{tabular}{lllll}
\hline
& \begin{tabular}[c]{@{}l@{}}Guest\\ + vHost\end{tabular} & \begin{tabular}[c]{@{}l@{}}Guest\\ + VFIO\end{tabular} & \begin{tabular}[c]{@{}l@{}}OPTI\\ Guest\end{tabular} & \begin{tabular}[c]{@{}l@{}}DTID\\ Guest\end{tabular}\\ \hline
HLT                & 4362  & 79765 & 0    & 0    \\ \hline
EPT Misconfig.     & 55071 & 0     & 0    & 0    \\ \hline
External Interrupt & 15702 & 219   & 498  & 1    \\ \hline
%Preemption Timer   & 406   & 271   & 601  & 0    \\ \hline
%IO Instruction     & 18    & 19    & 18   & 19   \\ \hline
%MSR Read           & 2     & 2     & 2    & 2    \\ \hline
%MSR Write          & 2248  & 3919  & 4495 & 2499 \\ \hline
%Pause Instruction  & 1266  & 0     & 0    & 0    \\ \hline
%Pending Interrupt  & 273   & 0     & 0    & 0    \\ \hline
%Total              & 79438 & 84195 & 5614 & 2521 \\ \hline
\end{tabular}
\caption{Comparison of VM Exits per Second Among Busy Guests.
The VM exits are recorded when the guest generates the TCP
outgoing traffic.}
\label{tab:vm_exit}
\end{table}
}



\vspace{-0.1in}
\subsection{Direct Interrupt Delivery Efficiency}
\vspace{-0.05in}
% DID efficiency.
% - Describe what we have done.
% - Include tables and figures.
% - Evaluate the performance and see if it match our goal.

\figw{cyclictest}{10}{Cumulative Probability Distribution of
Timer-Interrupt Latency.}

\begin{table}[tbp]
\begin{tabular}{|l|l|l|l|l|}
\hline
& IB & CORE & \begin{tabular}[c]{@{}l@{}}NIC\\ INTR\end{tabular} & \begin{tabular}[c]{@{}l@{}}TMR\\ INTR\end{tabular} \\ \hline
\multirow{2}{*}{HOST} & \multirow{2}{*}{37.39} & 0 & 118592 & 256 \\ \cline{3-5}
&  & 1 & 0 & 348 \\ \hline
\multirow{2}{*}{OPTI GUEST} & \multirow{2}{*}{37.37} & 0 & 123024 & 256 \\ \cline{3-5}
&  & 1 & 0 & 347 \\ \hline
\end{tabular}
\caption{Comparison of Network Interrupts Between Host and
Guest. The interrupts are reported as the average number of
interrupts per second. OPTI guest uses the assigned NIC with
the CPU optimization. IB: 40Gbps infiniband. NIC: network
interface card. INTR: interrupt. TMR: timer.}
\label{tab:network_interrupts}
\end{table}

\begin{table}[tbp]
\begin{tabular}{|l|l|l|l|l|l|}
\hline
& IB & CORE & \begin{tabular}[c]{@{}l@{}}NIC\\ INTR\end{tabular} & \begin{tabular}[c]{@{}l@{}}SPU\\ TMR\\ INTR\end{tabular} & \begin{tabular}[c]{@{}l@{}}TMR\\ INTR\end{tabular} \\ \hline
\multirow{2}{*}{\begin{tabular}[c]{@{}l@{}}DTID GUEST \end{tabular}} & \multirow{2}{*}{37.35} & 0 & 110856 & 100114 & 255 \\ \cline{3-6}
&  & 1 & 0 & 3933 & 348 \\ \hline
\end{tabular}
\caption{Analysis of Spurious Timer Interrupts. The interrupts
are reported as the average number of interrupts per second.
DTID guest uses the assigned NIC with the CPU optimization and
DTID enabled. IB: 40Gbps infiniband. NIC: network interface
card, INTR: interrupt, SPU TMR INTR: spurious timer
interrupt.}
\label{tab:spurious_timer_interrupt}
\end{table}

In this experiment, we demonstrate that the guest improve its
timer interrupt latency using the DTID mechanism. Typically,
the host virtualizes the timer and the time interrupt by its
high-resolution timer subsystem and the delivery of virtual
timer interrupts. DTID reduces the overheads by directly
delivering the timer interrupts and allowing the guest to
configure TMICT directly. We measure the timer-interrupt
latency using the cyclictest~\cite{cyclictest} in (a) host,
(b) guest and (c) DTID guest. For (b) and (c), we also apply
our CPU optimization. (c) has DTID enabled whereas (b) has
DTID disabled. We also show the additional cost to handle the
spurious timer interrupts in the DTID guest.

To estimate the cumulative probability distribution, we
collect the timer-interrupt latency by
cyclictest~\cite{cyclictest}. It measures the latency by
comparing the time taken the sleep and wakeup time. We run the
cyclictest with the sleep time of 200$\mu$s for $10^7$
iterations. The main thread of cyclictest runs on the core
which is not being evaluated, whereas the test thread runs on
the experiment core. We use the top 99\% of data to estimate
the CDF. In Figure~\ref{fig:cyclictest}, we observe that the
median of interrupt latency for the host, DTID guest and
unmodified guest are 1.2, 2.46 and 13.93 $\mu$s respectively.
Using the DTID and our CPU optimization, we improve the timer
interrupt latency by 11.47$\mu$s.

We measure the expected frequency of timer and network
interrupts, when running the iperf benchmark over the 40Gbps
infiniband. Both the host and guest uses two cores. One core
handles the network interrupts, while the other core runs the
iperf benchmark. In Table~\ref{tab:network_interrupts}, the
host and guest both receive the expected frequency of timer
and network interrupts and saturate the 40Gbps link.
Nonetheless, DTID guest needs to handle the spurious timer
interrupt before processing the network interrupts. This is
the addition CPU overhead. In
Table~\ref{tab:spurious_timer_interrupt}, the DTID guest
matches the baremetal network bandwidth performance, while
handling the additional spurious timer interrupts. Since the
PIR timer-interrupt is almost always set, for each network
interrupt, there is a spurious timer interrupt.  The frequency
of network interrupt and spurious timer interrupts are 110856
and 100114 respectively. The DTID algorithm simply ignores all
the spurious timer interrupts. In
Table~\ref{tab:network_performance} and
Table~\ref{tab:cpu_utilization_40gbps}, the DTID guest is able
to match the baremetal network bandwidth and latency, while
having 1.2\% additional CPU overhead to handle the spurious
timer interrupts. The spurious interrupt occurs on the core
running the iperf benchmark, but there is no network
interrupts. Since the spurious timer interrupt also happens
after the VM entry, the perf analysis suggests it is due to
other type of VM exits such as IO instructions, MSR writes or
IPIs.

% TODO: atopsar and micro-benchmark
Furthermore, We measure the overhead of handling spurious
interrupt. It takes --$\mu$s, while the typical handling of
timer interrupt is --$\mu$s. The data suggests our algorithm
works efficiently to deliver the timer interrupt and ignore
the spurious timer interrupts.

% DTID scalability parallel processing Kernbench
We are able to scale the DTID algorithm to all the 9 cores,
while the host has 1 dedicated core. Each vCPU receives the
expected number of timer interrupts. When the DTID guest uses
the periodic timer of 250Hz, each vCPU receives around 250
interrupts per second. To see how well the DTID guest performs
the parallel processing, we run Kernbench~\cite{kernbench} or
PARSEC~\cite{bienia:2008} benchmark. The result shows ....


\vspace{-0.1in}
\subsection{Parallel Application Performance}
\vspace{-0.05in}
% Multi-threaded Computation Performance

\figw{parsec_barchart}{8.4}{
Slowdown of PARSEC benchmark programs for VANILLA and DID guests
compared to bare-metal.
%\textbf{VANILLA}: unmodified guest has the configuration without our optimizations.
}

\figw{fpspeed_barchart}{8.5}{
Slowdown of floating point computation benchmarks in 
SPEC CPU 2017~\cite{bucek:2018} for VANILLA and 
DID guests compared to bare-metal. 
The x-axis refers to the IDs for the following benchmarks:
\textbf{603}: explosion modeling.  \textbf{607}: relativity.
\textbf{619}: fluid dynamics.  \textbf{621}: weather
forecasting.  \textbf{627}: atmosphere modeling.
\textbf{628}: wide-scale ocean modeling.  \textbf{638}: image
manipulation.  \textbf{644}: molecular dynamics.
\textbf{649}: electromagnetic.  \textbf{654}: regional ocean
modeling.}

\figw{intrate_barchart}{8.5}{
Slowdown of integer computation benchmarks in 
SPEC CPU 2017~\cite{bucek:2018} for VANILLA and 
DID guests compared to bare-metal. 
The x-axis refers to the IDs for the following benchmarks:
%\textbf{VANILLA}: unmodified guest has the configuration
%without our optimizations. 
{500}: perl scripts.
{502}: GCC. 
{505}: route planning.
{520}: OMNeT simulation. 
{523}: XML conversion.
{525}: video compression. 
{531}: chess.
{541}: GO. 
{548}: sudoku. 
{557}: data compression.}

To evaluate how well a HaaS VM performed for concurrent
CPU-bound applications, we measured the percentage of slowdown
or throughput reduction. Two popular thread models were
considered: Pthreads~\cite{lewis:1998} and
OpenMP~\cite{dagum:1998}. Pthreads~\cite{lewis:1998} provided
the finer-grained control over the thread management, while
OpenMP~\cite{dagum:1998} was the industry standard and easier
to scale than Pthreads~\cite{lewis:1998}. To measure the
slowdown, we used SPECspeed 2017 Floating
Point~\cite{bucek:2018} and PARSEC~\cite{lewis:1998}. To
measure the throughput reduction, we used SPECrate 2017
Integer~\cite{bucek:2018}.

In our experiment, the HaaS VM had 8 dedicated cores and 27GB
RAM whereas \na had the two cores, including physical core 0,
and the remaining 5GB RAM. Moreover, the topology of L1/L2/L3
CPU cache was exposed to the VM. The bare-metal was the
baseline and had the same configuration as the VM. We compared
the percent slowdown between the VMs of DID and vanilla
configuration, which did not use our optimizations except the
dedicated cores and cache information. To To measure the
slowdown, we used 8 threads in each in SPEC
CPU~\cite{bucek:2018} benchmark programs. To measure the
throughput reduction, we used 8 processes in each SPEC
CPU~\cite{bucek:2018} and PARSEC~\cite{lewis:1998} benchmark
programs.

For the cases of OpenMP~\cite{dagum:1998} in
Figure~\ref{fig:fpspeed_barchart}, the maximum performance
slowdown was 3\%. DID improved the performance further and did
not perform worse than the vanilla VM for the wide-scale ocean
modeling. For the cases of Pthreads~\cite{lewis:1998} in
Figure~\ref{fig:parsec_barchart}, most cases had the
percentage slowdown less than 5\% except canneal and
streamcluster. DID reduced the slowdown even further.

Especially for Dedup, DID trimmed 11.45\% of slowdown to
3.85\%. Although the performance of DID in Vips seemed to be
worse than Vanilla, the absolute values
differed less than 0.1 seconds.
For Canneal and Streamcluster,
even after tracking and eliminating all VM exits triggered by EPT violations
and I/O instructions from certain virtual devices
(virtual floppy and CD-ROM), the performance of 
Canneal and Stremcluster did not significantly improve.
% After analyzing the VM Exit reasons, we eliminated
%the two expensive VM exits: (1) I/O instructions and (2) EPT
%violations. To eliminate VM exits triggered by I/O instructions, 
%we removed the virtual devices for floppy disk and CD-ROM, 
%which accounted for about 20 I/O instructions per second. 
%To eliminate VM exits triggered by EPT violation, the HaaS agent in 
%the VM touched almost all guest pages at initiaization time, 
%which forced the \na to populate the corresponding EPT entries ahead of time. 
%These two optimizations reduced the rate of VM exits to around 11 VM exits per
%second. We re-ran the two benchmark programs and observed it
%had only marginal performance improvement. 
We suspect that the performance difference for these two cases 
might be due to VTx-related architecture issues; we are
further investigating the uderlying causes.

%As shown in Figure~\ref{fig:intrate_barchart}, DID improved
%the CPU throughput in all cases which had less than 10\% of
%throughput reductions. For 


\vspace{-0.1in}
\subsection{Live Migration Performance}
\vspace{-0.05in}
% Migration performance.
% - Describe what we have done.
% - Include tables and figures.
% - Evaluate the performance and see if it match our goal.

\figw{seamless_migration}{10}{Place holder for seamless
migration. Top: Guest sends the outgoing traffic. Bottom:
guest receives the incoming traffic.}

\begin{table}[tbp]
\begin{tabular}{|l|l|l|l|}
\hline
& UNPLUG & PLUG & OPTI. PLUG \\ \hline
UDP SENDER & 4.6 & 300 & 0 \\ \hline
UDP RECEIVER & 1.8 & 300 & 0 \\ \hline
\end{tabular}
\caption{Network Downtime}
\label{tab:migration_network_downtime}
\end{table}

In this experiment, we show the network downtime when
switching between the assigned and Virtio network device. We
also demonstrate the average number of missed local timer
interrupt in guest, when we disable the DTID.

In Table~\ref{tab:migration_network_downtime}, when switching
from the assigned to Virtio network device, the network is --
and -- ms for the UDP sender and receiver respectively. When
switching back from the Virtio to assigned network device, the
network is -- and -- ms for the UDP sender and receiver
respectively. After we hide the overhead of assigned NIC
creation during the VM migration, the network downtime is
reduced to -- and -- ms for the UDP sender and receiver
respectively.

When disabling the DTID, the timer interrupts received by the
guest per seconds matches the expected frequency of timer
interrupt received by the unmodified guest. Since KVM delivers
the virtual interrupts with or without DTID, the guest can
still receives its interrupt during the transition. The longer
the transition takes, the later the migration starts. The host
communicates with the guest by the TCP transmission. We expect
most of the transition time is due to the packet transmission
and processing. The average transition time is -- $\mu$s.

% TODO: Analysis of DTID algorithm in the guest shows the
% guest does not miss the next timer interrupt.

